\chapter{Pokyny k uživatelskému testování}\label{appendix:ui-testing-instructions}

%---------------------------------------------------------------
\section*{Úvod}
%---------------------------------------------------------------

Během testování budete dostávat psané instrukce, které se budete snažit splnit. Snažte se prosím komentovat své postupy, myšlenky a interakce s aplikací. Jakákoli intuice nebo očekávání, jak byste danou instrukci řešili, kde byste očekávali obrazovku nebo prvek aplikace, cokoli Vás napadne.

Pamatuje, že jakýkoli průběh, ať už splní nebo nesplní cíl instrukce, nikdy není vaší chybou, ale chybou rozhraní aplikace. Vaše dojmy, jakožto uživatele, který s aplikací interaguje poprvé, jsou v tomto ohledu zásadní pro zhodnocení uživatelského rozhraní aplikace.

Během testu bude nahrávána Vaše interakce s aplikací a bude s Vámi moderátor testu. Moderátor by Vám neměl radit v interakci s aplikací, pouze by vám měl pomoct s průběhem testu, či s vyjasněním zadání.

Kdykoli Vám bude něco v rozhraní připadat zvláštní, neintuitivní, budete něčím překvapeni nebo nebudete vědět, jak dané zadání splnit, zkuste vždy tuto situaci popsat.

Jakmile budete cíl daného scénáře testu považovat za splněný, informujte o tom moderátora.

%---------------------------------------------------------------
\section*{Dotazník před testem}
%---------------------------------------------------------------

\begin{itemize}
\item Jaké je vaše jméno?
\item Kolik Vám je let?
\item Používáte často mobilní aplikace? A s jakou platformou máte nejvíce zkušeností (iOS/Android)?
\item Používáte nějaké aplikace pro měření odpracovaného času? Pokud ano, jaké?
\end{itemize}

%---------------------------------------------------------------
\section*{Scénáře testu}
%---------------------------------------------------------------

%---------------------------------------------------------------
\subsection*{1. Tvorba uživatelského účtu}
%---------------------------------------------------------------

Na mobilním telefonu spusťte aplikaci \emph{Trackee} a pokuste se pro sebe vytvořit nový uživatelský účet. E-mail a heslo můžete zvolit jaké chcete.

%---------------------------------------------------------------
\subsection*{2. Vytvoření nového klienta}
%---------------------------------------------------------------

Vytvořte v aplikaci 2 nové klienty, pro které budete dále moci vytvořit projekty. Pro prvního klienta použijte název \emph{Matee Devs} a pro druhého klienta použije název \emph{FIT ČVUT}.

%---------------------------------------------------------------
\subsection*{3. Vytvoření nového projektu}
%---------------------------------------------------------------

Vytvořte v aplikaci 2 nové projekty s následujícími vlastnostmi:

\begin{itemize}
\item{První projekt}
  \begin{itemize}
  \item{\textbf{Klient}: Matee Devs}
  \item{\textbf{Název}: Unikátní půllitr}
  \item{\textbf{Typ projektu}: Work}
  \end{itemize}
\item{Druhý projekt}
  \begin{itemize}
  \item{\textbf{Klient}: FIT ČVUT}
  \item{\textbf{Název}: Diplomová práce}
  \item{\textbf{Typ projektu}: School} 
  \end{itemize}
\end{itemize}

%---------------------------------------------------------------
\subsection*{4. Spuštění časovače}
%---------------------------------------------------------------

Spusťte časovač pro měření odpracovaného času. Zvolte, aby časovač měřil práci pro projekt \emph{Unikátní půllitr} (klient \emph{Matee Devs}) a přidejte popis \emph{Code reviews}.

%---------------------------------------------------------------
\subsection*{5. Změna začátku časovače}
%---------------------------------------------------------------

Předpokládejte, že jste časovač spustili až později, než jste na této práci skutečně začali pracovat. Pokuste se proto posunout začátek, od kdy časovač měří, o 2 hodiny dříve, než je nyní.

%---------------------------------------------------------------
\subsection*{6. Zastavení časovače}
%---------------------------------------------------------------

Zastavte měření časovače, aby se Vám uložila do teď měřená práce do historie záznamů.

%---------------------------------------------------------------
\subsection*{7. Manuální přidání časového záznamu}
%---------------------------------------------------------------

Předpokládejte, že jste si vzpomněl(a), že jste včera od \emph{11:03} do \emph{13:49} pracovali na projektu \emph{Diplomová práce} (klient \emph{FIT ČVUT}), ale tuto skutečnost jste zapomněli zaevidovat do aplikace. Vytvořte pro to manuálně nový časový záznam podle těchto údajů a přidejte mu popis \emph{Analýza}, aby nyní tento záznam byl součástí historie záznamů.

%---------------------------------------------------------------
\subsection*{8. Úprava projektu}
%---------------------------------------------------------------

Upravte projekt \emph{Diplomová práce} (klient \emph{FIT ČVUT}) tak, aby nový název projektu byl \emph{Diplomová práce a obhajoba}. Zkontrolujte, že historie časových záznamů reflektuje tuto změnu.

%---------------------------------------------------------------
\subsection*{9. Odhlášení a přihlášení na testovací účet}
%---------------------------------------------------------------

Odhlašte se ze svého účtu a přihlašte se na testovací účet, který již obsahuje řadu klientů, projektů a časových záznamů. Jako e-mail použijte \emph{usertesting@trackee.app} a jako heslo použijte \emph{nejlepsiappka}.

%---------------------------------------------------------------
\subsection*{10. Odstranění časového záznamu}
%---------------------------------------------------------------

Smažte z historie záznamů záznam z \emph{8. února} s popisem \emph{Workshop na dojení zákazníků}.

%---------------------------------------------------------------
\subsection*{11. Export historie do CSV souboru}
%---------------------------------------------------------------

Vytvořte integraci pro export historie do CSV souboru. Název integrace použijte jaký chcete. Exportuje poté historii v intervalu od \emph{1.1.2024} do \emph{30.4.2024}. Výsledný soubor zkontrolujte v aplikaci \emph{Numbers}.

%---------------------------------------------------------------
\subsection*{12. Odstranění klienta}
%---------------------------------------------------------------

Předpokládejte, že jste si v exportované tabulce všimli záznamů projektu \emph{Vyšetřování podezřelých aktivit kolegy Hanžlíka} a že tyto záznamy v tabulce nechcete, protože vaše spolupráce s klientem \emph{Fitify} je tajná. Smažte proto radši celého klienta \emph{Fitify} úplně a znovu vyexportujte data do CSV souboru od \emph{1.1.2024} do \emph{30.4.2024} a v aplikaci \emph{Numbers} ověřte, že se zde zmíněné záznamy již nenacházejí.

%---------------------------------------------------------------
\subsection*{13. Tvorba automatizace pro spuštění časovače}
%---------------------------------------------------------------

Pomocí aplikace \emph{Zkratky} vytvořte novou automatizaci, která automaticky spustí časovač pro projekt \emph{Diplomová práce} (klient \emph{FIT ČVUT}) s popisem \emph{Obhajoba}, pokud dorazíte na místo \emph{Thákurova 9, Praha 6}. Zkuste zkratku spustit a vyzkoušet, že se časovač opravdu spustil.

%---------------------------------------------------------------
\section*{Dotazník po testu}
%---------------------------------------------------------------
\begin{itemize}
\item Jaký máte celkový dojem z používání aplikace? Co se Vám líbilo a co se Vám nelíbilo?
\item Stalo se Vám u nějakého bodu, že jste si nevěděl(a) rady s tím, jak ho splnit? Čím to podle Vás bylo a co by tomu pomohlo?
\item Používal(a) byste tuto aplikaci?
\end{itemize}
