Výslednou aplikaci a její uživatelské rozhraní je potřeba důkladně otestovat. Tato kapitola popisuje implementaci automatických testů a scénáře testování s~reálnými uživateli. Výsledky z~testování jsou poté zhodnoceny.

%---------------------------------------------------------------
\section{Automatické testování}
%---------------------------------------------------------------

Pro účely automatizovaného testování se u~Apple platforem používá knihovna \emph{XCTest} \cite{xctest}. Jak již bylo zmíněno, \emph{Clean architecture} ja navržená tak, aby se dala dobře otestovat jednotkovými testy, které jsou popsány v~následující sekci.

Dalšími možnostmi testování jsou UI testy a testy výkonu. Aplikace Trackee implementaci těchto typů testů neobsahuje. Jejich implementace může být podnětem pro budoucí vylepšení aplikace.

%---------------------------------------------------------------
\subsection{Jednotkové testování}
%---------------------------------------------------------------

V~aplikaci jsou jednotkovými testy pokryty všechny \emph{View Modely}. Pro každý z~nich je v~modulu, ke kterému patří, soubor, který daný \emph{View Model} testuje. Např. pro \texttt{LoginViewModel.swift} bude testovací soubor \texttt{LoginViewModelTests.swift}.

\emph{View Model} testy jsou navržené tak, aby testovaly každý jednotlivý \emph{Intent}, pokud existuje vhodná možnost, jak test provést. Některé intenty vhodně otestovat nelze – například takové, které otevírají nějaké systémové dialogy, které nemají žádný výstup pro \emph{View Model} aplikace. Daný test tedy vždy ověřuje, zda se stav \emph{View Modelu} po aplikování \emph{Intentu} změnil podle očekávání.

Jednotkové testy \emph{View Modelů} využívají \emph{Mocky} pro \emph{Use Cases}, tedy určité zjednodušené náhrady těchto \emph{Use Cases}, u~kterých si každý test může definovat, co bude daný \emph{Use Case Mock} vracet za hodnotu. Toto je klasický přístup k~jednotkovému testování – testujeme zde \emph{View Modely}, nikoli \emph{Use Cases}.

Předpokládejme tedy, že máme \emph{View Model} se strukturou ve výpisu \ref{code:tested-vm-structure}. Struktura testu poté bude vypadat jako ve výpisu \ref{code:vm-test-sctructure}. Struktura obsahuje následující:
\begin{itemize}
\item\texttt{flowController} – \emph{Mock} pro \emph{FlowController}, tedy náhrada reálného controlleru v aplikaci. Tento \emph{Mock} slouží k~tomu, abychom mohli otestovat, jaká hodnota byla vložena do posledního volání funkce \texttt{handleFlow(\_:)} a kolikrát byla tato funkce zavolána. Toto slouží k~otestování toho, zda se bude apikace snažit uživatele navigovat do předpokládané destinace.
\item\emph{Mocky} pro \emph{Use Cases} – zde vytváříme instance \emph{Mocků} pro \emph{Use Cases}, které jsou v~daném \emph{View Modelu} používány. Je potřeba to dělat takto ručně, protože kdybychom použili registrování pomocí \texttt{Container.shared.registerUseCaseMocks()}, tak bychom nemohli v~testech měnit hodnoty, které dané \emph{Use Cases} budou vracet.
\item\texttt{createViewModel()} – tato funkce se bude volat na začátku každého testu, aby získal instanci pro otestování. Pokud konstruktor \emph{View Modelu} vyžaduje nějaké parametry, můžou být předány jako parametr této funkce, nebo zde můžou mít nějakou definovanou hodnotu. V~této funkci se také registrují \emph{Mocky} pro \emph{Use Cases} definované výše.
\item Testovací funkce pro jednotlivé intenty – funkce, které mají v~názvu prefix \texttt{test}, tak se automaticky považují ze testovací funkce a spouštějí se během testu. Každý vhodně otestovatelný intent zde tedy bude mít vlastní funkci, případně více funkcí, pokud lze otestovat úspěch, neúspěch, různé formy úspěchu, a podobně. Tyto funkce dodržují strukturu given-when-then, tedy rozdělení do tří částí (předdefinované hodnoty a konstanty, interakce, ověření).
\end{itemize}

\begin{listing}
\caption{Struktura View Modelu pro testování}\label{code:tested-vm-structure}
\begin{minted}{Swift}
final class SomeViewModel: BaseViewModel, ViewModel, ObservableObject {
	
    // ...
	
    @Injected(\.someUseCase) private var someUseCase
	
    // ...
	
    enum Intent {
        case doThis
        case doThat
    }
	
    // ...
}
\end{minted}
\end{listing}

\begin{listing}
\caption{Struktura View Model testu}\label{code:vm-test-sctructure}
\begin{minted}{Swift}
@MainActor
final class SomeViewModelTests: XCTestCase {

    private let flowController = FlowControllerMock<SomeFlow>(
        navigationController: UINavigationController()
    )
    
    private let someUseCaseMock = SomeUseCaseMock(
        executeReturnValue: /* Some default return value */
    )
    
    private func createViewModel(
        someOptionalParam: SomeType
    ) -> SomeViewModel {
        Container.shared.someUseCase.register { self.someUseCaseMock }
	
        return SomeViewModel(
            someParam: someOptionalParam,
            flowController: flowController
        )
    }
    
    // MARK: - Tests
    
    func testDoThis() async {
        // given
        let vm = createViewModel()
        // ... some constants and given values
        let someConstant = ...
    	
        // when
        vm.onIntent(.doThis)
        // ... some interactions
        await vm.awaitAllTasks()
		
        // then
        // ... assertions
        XCTAssertEqual(vm.state.someValue, someConstant)
    }
    
    func testDoThat() async {
        // ...
    }
}
\end{minted}
\end{listing}

Jednotkové testy \emph{View Modelů} testují byznysovou prezentační vrstvy. Pro otestování dalších vrstev je možné v multi-platformním modulu testovat \emph{Use Cases}, \emph{Repositories} a \emph{Sources}. V~backendové části je také možné implementovat jednotkové testy pro \emph{Repositories}, \emph{Sources}, ale i~pro \emph{Routes}. Implementace testů pro tyto vrstvy může být podnětem pro budoucí vylepšení aplikace.

%---------------------------------------------------------------
\section{Uživatelské testování}
%---------------------------------------------------------------

Pro účely uživatelského testování byly navrženy scénáře, podle kterých se testeři mají řídit. Tyto scénáře vycházejí z~navržených případů užití v~sekci \ref{features} a měly by pokrývat většinu funkcionalit aplikace, kterých uživatel může využít. Povinnými účastníky během testování budou moderátor a uživatel. Moderátor bude uživateli dávat instrukce a dokumentovat, jak uživatel s~mobilním telefonem interaguje. Uživatel se bude snažit instrukce plnit a bude při tom nahlas popisovat své akce s~telefonem (co vidí, na co se chystá kliknout, co očekává, že se stane, atd.), aby usnadnil následnou analýzu celé situace. 

Uživatelského testování se zúčastnilo 5 testerů. Záznamy z~testování lze nahlédnout v~\cite{ui-testing-playlist}. Přehled testerů lze nahlédnout v~tabulce \ref{table:testers}. Instrukce, podle kterých se testeři během testu řídili, lze nahlédnout v~příloze \ref{appendix:ui-testing-instructions}. Protokol z~testování lze nahlédnout v~příloze \ref{appendix:ui-testing-protocol}.

Uživatelské testování probíhalo na produkčním prostředí aplikace Trackee, tedy na prostředí, které by mělo cílit na reálné uživatele a které poskytuje omezené možnosti ladění, jako jsou například omezené popisy neznámých chyb. Aplikace, kterou testeři testovali, také komunikovala s~instancí backendu na lokální síti, a nikoli s~instancí nasazenou v~aplikaci \emph{Railway} (více informací v~sekci \ref{backend-deployment}). Lokální instance byla zvolena proto, aby byli testeři během testování odstíněni od prodlev, které jsou způsobeny velkou vzdáleností od instance v~aplikaci \emph{Railawy}.

\begin{table}\centering
\begin{tabular}{l|c|c|c}
	Tester		& Pohlaví	& Používaný OS	& Používá aplikace pro měření času	\tabularnewline \hline 
 	F. W.		& Muž		& Android       & ANO (Clockify)	                \tabularnewline \hline
	D. K.		& Muž		& iOS	        & ANO (Clockify)                	\tabularnewline \hline
	D. Ž.		& Muž		& Android       & ANO (Clockify)               		\tabularnewline \hline
	E. Č.		& Žena		& iOS	        & NE 	                	        \tabularnewline \hline
	T. S.		& Muž		& Android       & ANO (Clockify, Toggl Track)	    \tabularnewline \hline
\end{tabular}
\vspace{0.5cm}
\caption[Přehled testerů aplikace]{~Přehled testerů aplikace}\label{table:testers}
\end{table} 

%---------------------------------------------------------------
\subsection{Scénář}
%---------------------------------------------------------------

\begin{description}
\item[Tvorba uživatelského účtu:] Každý nový uživatel mobilní aplikace Trackee si bude muset nejprve vytvořit svůj uživatelský účet. V~tomto scénáři bude uživatel instruován k~tomu, aby aplikaci poprvé spustil, vytvořil si nový uživatelský účet pomocí předepsaného e-mailu a hesla a následně se do aplikace pod tímto účtem přihlásil. 
\item[Vytvoření nového klienta:] Po úspěšné registraci bude uživatel instruován, aby si vytvořil 2 nové klienty s~předepsanými názvy.
\item[Vytvoření nového projektu:] Po úspěšném vytvoření klientů bude uživatel instruován, aby vytvořil 2 nové projekty s~předepsanými vlastnostmi.
\item[Spuštění časovače:] Po úspěšném vytvoření projektů dostane uživatel instrukci, aby spustil časovač pro měření odpracovaného času a přiřadil mu předepsaný projekt a popis.
\item[Změna začátku časovače:] Uživatel bude instruován k~tomu, aby u~spuštěného časovače změnil začátek na předepsaný čas.
\item[Zastavení časovače:] Uživatel bude instruován, aby ukončil časovač a uložil časový záznam, který do teď měřil.
\item[Manuální přidání časového záznamu:] Uživatel bude instruován, aby ručně přidal časový záznam s~předepsanými parametry a časem.
\item[Úprava projektu:] Uživatel bude instruován, aby aktualizoval předepsaný projekt s~novými předepsanými vlastnostmi.
\item[Odhlášení a přihlášení na testovací účet:] Uživatel ve svém nově vytvořeném účtě nebude mít dostatek klientů, projektů a časových záznamů, které by odpovídaly dlouhodbějšímu používání aplikace. Pro lepší otestování orientace v~aplikaci bude uživatel instruován, aby se přihlásil na předepsaný testovací účet, který obsahuje větší množství záznamů.
\item[Odstranění časového záznamu:] Uživatel bude instruován, aby odstranil předepsaný časový záznam z~historie.
\item[Export historie do CSV souboru:] Uživatel bude instruován, aby vytvořil integraci pro exportování do CSV souboru a exportoval data do tabulky pro předepsané časové období. Dále bude instruován, aby tuto tabulku otevřel ve vhodné aplikaci, kde si ji může prohlédnout (např. \emph{Numbers}).
\item[Odstranění klienta:] Budeme předpokládat, že si uživatel v~exportované tabulce všimne, že tam má záznamy patřící předepsanému klientovi, které tam mít nechce. Bude tedy instruován, aby klienta smazal a znovu vyexportoval CSV soubor, ve kterém zkontroluje, že žádné záznamy patřící k~tomuto klientovi nejsou.
\item[Tvorba automatizace pro spuštění časovače:] Uživatel dostane instrukci, aby pomocí aplikace \emph{Zkratky} vytvořil novou automatizaci, která zapne časovač s~předepsanými parametry, pokud se uživatel objeví na předepsané poloze. Bude také instruován, aby tuto automatizaci zkusil ručně spustit a v~aplikaci zkontroloval, že časovač opravu běží.
\item[Tvorba automatizace pro vypnutí časovače (odstraněno):] Uživatel bude požádán, aby vytvořil automatizaci, která se ho zeptá, zda nechce zastavit běžící časovač, pokud opustí předepsanou polohu. Zde bude také instruován, aby automatizaci zkusil ručně spustit. Tento scénář byl po několika testech ze seznamu odstraněn, protože bylo zhodnoceno, že je příliš podobný předchozímu scénáři a zároveň přímo netestuje rozhraní samotné aplikace.
\end{description}

%---------------------------------------------------------------
\subsection{Výsledky}\label{ui-testing-results}
%---------------------------------------------------------------

V~protokolu z~průběhu testování (příloha \ref{appendix:ui-testing-protocol}) byly označeny takzvané \emph{kritické} a \emph{důležité poznatky}. Kritické poznatky jsou takové, u~kterých bylo zhodnoceno, že se jedná o~závažnou chybu rozhraní aplikace, která může uživateli závažným způsobem zhoršit zkušenost s~jejím používáním. Důležité poznatky jsou potom takové, které nemusí být nutně nějakou chybou v~rozhraní, ale potenciálním podnětem pro zlepšení rozhraní. Tato sekce rozebírá a analyzuje tyto dva druhy poznatků. Následující sekce postupně rozebírají kritické a důležité poznatky, seřazené podle toho, jaká jim byla přidělena priorita (první má největší prioritu).

%---------------------------------------------------------------
\subsubsection{Kritické poznatky}
%---------------------------------------------------------------

Kritické poznatky byly v~průběhu testování zaznamenány hned u~prvního testera (tester F. W.). Vzhledem k~jejich závažnosti byly chyby objevené těmito poznatky opraveny okamžitě. Z~tohoto důvodu se už u~dalších testerů tyto stejné chyby objevit nemohly.

\begin{itemize}
\item\textbf{Opakované ukazování neznámé chyby během registrace.} Testerovi F. W. se po zadání údajů pro registraci nedařilo registraci dokončit, protože po klikání na tlačítko \emph{Registrovat} se neustále ukazovala \emph{Neznámá chyba}. Tato chybová hláška neposkytovala žádný popis toho, co konkrétně by mohlo být špatně, ani návrh na to, jak chybu opravit. Tester musel opakovaně zkoušet vyplňovat e-mail v~jiných formátech. Nakonec se mu přihlášení podařilo a hádal, že chyby asi byly kvůli špatnému formátu e-mailu nebo slabému heslu.
\end{itemize}

%---------------------------------------------------------------
\subsubsection{Důležité poznatky}
%---------------------------------------------------------------

\begin{itemize}
\item\textbf{Obtížná klikatelnost prvků v~navigační liště.} Většina testerů musela opakovaně klikat na tlačítka \emph{Uložit} nebo \emph{Exportovat} v~navigační liště, protože se jim do nich nedařilo trefit.
\item\textbf{Redundantní potvrzení po výběru klienta nebo projektu.} Někteří testeři zmínili, že jim přijde redundantní klikat na tlačítko \emph{Uložit} po výběru projektu pro ovladač časovače, nebo po výběru klienta pro projekt. Očekávali, že po zvolení projektu nebo klienta se volba uloží automaticky a aplikace se vrátí o~obrazovku zpět.
\item\textbf{Absence možnosti úpravy časového záznamu.} Několik testerů se při různých fázích scénáře snažilo klikat nebo podržet na časový záznam v~historii, pravděpodobně s~očekáváním, že to bude mít nějakou odezvu, jako otevření detailu. Aplikace ale nic takového neimplementuje, pouze swipe-to-delete.
\item\textbf{Počáteční zmatení nad přepínáním ovladače mezi časovač a manuální zadávání.} Někteří testeři byli zpočátku zmateni nad rozhraním ovladače pro přepínání mezi klasickým měřením pomocí časovače a mezi manuálním zadáváním času nového záznamu. Jeden tester zmínil, že by pro manuální zadání času preferoval nějaký dialog místo in-place změny rozhraní časovače, ale poté zmínil, že přepínání nakonec pochopil a že jde možná o~zvyk.
\item\textbf{Absence swipe-to-delete u~některých seznamů – nekonzistence v~rámci aplikace.} Jeden tester zmínil, že byl zmatený z~toho, že při pokusu o~smazání klienta nefungovalo gesto swipe-to-delete, které fungovalo při mazání časových záznamů, a používání tohoto gesta tak bylo v~rámci aplikace nekonzistentní.
\item\textbf{Klávesnice při zadávání e-mailové adresy nenabízí v~základním rozmístění kláves tečku.} Při zadávání e-mailové adresy se zobrazí klávesnice, která ve svém výchozím rozmístění neobsahuje klávesu pro tečku, ale ani například klávesu pro zavináč. Uživatel tak musí dodatečně klikat na klávesu pro zobrazení speciálních znaků.
\item\textbf{Malá klikatelná plocha tlačítka pro smazání.} Jeden tester si všiml, že pro kliknutí na tlačítko pro mazání je potřeba kliknout na text tlačítka, nefunguje kliknutí pouze na pozadí tlačítka.
\item\textbf{Neviditelný kurzor při zobrazení zadaného hesla.} Při zadávání hesla aplikace nabízí možnost zobrazit přímo text hesla, které uživatel zadal, místo bezpečnostního zakrytí tečkami. Při tomto zobrazení ale z~pole zmizí kurzor, přestože do pole lze normálně psát i~nadále.
\item\textbf{Jednotlivé karty aplikace si pamatují zanoření.} Někteří testeři byli zmatení nad tím, že při přepnutí do karty profilu byli zanořeni v~obrazovce, do které se proklikli při jednom z~předchozích scénářů. Očekávali, že kliknutí na karty profilu je dostane do přehledu profilu.
\item\textbf{Absence onboardingu nebo možnosti vytvoření počátečních klientů a projektů.} Pokud účet uživatele neobsahuje žádná data, zpravidla po registraci, tak sice zobrazuje informaci, že uživatel například nemá žádné projekty a že si je může vytvořit v~profilu, ale nenabízí nějaký jednoduchý proklik, který by toto počáteční zadávání dat usnadnil. Někteří testeři zmínili, že by aplikace mohla obsahovat nějakou formu onboardingu, což je obvykle nějaká úvodní nápověda, v~rámci které si uživatel může například právě vytvořit nějaká počáteční data.
\item\textbf{Nedostatečná vizuální indikace běžícího časovače.} Jeden tester zmínil, že by při běžícím časovači přidal tlačítku pro zastavení červené pozadí, jelikož se jedná o~zvyk u~existujících řešení pro měření a správu času.
\item\textbf{Ne příliš výstižný popis tlačítka Uložit.} Jeden tester zmínil, že při vytváření nového klienta nebo projektu by upřednostnil textaci tlačítka jako \emph{Vytvořit} místo \emph{Uložit}.
\item\textbf{Ne příliš výstižný popis tlačítka Smazat.} Jeden tester zmínil, že v~potvrzovacím dialogu pro smazání klienta (tedy i~projektu) by očekával text Smazat místo Ano.
\item\textbf{Přednost popisu záznamu před názvem klienta nebo projektu.} Dva testeři zmínili, že jim přijde, že popis časového záznamu je důležitější, než název projektu nebo klienta.
\item\textbf{Absence potvrzení před smazáním časového záznamu.} Dva testeři zmínili, že je překvapilo, že pro mazání časového záznamu nebyl žádný potvrzovací dialog.
\item\textbf{Údiv nad zařazením exportu do CSV mezi integrace.} Dva testeři zmínili, že jim přijde zvláštní, že je export do CSV souboru zařazen mezi integrace.
\item\textbf{Aplikace se po exportu dat nevrací o~obrazovku zpět.} Jeden tester zmínil, že by po úspěšném exportu očekával, že se aplikace vrátí o~obrazovku zpět.
\item\textbf{Absence indikace spuštěné zkratky v~aplikaci.} Dva testeři zmínili, že by po spuštění zkratky očekávali, že se v~aplikaci objeví nějaká indikace, že byla zkratka provedena.
\item\textbf{Při snaze o~úpravu zadaného hesla se maže celé pole.} Někteří testeři byli podráždění tím, že když potřebovali upravit zadané heslo, tak se při pokusu o~smazání jednoho znaku smazalo celé pole.
\item\textbf{Zapnutí časovače v~aplikaci Zkratky nemá automaticky otevřený dialog pro parametry.} Jeden tester zmínil, že při tvorbě zkratky by čekal, že nabídka parametrů bude otevřená automaticky.
\end{itemize}

%---------------------------------------------------------------
\subsection{Zhodnocení}
%---------------------------------------------------------------

Tato sekce rozebírá poznatky z~uživatelského testování, diskutuje nad tím, čím byly způsobeny a jakým způsobem by případně mělo na jejich základě být upraveno rozhraní aplikace.

Co se týče kritických poznatků, který byl jen jeden, tak jak již bylo zmíněno, byly opraveny ihned po provedení uživatelského testování s~prvním testerem. Jednalo se o~chybu rozhraní, které nezobrazovalo dostatečný popis chyby během registrace. Pokud uživatel zadal nevalidní formát e-mailové adresy nebo příliš slabé heslo, tak se tato chyba uživateli zobrazila pouze jako \emph{Neznámá chyba}, z~čehož by uživatel měl jen velmi nízkou šanci pochopit, v~čem udělal chybu. Tato chyba byla způsobena tím, že aplikace Trackee v~produkčním prostředí neposkytuje technické popisy chyb pro neznámé druhy chyb, tedy takové, které nejsou explicitně detekovány a není jim přidělena vlastní lokalizovaná textace. A to přesně se dělo u~těchto chyb, kdy z~knihovny \emph{Firebase} autentizace přišla nějaká chyba, která sice obsahovala nějaký popis chyby z~knihovny, ale ten aplikace v~produkčním prostředí neukazovala a z~chyby se tak stala \emph{Neznámá chyba}. Byla tedy přidána explicitní detekce chyb pro špatný formát e-mailu, pro příliš slabé heslo a pro pokus o~registraci s~již existujícím e-mailem.

Důležité poznatky byly zhodnoceny následovně:

\begin{itemize}
\item\textbf{Obtížná klikatelnost prvků v~navigační liště:} U~této chyby se lze do jisté míry odvolat na systémové rozhraní iOS, protože způsob, kterým jsou v~aplikaci přidávány tlačítka do navigační lišty, přesně dodržuje způsob navržený Applem a jedná se o~způsob, který sám Apple ve svých aplikací používá. Do navigační lišty je zkrátka nutné kliknout přesně, jinak aplikace dotek zaregistruje mimo tlačítko. Ale vzhledem k~tomu, že se snad žádnému testerovi nepodařilo kliknout na všechny navigační tlačítka na první pokus, tak se jedná o~tak závažný problém, že by se mu budoucí vývoj měl určitě věnovat i~tak. Mohlo by být vyvinuto nějaké úsilí na pokus rozšíření klikatelné plochy v~navigační liště, a pokud by tento postup byl při použití systémové navigační lišty obtížný, mohly by být potvrzovací tlačítka z~navigační lišty zcela odstraněny a nahrazeny primárním tlačítkem přímo vespod obsahu dané obrazovky.
\item\textbf{Redundantní potvrzení po výběru klienta nebo projektu:} Někteří testeři toto přímo zmínili, že jim to přijde zbytečné. A vzhledem k~tomu, že se nejedná o~žádnou nevratnou akci, kterou by uživatel nemohl například při špatné volbě ihned změnit, tak není důvod odmítat změnu rozhraní, ve kterém se při volbě klienta nebo projektu aplikace opravdu automaticky vrátí o~obrazovku zpět a volbu sama uloží. Ve všech případech, kde se tento přístup používá, se lze na výběr ihned vrátit.
\item\textbf{Absence možnosti úpravy časového záznamu:} Během různých úmyslů se testeři snažili na záznam kliknout, ale nic se nestalo. Možnost vytvoření nějakého rozhraní pro úpravu záznamu je tedy určitě podstatným podnětem pro další vývoj aplikace.
\item\textbf{Počáteční zmatení nad přepínáním ovladače mezi časovač a manuální zadávání:} Rozhraní ovladače, konkrétně přepínaní mezi stavem časovače a stavem manuálního zadávání, bylo inspirováno ovladačem časovače v~aplikaci \emph{Clockify} \cite{clockify-ios}. Uživatelům umožňuje ovládat časovač různými způsoby na jednom místě. Uživatelé zmínili, že jakmile způsob přepínání pochopili, tak s~rozhraním problém neměli. Je tedy na pováženou, zda se smířit s~tím, že ovladač časovače vyžaduje nějaké pochopení, které není zcela intuitivní, ale jakmile k~pochopení dojde, umožní snadnější tvorbu záznamů. Protože například použití dialogu pro manuální tvorbu záznamu, jak navrhl jeden tester, by mohlo způsobit, že takový dialog překryje část viditelného rozhraní historie a tím uživateli znemožní se inspirovat předchozími záznamy, ověřit si, že na sebe časy navazují, a podobně.
\item\textbf{Absence swipe-to-delete u~některých seznamů – nekonzistence v~rámci aplikace:} Poznámka testera, že to způsobuje nekonzistenci v~rámci aplikace, byla zcela na místě. Podnětem pro další vývoj aplikace by mělo být sjednocení intearkčních prvků ve všech seznamech, aby všude byl implementován detail i~swipe-to-delete, a všechny se tak chovaly stejným způsobem.
\item\textbf{Klávesnice při zadávání e-mailové adresy nenabízí v~základním rozmístění kláves tečku:} Systém iOS umožňuje různým textovým polím přidělovat různé typy klávesnic, u~pole pro e-mailovou adresu by tohoto tedy mělo být využito, aby se zobrazila klávesnice se specializovaným rozmístěním kláves pro zadávání e-mailu.
\item\textbf{Malá klikatelná plocha tlačítka pro smazání:} U~obrazovek, kde vznikl tento problém, se používá nativní UI komponenta \emph{List}, která pracuje s~tlačítky v~řádcích takovým způsobem, že detekuje kliknutí na ně pouze tehdy, pokud se klikne na text tlačítka, alespoň v~základním použití. Podnětem dalšího vývoje by proto mělo být zkoumání, jak tento problém vyřešit. Buď snahou o~rozšíření klikatelné plochy, nebo nahrazení vlastním primárním tlačítkem mimo nativní seznam.
\item\textbf{Neviditelný kurzor při zobrazení zadaného hesla:} Aplikace používá vlastní řešení pro přepínání mezi viditelným a neviditelným stavem pole pro hesla, které odstraňuje chybu nativního řešení, které při přepnutí viditelnosti zruší \emph{focus} na pole, což způsobí zmizení klávesnice a uživatel na pole musí kliknout znovu. Vedlejším produktem tohoto řešení je ale zřejmě tento problém, je tedy potřeba použité řešení zrevidovat a pokusit se tento problém odstranit.
\item\textbf{Jednotlivé karty aplikace si pamatují zanoření:} U~tohoto poznatku bylo překvapující, že vznikl i~u~uživatelů systému iOS, ve kterém je toto chování při používání několika navigačních karet ve spodní liště zcela standardní. Je opět na pováženou, jestli se snažit toto chování měnit a určovat tento poznatek za určující, když se jedná o~standardní chování.
\item\textbf{Absence onboardingu nebo možnosti vytvoření počátečních klientů a projektů:} Prázdné stavy jednotlivých obrazovek by mohly být mimo samotný informující text rozšířeny o~tlačítko pro~přidání prvku, které je jinak pouze v~navigační liště. Implementace úvodní nápovědy poté může být podnětem pro další vývoj aplikace.
\item\textbf{Nedostatečná vizuální indikace běžícího časovače:} Tester, který tento poznatek vznesl, zmínil, že je z~ostatních řešení zvyklý, že při běžícím časovači má ovládací tlačítko červené pozadí, aby dostatečně vizuálně odlišilo stav běžícího časovače. Jestliže se jedná o běžnou praxi u~měřičů odpracovaného času, není důvod se této praxi vyhýbat a červené pozadí nepřidat také.
\item\textbf{Ne příliš výstižný popis tlačítka Uložit:} Navigační tlačítko \emph{Uložit} může být v~případě vytváření nového klienta nebo projektu změněno na \emph{Vytvořit}. Tento text je výstižnější a není důvod ho nepoužít.
\item\textbf{Ne příliš výstižný popis tlačítka Smazat:} Tlačítko \emph{Ano} může být nahrazeno textací \emph{Smazat}, která bude pravděpodobně více intuitivní.
\item\textbf{Přednost popisu záznamu před názvem klienta nebo projektu:} První tester, který tento podnět vznesl, nazval popis jako \emph{title} – jeho význam tedy asi pochopil jako název či nadpis záznamu. Popis ale v~záznamech poskytuje až sekundární roli, je nepovinný. Projekt je naopak u~záznamu povinný. Tento názor ale později vyjádřil i~další tester. Mělo by být tedy zváženo, zda v~rozhraní neupřednostnit popis záznamu před projektem, přestože vůbec nemusí být přítomen.
\item\textbf{Absence potvrzení před smazáním časového záznamu:} Testeři, kteří tento podnět vznesli, měli pravdu v~tom, že se jedná o~destruktivní akci a pokud ji uživatel udělá omylem, přijde o~data a nemá možnost je získat zpět. Přidání explicitního dialogu se ale zdá být zase příliš, protože pokud by uživatel chtěl smazat více záznamů, musel by u~každého záznamu potvrzovat dialog, a zas tolik dat se pod jedním záznamem neskrývá. Ideálním řešením by tedy mohla být přidaná možnost vrátit akci mazání. Po smazání záznamu, které by bylo učiněno stejným způsobem jako do teď, by se mohl zobrazit \emph{Toast} s~informací, že záznam byl smazán, a s~tlačítkem \emph{Vrátit}, které by akci vrátilo a smazaný záznam vrátilo zpět.
\item\textbf{Údiv nad zařazením exportu do CSV mezi integrace:} Dva testeři vyjádřili údiv nad tím, že export do CSV je zařazen mezi integrace. Je pravda, že samotný soubor CSV sám o~sobě integraci netvoří. CSV soubor je ale integračním prvkem, který umožňuje integraci v~podstatě s~čímkoli – s~řadou dalších systémů, s~nějakými vizualizačními nástroji, a podobně. A~vzhledem k~tomu, že uživatel ho právě za účelem integrace bude pravděpodobně využívat, tak má určitě své místo na kartě integrací. Ostatně, oddělit ho od této karty a hledat mu jiné místo by pravděpodobně nedávalo smysl.
\item\textbf{Aplikace se po exportu dat nevrací o~obrazovku zpět:} Tento podnět vznesl jeden tester. Bylo zhodnoceno, že toto chování by bylo očekávatelné v~případě, kdyby místo obrazovky sloužil pro export dat nějaký typ dialogu. Ale protože se jedná o~standardní obrazovku, jejíž přístup uživatelům umožní například exportovat více časových období za sebou, nebo opětovné exportování se stejnými parametry (které bylo dokonce předmětem jednoho scénáře), bylo rozhodnuto, že automatické vrácení zpět, které by způsobilo ztrátu dat na této obrazovce, by být implementováno nemělo.
\item\textbf{Absence indikace spuštěné zkratky v~aplikaci:} Tento podnět zmínili dva testeři, kteří při návratu do aplikace po spuštění zkratky neviděl žádnou explicitní indikaci toho, že byla zkratka provedena. Zkratky implementované v~aplikaci ale nemají záměr uživatele po jejich provedení přesměrovávat do aplikace, k~tomu slouží jiné, \emph{otevírací} typy zkratek. Zkratky aplikace Trackee slouží primárně pro automatizaci, uživatel nebo automatizace tedy zkratku spustí, aplikace vykoná úkony a uživatel vůbec nemá potřebu aplikaci otevírat. Tento podnět tedy byl pravděpodobně vznesen kvůli tomu, že instrukce scénáře testerovi přímo říkala, aby po spuštění zkratky zkontroloval, že se akce v~aplikaci provedla, ale toto uživatelé ve skutečnosti vůbec dělat nemusí. Jedná se tedy o~lehce zavádějící instrukci testovacího scénáře, která testery instruuje k~tomu, aby udělali něco, co není klasickým případem užití. Úmyslem tohoto scénáře bylo spíše to, aby testeři pochopili, co vlastně zkratka udělala.
\item\textbf{Při snaze o~úpravu zadaného hesla se maže celé pole:} Několik testerů toto chování podráždilo, ale jedná se o~standardní chování textových polí pro citlivé údaje. Pokud uživatel přestane s~úpravou takového pole, později s~úpravou zase začne a pokusí se smazat jeden znak, smaže se celý text pole. Jedná se o~bezpečností prvek, není tedy vhodné se toto chování snažit měnit.
\item\textbf{Zapnutí časovače v~aplikaci Zkratky nemá automaticky otevřený dialog pro parametry:} V~dokumentaci pro \emph{App Intents} \cite{ios-app-intents} nebyla nalezena zmínka o~tom, že by existovala nějaká možnost, jak aplikaci \emph{Zkratky} donutit, aby ve výchozím stavu otevřela dialog pro parametry. Toto chování tedy aplikace nemůže ovlivnit.
\end{itemize}































































