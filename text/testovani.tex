Výslednou aplikaci a její uživatelské rozhraní je potřeba důkladně otestovat. Tato kapitola popisuje implementaci automatických testů a scénáře testování s~reálnými uživateli. Výsledky z~testování jsou poté zhodnoceny.

%---------------------------------------------------------------
\section{Automatické testování}
%---------------------------------------------------------------

Pro účely automatizovaného testování se u~Apple platforem používá knihovna \emph{XCTest} \cite{xctest}. Jak již bylo zmíněno, \emph{Clean architecture} ja navržená tak, aby se dala dobře otestovat jednotkovými testy, které jsou popsány v~následující sekci.

Dalšími možnostmi testování jsou UI testy a testy výkonu. Aplikace Trackee implementaci těchto typů testů neobsahuje. Jejich implementace může být podnětem pro budoucí vylepšení aplikace.

%---------------------------------------------------------------
\subsection{Jednotkové testování}
%---------------------------------------------------------------

V~aplikaci jsou jednotkovými testy pokryty všechny \emph{View Modely}. Pro každý z~nich je v~modulu, ke kterému patří, soubor, který daný \emph{View Model} testuje. Např. pro \texttt{LoginViewModel.swift} bude testovací soubor \texttt{LoginViewModelTests.swift}.

\emph{View Model} testy jsou navržené tak, aby testovaly každý jednotlivý \emph{Intent}, pokud existuje vhodná možnost, jak test provést. Některé intenty vhodně otestovat nelze – například takové, které otevírají nějaké systémové dialogy, které nemají žádný výstup pro \emph{View Model} aplikace. Daný test tedy vždy ověřuje, zda se stav \emph{View Modelu} po aplikování \emph{Intentu} změnil podle očekávání.

Jednotkové testy \emph{View Modelů} využívají \emph{Mocky} pro \emph{Use Cases}, tedy určité zjednodušené náhrady těchto \emph{Use Cases}, u~kterých si každý test může definovat, co bude daný \emph{Use Case Mock} vracet za hodnotu. Toto je klasický přístup k~jednotkovému testování – testujeme zde \emph{View Modely}, nikoli \emph{Use Cases}.

Předpokládejme tedy, že máme \emph{View Model} se strukturou ve výpisu \ref{code:tested-vm-structure}. Struktura testu poté bude vypadat jako ve výpisu \ref{code:vm-test-sctructure}. Struktura obsahuje následující:
\begin{itemize}
\item\texttt{flowController} – \emph{Mock} pro \emph{FlowController}, tedy náhrada reálného controlleru v aplikaci. Tento \emph{Mock} slouží k~tomu, abychom mohli otestovat, jaká hodnota byla vložena do posledního volání funkce \texttt{handleFlow(\_:)} a kolikrát byla tato funkce zavolána. Toto slouží k~otestování toho, zda se bude apikace snažit uživatele navigovat do předpokládané destinace.
\item\emph{Mocky} pro \emph{Use Cases} – zde vytváříme instance \emph{Mocků} pro \emph{Use Cases}, které jsou v~daném \emph{View Modelu} používány. Je potřeba to dělat takto ručně, protože kdybychom použili registrování pomocí \texttt{Container.shared.registerUseCaseMocks()}, tak bychom nemohli v~testech měnit hodnoty, které dané \emph{Use Cases} budou vracet.
\item\texttt{createViewModel()} – tato funkce se bude volat na začátku každého testu, aby získal instanci pro otestování. Pokud konstruktor \emph{View Modelu} vyžaduje nějaké parametry, můžou být předány jako parametr této funkce, nebo zde můžou mít nějakou definovanou hodnotu. V~této funkci se také registrují \emph{Mocky} pro \emph{Use Cases} definované výše.
\item Testovací funkce pro jednotlivé intenty – funkce, které mají v~názvu prefix \texttt{test}, tak se automaticky považují ze testovací funkce a spouštějí se během testu. Každý vhodně otestovatelný intent zde tedy bude mít vlastní funkci, případně více funkcí, pokud lze otestovat úspěch, neúspěch, různé formy úspěchu, a podobně. Tyto funkce dodržují strukturu given-when-then, tedy rozdělení do tří částí (předdefinované hodnoty a konstanty, interakce, ověření).
\end{itemize}

\begin{listing}
\caption{Struktura View Modelu pro testování}\label{code:tested-vm-structure}
\begin{minted}{Swift}
final class SomeViewModel: BaseViewModel, ViewModel, ObservableObject {
	
    // ...
	
    @Injected(\.someUseCase) private var someUseCase
	
    // ...
	
    enum Intent {
        case doThis
        case doThat
    }
	
    // ...
}
\end{minted}
\end{listing}

\begin{listing}
\caption{Struktura View Model testu}\label{code:vm-test-sctructure}
\begin{minted}{Swift}
@MainActor
final class SomeViewModelTests: XCTestCase {

    private let flowController = FlowControllerMock<SomeFlow>(
        navigationController: UINavigationController()
    )
    
    private let someUseCaseMock = SomeUseCaseMock(
        executeReturnValue: /* Some default return value */
    )
    
    private func createViewModel(
        someOptionalParam: SomeType
    ) -> SomeViewModel {
        Container.shared.someUseCase.register { self.someUseCaseMock }
	
        return SomeViewModel(
            someParam: someOptionalParam,
            flowController: flowController
        )
    }
    
    // MARK: - Tests
    
    func testDoThis() async {
        // given
        let vm = createViewModel()
        // ... some constants and given values
        let someConstant = ...
    	
        // when
        vm.onIntent(.doThis)
        // ... some interactions
        await vm.awaitAllTasks()
		
        // then
        // ... assertions
        XCTAssertEqual(vm.state.someValue, someConstant)
    }
    
    func testDoThat() async {
        // ...
    }
}
\end{minted}
\end{listing}

Jednotkové testy \emph{View Modelů} testují byznysovou prezentační vrstvy. Pro otestování dalších vrstev je možné v multiplatformním modulu testovat \emph{Use Cases}, \emph{Repositories} a \emph{Sources}. V backendové části je také možné implementovat jednotkové testy pro \emph{Repositories}, \emph{Sources}, ale i~pro \emph{Routes}. Implementace testů pro tyto vrstvy může být podnětem pro budoucí vylepšení aplikace.

%---------------------------------------------------------------
\section{Uživatelské testování}
%---------------------------------------------------------------

Pro účely uživatelského testování byl navržen scénář, podle kterého se testeři mají řídit.

%---------------------------------------------------------------
\subsection{Scénář}
%---------------------------------------------------------------

Proces testování je rozdělen do několika scénářů, které by měly pokrývat většinu funkcionalit aplikace, kterých uživatel může využít. Povinnými účastníky během testování budou moderátor a uživatel. Moderátor bude uživateli dávat instrukce a dokumentovat, jak uživatel s~mobilním telefonem interaguje. Uživatel se bude snažit instrukce plnit a bude při tom nahlas popisovat své akce s~telefonem (co vidí, na co se chystá kliknout, co očekává, že se stane, atd.), aby usnadnil následnou analýzu celé situace.

\begin{description}
\item[Tvorba uživatelského účtu:] Každý nový uživatel mobilní aplikace Trackee si bude muset nejprve vytvořit svůj uživatelský účet. V~tomto scénáři bude uživatel instruován k~tomu, aby aplikaci poprvé spustil, vytvořil si nový uživatelský účet pomocí předepsaného e-mailu a hesla a následně se do aplikace pod tímto účtem přihlásil. 
\item[Vytvoření nového klienta:] Po úspěšné registraci bude uživatel instruován, aby si vytvořil 2 nové klienty s~předepsanými názvy.
\item[Vytvoření nového projektu:] Po úspěšném vytvoření klientů bude uživatel instruován, aby vytvořil 2 nové projekty s~předepsanými vlastnostmi.
\item[Spuštění časovače:] Po úspěšném vytvoření projektů dostane uživatel instrukci, aby spustil časovač pro měření odpracovaného času a přiřadil mu předepsaný projekt a popis.
\item[Změna začátku časovače:] Uživatel bude instruován k~tomu, aby u~spuštěného časovače změnil začátek na předepsaný čas.
\item[Zastavení časovače:] Uživatel bude instruován, aby ukončil časovač a uložil časový záznam, který do teď měřil.
\item[Manuální přidání časového záznamu:] Uživatel bude instruován, aby ručně přidal časový záznam s~předepsanými parametry a časem.
\item[Úprava projektu:] Uživatel bude instruován, aby aktualizoval předepsaný projekt s~novými předepsanými vlastnostmi.
\item[Odhlášení a přihlášení na testovací účet:] Uživatel ve svém nově vytvořeném účtě nebude mít dostatek klientů, projektů a časových záznamů, které by odpovídaly dlouhodbějšímu používání aplikace. Pro lepší otestování orientace v~aplikaci bude uživatel instruován, aby se přihlásil na předepsaný testovací účet, který obsahuje větší množství záznamů.
\item[Odstranění časového záznamu:] Uživatel bude instruován, aby odstranil předepsaný časový záznam z~historie.
\item[Export historie do CSV souboru:] Uživatel bude instruován, aby vytvořil integraci pro exportování do CSV souboru a exportoval data do tabulky pro předepsané časové období. Dále bude instruován, aby tuto tabulku otevřel ve vhodné aplikaci, kde si ji může prohlédnout (např. \emph{Numbers}).
\item[Odstranění klienta:] Budeme předpokládat, že si uživatel v~exportované tabulce všimne, že tam má záznamy patřící předepsanému klientovi, které tam mít nechce. Bude tedy instruován, aby klienta smazal a znovu vyexportoval CSV soubor, ve kterém zkontroluje, že žádné záznamy patřící k~tomuto klientovi nejsou.
\item[Tvorba automatizace pro spuštění časovače:] Uživatel dostane instrukci, aby pomocí aplikace \emph{Zkratky} vytvořil novou automatizaci, která zapne časovač s~předepsanými parametry, pokud se uživatel objeví na předepsané poloze. Bude také instruován, aby tuto automatizaci zkusil ručně spustit a v~aplikaci zkontroloval, že časovač opravu běží.
\item[Tvorba automatizace pro vypnutí časovače:] Uživatel bude požádán, aby vytvořil automatizaci, která se ho zeptá, zda nechce zastavit běžící časovač, pokud opustí předepsanou polohu. Zde bude také instruován, aby automatizaci zkusil ručně spustit.
\end{description}
