\chapter{Protokol z~uživatelského testování}\label{appendix:ui-testing-protocol}

\begin{tabular}{rl}
iOS & Uživatel platformy iOS\\
Android & Uživatel platformy Android\\
Clockify & Uživatel aplikace Clockify\\[0.5cm]

\textbf{Zvýrazněný text} & Důležitý poznatek testu\\
(!) & Kritický poznatek testu\\[0.5cm]
\end{tabular}

%---------------------------------------------------------------
\section*{Tester F. W. (27 let, Android, Clockify)}\label{tester:fw}
%---------------------------------------------------------------

\begin{description}
\item[Tvorba uživatelského účtu:] Klikl na tlačítko Registrovat a pokusil se do pole e-mailu zadat nevalidní e-mail ve zvědavosti, zda to aplikace správně detekuje. Poté se také zkusil zadat různá hesla, aby zjistil, zda to aplikace verifikuje, a zjistil, že ano. Chtěl druhé heslo upravit a \textbf{vyjádřil zmatení nad tím, že ve stavu, kdy má heslo zobrazené, nevidí kurzor}. Heslo opravil, ale aplikace po snaze se registrovat \textbf{ukázala neznámou chybu} (!). Uživatel předpokládal, že je to kvůli nevalidnímu formátu e-mailu, tak ho upravil, ale \textbf{aplikace stále ukazovala neznámou chybu} (!). Také zmínil, že mu vadí, že \textbf{klávesnice při zadávání e-mailu nenabízí v~primárním rozmístění kláves tečku}. Tak zkusil celý proces znovu s~jiným heslem a registrace už se podařila, a uživatel předpokládal, že \textbf{měl asi příliš krátké heslo}. Poté si začal prohlížet aplikaci a zmínil, že by mu \textbf{zde dával smysl nějaký onboarding, aby věděl, že si musí vytvořit projekt}. Při prohlížení karty integrací zmínil, že \textbf{CSV mu nepřijde jako typ integrace}, ale že tomu rozumí.
\item[Vytvoření nového klienta:] Během prohlížení aplikace už se dostal na kartu profilu, takže poté rovnou intuitivně klikl na tlačítko Klienti. Ihned poté si všiml tlačítka + a klikl na něj, poté zadal název klienta. Poté klikl na tlačítko Uložit a zmínil, že by mu \textbf{dávalo větší smysl například Vytvořit}. Poté stejným způsobem vytvořil druhého klienta. Kliknout na tlačítko Uložit se mu ale \textbf{podařilo až na 3. pokus}.
\item[Vytvoření nového projektu:] Rovnou se vrátil na profil a klikl na tlačítko Projekty a poté na tlačítko +. Zadal předepsané parametry a při výběru klienta zmínil, že by očekával, že \textbf{při kliknutí na klienta se aplikace rovnou vrátí o~obrazovku zpět}, že to nebude muset dělat dalším kliknutím na Uložit. Při vybírání typu projektu zmínil, že neví, kde se tento seznam typů vzal, že ho nevytvářel, takže je to asi výchozí. Opět se mu \textbf{nedařilo kliknout na tlačítko + a musel to dělat vícekrát}. Druhý projekt už vytvořil bez problémů, pouze zmínil údiv nad autokorekcí systému iOS.
\item[Spuštění časovače:] Přesunul se na kartu časovače a intuitivně klikl na tlačítko pro spuštění. Ihned poté klikl na výběr projektu, vybral projekt a zmínil, že \textbf{je otravné, že opět musí klikat na tlačítko Uložit}. Popis zadal bez problému.
\item[Změna začátku časovače:] Intuitivně rovnou klikl na stopky časovače a správně posunul čas začátku. Poté si ještě vyzkoušel, co se stane, když nastaví čas začátku v~budoucnu, a aplikace mu správně vrátila chybu, že to nejde.
\item[Zastavení časovače:] Intuitivně klikl na tlačítko pro stopnutí a viděl, že se záznam uložil. Také zmínil, že mu nastavení z~minulého měření zůstalo v~časovači.
\item[Manuální přidání časového záznamu:] Intuitivně klikl na tlačítko +, ale \textbf{zmátlo ho, jak na to rozhraní ovladače reagovalo}, že se mu tam objevil nějaký náhodný čas. Chvíli to zkoumal a zmínil, že mu to přijde \textbf{neintuitivní} a že by \textbf{spíše čekal, že se mu objeví nějaký dialog}. Po chvíli pochopil, že jde o~nějakou formu přepínání, a dodal, že je to asi o~zvyku. Poté ale už správně zadal čas začátku a konce a zmínil, že informace o~tom, jak dlouhý bude výsledný interval, je poměrně schovaná. Poté přemýšlel, jak záznam uloží, a klikl na tlačítko + a viděl, že se záznam uložil správně. Poté si všiml, že zapomněl změnit projekt a popis podle zadání, a \textbf{snažil se záznam upravit}, ale všiml si, že to asi nepůjde. Moderátor ho ujistil, že znova vytvářet záznam kvůli tomu nemusí. Zopakoval, že dialog by mu dával větší smysl, protože mu přesně nedošlo, co se děje.
\item[Úprava projektu:] Klikl na kartu profilu a byl chvíli zmaten, že zůstala zanořena v~projektech. Klikl na správný projekt, změnil jeho název, přešel na časovač, ale tam se nic nezměnilo, protože u~záznamu zapomněl přiřadit správný projekt. 
\item[Odhlášení a přihlášení na testovací účet:] Šel na kartu profilu, odhlásil se a zadal údaje testovacího účtu. Po přihlášení si všiml, že záznamů už je více.
\item[Odstranění časového záznamu:] Nejprve záznam nemohl najít, protože \textbf{čekal, že popis (doslovně ale řekl title) bude nahoře, ne pod projektem a klientem}. Chvíli zkoumal, jak záznam smazat, několikrát se na něj pokusil kliknout, a poté intuitivně zkusil swipe-to-delete gesto, u~kterého ho \textbf{překvapilo, že není žádný potvrzující dialog}.
\item[Export historie do CSV souboru:] Zmínil, že mu přijde \textbf{zvláštní mít export v~integracích}, ale přešel to. Novou integraci vytvořil a data exportoval bez problémů. Moderátor poté poradil, jak soubor otevřít v~Numbers, jelikož tento dialog už není součástí aplikace.
\item[Odstranění klienta:] Při návratu do aplikace zmínil, že \textbf{by čekal, že se aplikace po exportu vrátí o~obrazovku zpět}. Smazání provedl bez problémů, pouze zmínil, že \textbf{pro kliknutí na tlačítko musí kliknout na jeho text}. Druhý export byl také bez problémů.
\item[Tvorba automatizace pro spuštění časovače:] Moderátor testerovi radil, jak zkratky a automatizace tvořit, jelikož se nejedná o~rozhraní aplikace. Nejprve zapomněl na nastavení parametrů časovače, moderátor mu to musel připomenout.
\item[Tvorba automatizace pro vypnutí časovače:] Upravil existující automatizaci a změnil zapnutí časovače na vypnutí.
\item[Shrnutí:] Zopakoval přednost popisů před názvem projektu a klienta, a také přepínání ovladače časovače. Také zopakoval, že by se mu hodil onboarding, ale jinak že mu používání aplikace dává smysl. Poté ještě zopakoval absenci potvrzení při mazání záznamu a poukázal na chybnou animaci při mazání. Zmínil, že mu přijde, že souhrn odpracovaných hodin mu přijde schovaný, a že by dnešek čekal vedle nadpisu skupiny, jako je to u~ostatních dnů. Pak ho ještě napadlo, že integrace by možná nečekal v~navigační liště, že to uživatelé asi nebudou používat příliš často, že to jen nastaví a pak to bude fungovat. Při zhodnocení, zda by aplikaci používal, tak zmínil, že asi ne, že mu aplikace přijde pomalá, nemůže upravovat záznamy, že v~této formě by jí asi nepoužíval, protože je zkrátka zvyklý na jinou aplikaci.
\end{description}

%---------------------------------------------------------------
\section*{Tester D. K. (24 let, iOS, Clockify)}\label{tester:dk}
%---------------------------------------------------------------

\begin{description}
\item[Tvorba uživatelského účtu:] Proběhlo bez problému, vše našel intuitivně.
\item[Vytvoření nového klienta:] \textbf{Nejprve klikl na + na ovladači časovače}, kde zjistil, že to dělá něco jiného. Poté \textbf{klikl na výběr projektu}, kde přečetl instrukci, že si musí vytvořit projekt. Moderátor ho upozornil, že přeskočil instrukci pro vytvoření klienta, poté omylem klikl na odhlášení a musel se znovu přihlásit. Vytvoření klienta poté proběhlo bez problémů.
\item[Vytvoření nového projektu:] Proběhlo bez problémů, vše našel intuitivně.
\item[Spuštění časovače:] Vybral projekt, poté spustil časovač a poté přidal popis.
\item[Změna začátku časovače:] Proběhlo bez problémů, vše našel intuitivně.
\item[Zastavení časovače:] Proběhlo bez problémů.
\item[Manuální přidání časového záznamu:] Proběhlo bez problémů, vše našel intuitivně.
\item[Úprava projektu:] \textbf{Nejprve zkusil klikat na záznam} a když zjistil, že to nic nedělá, upravil projekt standardně v~profilu.
\item[Odhlášení a přihlášení na testovací účet:] Při přihlášení začal zadávat vlastní údaje, moderátor ho upozornil, aby zadal předepsané údaje. Zbytek proběhl bez problémů.
\item[Odstranění časového záznamu:] V~testovacích datech chyběl předepsaný záznam pro smazání, což bylo chybou moderátora, že data špatně připravil. Instruoval proto testera, aby smazal jiný záznam. \textbf{Nejprve zkusil na záznam kliknout} – nic se nestalo. \textbf{Poté na něm zkusil podržet prst} – nic se nestalo. Hned poté zkusil swipe-to-delete a záznam smazal. Měl u~toho pocit, že se animace trochu zasekává, ale přešel to a pokračoval.
\item[Export historie do CSV souboru:] Proběhlo bez problémů, vše našel intuitivně.
\item[Odstranění klienta:] Proběhlo bez problémů, vše našel intuitivně.
\item[Tvorba automatizace pro spuštění časovače:] Moderátor testerovi poradil, jak aplikaci Zkratky používat. Po spuštění zkratky zmínil, že \textbf{by čekal, že se aplikace automaticky scrollne dolů, nebo nějakým jiným způsobem naznačí, že se něco stalo}.
\item[Shrnutí:] Zopakoval, že by přidal nějakou indikaci, že se po spuštění zkratky něco stalo. Také zmínil, že by na hlavní obrazovku přidal tlačítko pro přidání profilu nebo klienta, pokud žádné nemá, nebo rovnou celý onboarding.
\end{description}

%---------------------------------------------------------------
\section*{Tester D. Ž. (31 let, Android, Clockify)}
%---------------------------------------------------------------

\begin{description}
\item[Tvorba uživatelského účtu:] Během ověřování zadaného hesla napsal druhé heslo chybně, a když ho chtěl upravit a smazat poslední znak, tak ho znepokojilo, že se smazal celý obsah pole. Znovu heslo zadal a úspěšně se registroval.
\item[Vytvoření nového klienta:] Proběhlo bez problému, vše našel intuitivně. \textbf{Klikání na tlačítko + nebo Uložit se mu podařilo až na více pokusů}.
\item[Vytvoření nového projektu:] Vše našel intuitivně. \textbf{Pozastavil se nad tím, že po vybrání klienta k~projektu musí klikat na Uložit}.
\item[Spuštění časovače:] Proběhlo bez problému, vše našel intuitivně.
\item[Změna začátku časovače:] Nejdříve zastavil časovač a poté přemýšlel, co má dělat, snažil se nejprve upravit uložený záznam. Poté přepnul ovladač do manuálního zadávání času a zadal čas. Poté přemýšlel, co dělat dál – nevěděl, jak záznam uložit. Ptal se, že by se záznam po uložení měl asi zobrazit v~historii. Moderátor poté upřesnil instrukci, že začátek času měl být upraven u~již běžícího časovače. Poté časovač znovu zapl a začátek upravil správně.
\item[Zastavení časovače:] Zmínil, že už to udělal v~minulém kroku, a test přeskočil.
\item[Manuální přidání časového záznamu:] Zadání času našel intuitivně. \textbf{Musel opět několikrát kliknout na tlačítko Uložit, aby se dotyk zaregistroval}. Poté tlačítkem + záznam uložil.
\item[Úprava projektu:] Proběhlo bez problému, vše našel intuitivně.
\item[Odhlášení a přihlášení na testovací účet:] \textbf{V~kartě profilu byl zmaten, že není v~profilu, ale ve vnořené obrazovce projektů}, ale pak se vrátil o~obrazovku zpátky a zbytek proběhl bez problému.
\item[Odstranění časového záznamu:] \textbf{Nejprve se snažil kliknout} a poté vyzkoušel swipe-to-delete gesto a záznam smazal.
\item[Export historie do CSV souboru:] Proběhlo bez problému, vše našel intuitivně.
\item[Odstranění klienta:] Proběhlo bez problému, vše našel intuitivně.
\item[Tvorba automatizace pro spuštění časovače:] Moderátor radil, jak aplikaci Zkratky používat. Zbytek proběhl bez problému.
\item[Shrnutí:] Zmínil, že protože je zvyklý na platformu Android, tak občas nevěděl, co se kde nachází, ale jinak neměl problém. Líbilo se mu, že aplikace je intuitivní. Zmínil, že byl akorát zmatený z~editace času. 
\end{description}

%---------------------------------------------------------------
\section*{Tester E. Č. (24 let, iOS)}
%---------------------------------------------------------------

\begin{description}
\item[Tvorba uživatelského účtu:] Nejprve se snažil zadat přihlašovací údaje v~obrazovce pro přihlášení, ne pro registraci. Poté pochopil, že se musí nejdřív registrovat. Zbytek proběhl bez problému.
\item[Vytvoření nového klienta:] Proběhlo bez problému, vše našel intuitivně.
\item[Vytvoření nového projektu:] Proběhlo bez problému, vše našel intuitivně.
\item[Spuštění časovače:] Zapnul časovač a poté vybral projekt. Zeptal se moderátora, jestli časovač běží od doby, co byl zapnut, nebo od doby, co byl vybraný projekt. Moderátor poradil, ať se řídí podle toho, co ukazují stopy časovače.
\item[Změna začátku časovače:] Zeptal se moderátora, jestli má časovač stopovat. Ten poté poradil, ať se jen pokusí změnit čas začátku. \textbf{Nejdřív se pokusil kliknout na shrnutí odpracovaných hodin za den a týden}, nic se nestalo, tak poté klikl správně na stopky.
\item[Zastavení časovače:] Již vyzkoušel na konci předchozího scénáře.
\item[Manuální přidání časového záznamu:] Potřeboval pomoct s~vysvětlením, co má vlastně udělat, že se jedná o~časový záznam a ne o~projekt. Poté si nebyl jistý, jak má záznam uložit, protože nikde nevidí tlačítko Uložit. Vyzkoušel tlačítko +, které záznam úspěšně uložilo. Poté si uvědomil, že zapomněl změnit popis, a \textbf{snažil se vytvořený záznam upravit}, což nešlo. Rovnou ho tedy zkusil smazat intuitivně pomocí swipe-to-delete gesta. Poté korektně vytvořil nový záznam.
\item[Úprava projektu:] \textbf{Nejdříve se snažil kliknout na časový záznam, aby ho upravil}. Když se nic nestalo, tak šel na profil do projektů a projekt upravil.
\item[Odhlášení a přihlášení na testovací účet:] Byl zmatený, že po kliknutí na profil se nenacházel na profilu, ale ve vnořené obrazovce projektů, a nevěděl, jak se má dostat do profilu. Moderátor poradil, že je potřeba se vrátit ze zanoření. Při přihlášení omylem zadal v~hesle tečku navíc, a \textbf{když ho chtěl poté upravit, tak ho podráždilo, že se smazalo celé heslo a ne jen tečka}.
\item[Odstranění časového záznamu:] Proběhlo bez problému, vše našel intuitivně.
\item[Export historie do CSV souboru:] Potřeboval poradit, co je integrace a co je CSV soubor. Moderátor podal instrukci, ať se to pokusí v~aplikaci najít, a zbytek proběhl bez problémů.
\item[Odstranění klienta:] \textbf{Nejdříve se klienta pokusil smazat pomocí swipe-to-delete, které ale u~klientů nefunguje} – tuto skutečnost také popsal, že je to matoucí. Poté rozklikl detail a smazal klienta tam. Poté také zmínil, že mazání trvá dlouho.
\item[Tvorba automatizace pro spuštění časovače:] Moderátor poradil, jak aplikaci Zkratky používat. Tester upozornil na to, že v~zadání testu je špatný název projektu, který má být vybrán.
\item[Shrnutí:] Zmínil, že mobilní aplikace moc nepoužívá, takže byl občas zmaten, jak se co ovládá. S~rozhraním aplikace byl spokojený, ale zmínil, že aplikace pro měření času nepoužívá, takže mu doména není moc blízká.
\end{description}

%---------------------------------------------------------------
\section*{Tester T. S. (32 let, Android, Clockify)}
%---------------------------------------------------------------

\begin{description}
\item[Tvorba uživatelského účtu:] \textbf{Divil se, že při zadávání e-mailové adresy není rozmístění kláves přizpůsobeno pro zadávání e-mailové adresy.} \textbf{Při úpravě hesla měl pocit, že nemůže do textového pole psát, protože nebyl při viditelném stavu vidět kurzor, a opakovaně na pole klikal.} Tuto skutečnost také zmínil.
\item[Vytvoření nového klienta:] Proběhlo bez problému, vše našel intuitivně.
\item[Vytvoření nového projektu:] Vyzkoušel, co se stane, když nevybere klienta. Aplikace zareagovala příslušnou chybovou hláškou. Zadal jiné názvy, ale moderátor podotkl, že to nevadí. Zbytek proběhl bez problému.
\item[Spuštění časovače:] \textbf{Zmínil, že po výběru projektu by čekal, že se dialog zavře}, že klikání na Uložit je otravné. Zbytek proběhl bez problému.
\item[Změna začátku časovače:] Proběhlo bez problému, vše našel intuitivně.
\item[Zastavení časovače:] Proběhlo bez problému, vše našel intuitivně. Po uložení záznamu zmínil, že až teď pochopil, proč byla obrazovka nad ovladačem časovače prázdná.
\item[Manuální přidání časového záznamu:] \textbf{Zmínil, že klikatelná plocha popisu je poměrně malá.} Zbytek proběhl bez problému.
\item[Úprava projektu:] Proběhlo bez problému, vše našel intuitivně.
\item[Odhlášení a přihlášení na testovací účet:] Proběhlo bez problému, vše našel intuitivně.
\item[Odstranění časového záznamu:] \textbf{Nejprve si daného záznamu nevšiml, protože popis očekával na prvním řádku.} Zmínil, že má dojem, že větší důraz je na popis, než projekt.
\item[Export historie do CSV souboru:] Nejprve hledal, kde integraci vytvořit, \textbf{byl zmaten z~toho, že export do CSV je mezi integracemi}. Zbytek proběhl bez problému.
\item[Odstranění klienta:] \textbf{Nejprve zkoušel podržet na klientovi, poté swipe-to-delete}, nic se ale nedělo. Poté až se dostal na detail, kde vidět tlačítko pro smazání. U~potvrzovacího dialogu zmínil, že \textbf{by čekal, že potvrzovací tlačítko bude obsahovat text Smazat}. Zbytek proběhl bez problému.
\item[Tvorba automatizace pro spuštění časovače:] \textbf{Zmínil, že by čekal, že nabídka parametrů bude otevřená rovnou.} \textbf{Po spuštění zkratky zmínil, že by čekal, že se obrazovka automaticky posune dolů.} Dále zmínil, že by \textbf{přidal červenou barvu jako pozadí ovládacího tlačítka časovače, pokud běží, jelikož to je zvykem u~existujících řešení}.
\item[Shrnutí:] Přišlo mu, že aplikace se chová standardně. Líbilo se mu, že se nahoře v~časovači vidí shrnutí odpracovaného času a že seznam historie je řazen zespodu. Zopakoval, že si není jistý, jestli by projekt měl být před popisem, že by dal popisu větší prioritu. Také zmínil, že by přidal potvrzovací dialog při mazání záznamu. Také zmínil překvapení nad tím, že export do CSV je řazen mezi integrace, když se jedná o~export – přemýšlel nad tím, že by to oddělil, ale tím si taky nebyl jistý.
\end{description}





















































