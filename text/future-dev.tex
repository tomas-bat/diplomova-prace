Během tvorby této práce vzniklo mnoho podnětů pro budoucí vývoj a vylepšení aplikace. Už v~samotném návrhu nebo realizaci byla rozebírána řada možností, které v~této práci realizovány nejsou, ale bylo by je vhodné v~budoucnu implementovat. Během uživatelského testování pak vznikla řada doporučení pro úpravu rozhraní aplikace.

%---------------------------------------------------------------
\section{Zrychlení komunikace a lokální data}
%---------------------------------------------------------------

Prvním tématem, kterému by se budoucí provoz aplikace měl zabývat, je celkové zrychlení manipulace s~aplikací. Jsou dvě zásadní věci, které zpomalují interakci s~aplikací a které vyžadují vylepšení.

%---------------------------------------------------------------
\subsection{Komunikace s~databází}
%---------------------------------------------------------------

Pomalá komunikace s~databází je způsobena hlavně tím, že zvolená řešení pro nasazení backendu a databáze provozují jejich instance navzájem vzdálené 8 000 km od sebe. Důvody, proč tomu tak je, byly popsány v~sekci \ref{development-tools}. Nejlepším řešením tohoto problému by bylo používat takové nástroje, které minimalizují vzdálenost mezi backendem a databází, která ale budou pravděpodobně vyžadovat placené plány. Mezi backendem a databází ale probíhá největší počet požadavků na čtení či zápis, jejich vzájemná komunikace je proto ještě důležitější, než komunikace mezi backendem a samotným klientem.

%---------------------------------------------------------------
\subsection{Lokálních data (cache)}
%---------------------------------------------------------------

Moderní aplikace obvykle využívají nějakou formu lokální \emph{cache}, která může mít mnoho využití, jako například dočasné zastoupení vzdálených dat z~databáze. Vhodné navržení a implementace takové \emph{cache} je ale poměrně komplexní záležitost a proto nemohla být realizována v~rámci této práce. Lokální \emph{cache} by šlo využít mimo jiné takto:
\begin{itemize}
\item{\emph{UseCases}, které vrací různé seznamy dat (záznamy, klienty, projekty, \dots), by místo prostého vrácení jedné hodnoty mohly vracet nějaký \emph{stream} dat (napříkad \emph{KotlinX Flow} \cite{kotlinx-flow}), které by nejprve vrátily data z~\emph{cache} lokální databáze, a hned jak by přišly data ze vzdálené databáze, tak by se data nahradily těmito daty a také by byla aktualizována \emph{cache}. Tímto způsobem by v~podstatě zmizely prodlevy, ve kterých musí uživatel čekat, než se mu zobrazí data a může interagovat s~aplikací. Je však potřeba v~jednotlivých případech počítat s~tím, kdy aplikace nějaká data aktualizuje – tehdy je potřeba na vzdálená data počkat.}
\item{Funkcionality, které nějakým způsobem aktualizují data (ovládání časovače, uložení záznamu, tvorba/úprava klienta nebo projektu, \dots), by nemusely po aplikování změn čekat na to, až aktualizovaná data přijdou, ale mohly by rovnou zobrazit data v~takovém formátu, v~jakém by aplikace předpokládala, že aktualizovaná data budou vypadat (například při vytvoření klienta se klient rovnou zobrazí v~seznamu klientů a až po chvíli jeho přidání potvrdí vzdálená data). Bylo by však třeba v~případě, že aktualizace dat selže, vrátit data do původního stavu, před očekávanou aktualizací.}
\end{itemize}

Implementace aplikace je pro využití lokální \emph{cache} dobře připravena. V~multi-platformní části již obsahuje nástroje pro manipulaci s~lokální databází pomocí knihovny \emph{SQLDelight} \cite{sqldelight}. Lze také v~implementaci nahlédnout, že v~multi-platformní části se jednotlivé \emph{sources} jmenují například \texttt{RemoteIntegrationSource}, což představuje zdroj vzdálených dat (z~backendu) pro integraci. Při použití lokální databáze by se definoval \texttt{LocalIntegrationSource}, který by implementoval obdobné funkcionality, a logika toho, kdy a jak se má který typ \emph{source} použít, by byla obsažena v~konkrétní \emph{repository}, v~uvedeném příkladě \texttt{IntegrationRepository}.

%---------------------------------------------------------------
\section{Úpravy uživatelského rozhraní}
%---------------------------------------------------------------

Z~průběhu uživatelského testování vzešla řada poznatků, ať už velmi důležitých nebo méně důležitých, které poukázaly na nějaké nedostatky či možnosti vylepšení v~uživatelském rozhraní aplikace. Důkladný popis, analýza a doporučení pro další vývoj aplikace, vycházejících z~těchto poznatků, je v~sekci \ref{ui-testing-results}, která by měla sloužit jako podnět budoucího vývoje.

%---------------------------------------------------------------
\section{Rozšíření funkcionalit}
%---------------------------------------------------------------

Funkcionality, které aplikace implementuje, mají také široký potenciál, jak rozšířit jejich možnosti a přinést tak uživatelům další funkce při používání aplikace. Mezi tyto možnosti může patřit:
\begin{itemize}
\item\textbf{Možnost úpravy časového záznamu.} Tato funkce by uživatelům ulehčila práci se záznamy, pokud například špatně zadají nějaké parametry záznamu. Také se jednalo o~jeden z~poznatků uživatelského testování – zájem ze strany uživatelů o~tuto funkci tedy určitě je.
\item\textbf{Více parametrů pro klienty.} U~klientů lze v~realizované aplikaci zadávat pouze název, pro různé budoucí účely se ale mohou hodit další parametry, jako třeba fakturační adresa klienta, různé poznámky, nebo barevné rozlišení.
\item\textbf{Více parametrů pro projekty.} U~projektů se také v~budoucnu mohou hodit další parametry, než ty, které aplikace v~tuto chvíli nabízí. Mezi ty může například patřit cena práce za hodinu u~konkrétního projektu, kterou lze využít během fakturace, nebo možnost vytvořit si vlastní typy projektů. 
\item\textbf{Reporty a shrnutí.} Mnoho existujících řešení pro měření a správu odpracovaného času umožňuje vytvářet nějaké formy reportů nebo vizualizací odpracovaného času. Uživatelé si mohou prohlížet, kolik práce vykonali v~určitých obdobích u~různých klientů a projektů, a podobně. Aplikace tuto funkci v~současné chvíli nemá a tvorbu takových shrnutí umožňuje pouze pomocí jiných nástrojů, například z~CSV dat, která aplikace poskytne.
\item\textbf{Integrace s~dalšími existujícími systémy.} V~analýze v~sekci \ref{existing-tracking-solutions} byly popsány různé existující systémy pro měření a správu odpracovaného času, spolu s~možnostmi integrace s~nimi. Realizace aplikace implementuje základ pro tyto integrace (CSV, \emph{Clockify}), ale potenciál těchto integrací je mnohem větší. Předmětem dalšího vývoje by tedy mohla být nějaká analýza toho, jaké další integrace by bylo vhodné implementovat a tyto integrace poté realizovat.
\item\textbf{Integrace s~hardwarovými spouštěči.} V~analýze byly také popsány možnosti hardwarových spouštěčů (fyzické ovladače apod.), které by šly využít při integraci s~aplikací. Pro propojení s~jedním z~nich by bylo potřeba implementovat BLE komunikaci. Zvážení této funkce také může být předmětem dalšího vývoje.
\item\textbf{Přihlášení přes externí poskytovatele.} \emph{Firebase} autentizace, které aplikace využívá, nabízí možnosti přihlášení přes \emph{Apple}, \emph{Facebook} a \emph{Google}. V~aplikaci je dokonce připraven \emph{provider} pro přihlášení přes \emph{Apple} – \texttt{AppleSignInProvider}.
\end{itemize}

%---------------------------------------------------------------
\section{Nasazení mezi reálné uživatele}
%---------------------------------------------------------------

V~realizaci v~sekci \ref{testflight} bylo popsáno, jakým způsobem byla aplikace nasazována pro účely testování během vývoje. Podobným způsobem se poté realizuje i~nasazení mezi reálné uživatele v~obchodě \emph{App Store}. Pro nahrání aplikace do tohoto obchodu také slouží nástroj \emph{App Store Connect}.

Jak již bylo také zmíněno, aplikace musí před schválením pro nasazení do obchodu \emph{App Store} projít procesem \emph{App Review}, ve kterém \emph{Apple} kontroluje splnění všech pravidel pro aplikace, které do obchodu míří. Aplikace už ale tímto procesem úspěšně prošla už v~rámci veřejného testování. \emph{App Review} pro \emph{App Store} je sice o~něco pečlivější než pro veřejné testování, ale zásadní porušení pravidel pro šíření aplikací v~\emph{App Store} by mělo být odhaleno již v~tomto procesu.

Ostatní nástroje, které jsou aplikací využívány, pak žádnou další konfiguraci pro nasazení mezi reálné uživatele nepotřebují. Backend i~databáze jsou nasazeny pod veřejně přístupnými doménami a aplikace je na ně napojena. Bude pouze potřeba monitorovat využití backendu nebo databáze, které jsou sice prozatím pod neplacenými plány, ale v~budoucnu budou pravděpodobně zapotřebí placené plány. Také bude podle využití potřeba zhodnotit, zda se stále vyplatí využívat funkce pro uspávání backendu, kterou \emph{Railway} nabízí.

%---------------------------------------------------------------
\section{Podpora pro další platformy}
%---------------------------------------------------------------

Návrh architektury platformy (sekce \ref{platform-architecture}) počítal s~možnostmi budoucího rozšíření podpory pro další platformy, jako je Android aplikace, webová aplikace a další. Toto rozšíření může být implementováno s~pomocí technologie \emph{Kotlin Multiplatform}, kterou klientská aplikace používá. Stačí přidat moduly pro nové platformy a jejich aplikace implementovat podobným stylem, jako nativní iOS aplikaci. Detailnější popis struktury celého projektu v~rámci \emph{monorepa} je v~realizaci v~sekci \ref{project-structure}.

%---------------------------------------------------------------
\section{Analytika}
%---------------------------------------------------------------

U~různých softwarových aplikací může být užitečné nějakým způsobem zaznamenávat a analyzovat chování uživatelů v~aplikaci. Lze zaznamenávat, na co jak často klikají, jak používají různé interakční prvky, a podobně. Tato data poté mohou být vhodným způsobem analyzována pro zhodnocení účinností jednotlivých prvků rozhraní, testování, co funguje lépe, a další. Je však potřeba si dávat pozor na to, aby takové sbírání dat podléhalo všem zákonům a regulacím o~sbírání uživatelských dat. 

Jelikož je v~aplikaci již používána technologie \emph{Firebase}, lze pro tyto účely použít nástroj \emph{Firebase Analytics} \cite{firebase-analytics}, který slouží pro zaznamenávání a analýzu různých akcí (klikání na některé prvky, interakce, \dots), a nástroj \emph{Firebase Crashlytics} \cite{firebase-crashlytics}, který slouží pro analýzu chyb a pádů aplikace.







































