% Do not forget to include Introduction

\graphicspath{ {./images} }

%---------------------------------------------------------------
\chapter*{Úvod}
%---------------------------------------------------------------
\addcontentsline{toc}{chapter}{Úvod}
\setcounter{page}{1}
\markboth{Úvod}{Úvod}

Zaznamenávání a~správa odpracovaného času je problematikou, která může sloužit velkému množství uživatelů různých odvětví zaměstnání. Ať už se jedná o~sledování pracovních hodin za účelem fakturace, za účelem sledování docházky, nebo za účelem analýzy produktivity, jedná se o~službu, kterou vyžaduje a~používá velké množství zaměstnanců a~zaměstnavatelů.

Vzhledem k~tomu, že je poptávka po službě zprostředkovávající měření a~správu odpracovaného času poměrně vysoká, existuje řada řešení, které se tuto službu snaží s~různými přístupy a~funkcionalitami nabízet. Mohou proto nastávat situace, kdy budou uživatelé potřebovat nějakou formu integračního prvku, který bude moct propojovat více řešení a~přístupů, jak odpracovaný čas měřit a~jak ho analyzovat. 

Tato práce nabízí implementaci vlastního řešení aplikace pro měření a~správu odpracovaného času a~jako přidanou hodnotu analyzuje možnosti integrace se spouštěči měření času (fyzické ovladače, automatizace, \dots), možnosti integrace s~existujícími systémy pro správu odpracovaného času, poté navrhuje, jaké typy integrací implementovat, a~v~realizované aplikaci tyto integrace nabízí.

Podkladem pro funkcionality realizované aplikace je analýza existujících řešení a~přístupů pro měření správu času, analýza možností pro integraci a~návrh používání aplikace z~výstupů analýzy.

Implementace aplikace zhotovená v~rámci této práce nemá šanci se vyrovnat existujícím aplikacím řešícím problematiku měření a~správy času, které mají mnohem větší časové a~finanční rozpočty a~stojí za nimi roky vývoje velkých týmů. Má ale šanci poskytnout základ aplikace, která mimo samotnou možnost měření a~správy odpracovaného času dokáže nabídnout i~nějaké vhodné formy integrace se spouštěči měření a~těmito zmíněnými existujícími aplikacemi. Na tomto základu poté může být navázáno dalším úsilím, které funkcionality a~možnosti integrace aplikace rozšíří. 

Mimo poskytnutí základu integrační aplikace pro měření a~správu odpracovaného času má také tato práce potenciál nabídnout seznámení s~aktuálními moderními přístupy vývoje mobilních aplikací a~s~nástroji k~tomu potřebnými.

%---------------------------------------------------------------
\chapter{Cíl práce}
%---------------------------------------------------------------

Hlavním cílem diplomové práce je analýza, návrh a~implementace mobilní aplikace pro systém iOS (frontend i~backend) pro zaznamenávání odpracovaného času, která umožňuje integraci s~různými spouštěči akcí a~propojení s~dalšími systémy pro zaznamenávání a~spravování odpracovaného času. V~rámci návrhu je cílem navrhnout vhodné technické řešení pro implementaci aplikace, funkcionality aplikace a~jejich uživatelské rozhraní a~tohoto návrhu se poté při realizaci aplikace držet. Následně je cílem práce realizované řešení vhodně otestovat a~provést uživatelské testování. 

Vedlejším cílem realizovaného řešení je flexibilita a~rozšiřovatelnost technické implementace aplikace. Jelikož základ integrační aplikace pro měření a~správu odpracovaného času, který tato práce realizuje, má široké možnosti budoucího rozšíření, ať už o~aplikace pro další platformy (Android, Web, \dots), tak o~další funkcionality (propojení s~mnohými dalšími řešeními), tak je přinejmenším vhodné, aby na tyto rozšíření byla co nejlepším způsobem připravena.

Výsledky diplomové práce budou přínosné pro potenciální budoucí uživatele nové realizované platformy, ale také pro čtenáře, kteří se chtějí seznámit s~technickým přístupem řešení, které bylo v~této práci zvoleno, proč bylo zvoleno, jaké byly alternativy a~jaké je poté v~závěru zhodnocení tohoto přístupu. Vývoj mobilních aplikací je~stále rostoucím průmyslem, který je stále ve fázi objevování nových přístupů a~řešení, a~tato práce má potenciál do něj přispět zkušenostmi, které byly při její realizaci získány.

%---------------------------------------------------------------
\chapter{Analýza}
%---------------------------------------------------------------

Tato kapitola rozebírá, k~čemu slouží zaznamenávání odpracovaného času. Poté popisuje možnosti, jak měření času spouštět a~také se věnuje příkladům řešení, která již pro zaznamenávání odpracovaného času existují, a~jaké jsou možnosti propojení s~těmito řešeními. V~dalších sekcích se poté obecně zaměřuje na vývoj aplikací pro mobilní platformu iOS včetně možností multi-platformního a~cross-platformního vývoje.

%---------------------------------------------------------------
\section{Zaznamenávání odpracovaného času}
%---------------------------------------------------------------

Zaznamenávání odpracovaného času může být potřeba z~několika různých důvodů, buď z~pohledu jednotlivce, nebo z~pohledu spolupráce v~nějakém týmu. Mezi tyto důvody může patřit například:
\begin{itemize}
\item\textbf{Sledování pracovních hodin:} Pomáhá zaměstnancům a~podnikům sledovat, kolik času strávili na jednotlivých úkolech, projektech a~aktivitách během pracovní doby. To může být užitečné pro sledování produktivity, hodnocení efektivity práce a~plánování rozpočtu času.
\item\textbf{Fakturace a~účtování:} Pro profesionály a~firmy, které účtují za své služby na základě odpracovaného času, umožňuje zaznamenávání snadné sledování času stráveného na určitých projektech a~klientech pro účely fakturace.
\item\textbf{Analýza produktivity:} Poskytuje data a~statistiky o~tom, jaký čas je věnován různým úkolům a~projektům. To umožňuje identifikovat trendy v~pracovních návycích, optimalizovat časové plánování a~zlepšit efektivitu práce.
\item\textbf{Správa projektů:} Pomáhá organizovat časové údaje spojené s~různými projekty a~úkoly, což usnadňuje plánování, delegování a~monitorování pokroku.
\item\textbf{Transparentnost a~komunikace:} Pro týmy umožňuje transparentně sdílet informace o~čase stráveném na různých aktivitách, což podporuje spolupráci a~komunikaci v~rámci týmu.
\end{itemize}

Celkově zaznamenávání slouží k~lepšímu řízení času, sledování produktivity a~optimalizaci využití pracovního času pro jednotlivce i~organizace.

\begin{figure}[h]
	\centering
	\includegraphics[width=10cm]{clockify-ios.png}
	\caption{Clockify – iOS aplikace pro měření času \cite{clockify-ios}}
	\label{fig:clockify-ios}
\end{figure}

%---------------------------------------------------------------
\section{Spouštěče měření času}\label{tracking-triggers}
%---------------------------------------------------------------

Jedním z~cílů této práce je prozkoumat různá řešení pro spouštěče měření času. Nejjednodušším spouštěčem je samozřejmě ruční zapnutí nějaké formy časovače v~samotné aplikaci, která pro měření času slouží. To ale umí kdejaké již existující řešení a~přidanou hodnotou této práce by měly být možnosti, jak spouštění uživateli ulehčit. Tyto možnosti se dají rozdělit do dvou kategorií – fyzické a~softwarové.

%---------------------------------------------------------------
\subsection{Fyzické}
%---------------------------------------------------------------

Za fyzické spouštěče měření času lze považovat jakoukoli formu fyzického ovladače, kterou může uživatel ovládat a~tím spouštět časovač měření času.

Častým typem fyzického spouštěče je nějaká forma takzvaného platónského tělesa (krychle, osmistěn, dvanáctistěn, atd.), které může uživatel otáčet. Podle toho, na kterou stranu ho položí, se spustí časovač s~danými parametry. Různé strany tělesa mohou sloužit například pro identifikování toho, na kterém projektu uživatel zrovna pracuje.

Mezi fyzické spouštěče patří například tyto produkty:
\begin{itemize}
\item\textbf{TIMEFLIP:} Dvanáctistěnné těleso, na které si uživatel může nalepovat cokoli, co bude identifikovat odpracovaný čas. Těleso se propojí s~mobilní aplikací, která poté spravuje naměřený čas. \cite{timeflip}
\item\textbf{TIMEULAR:} Stejný princip jako \emph{TIMEFLIP}, akorát používá osmistěnné těleso. Jednotlivé strany jdou také přizpůsobovat nálepkami. \cite{timeular}
\item\textbf{timeBuzzer:} Fyzické tlačítko, které lze umístit na stůl a~zapojit do počítače. Stisknutí tlačítka otevře okno měřící aplikace, otočením tlačítka lze vybrat projekt a~opětovným stisknutím se začne daný projekt měřit. \cite{timebuzzer}
\end{itemize}

\begin{figure}[h]
	\centering
	\begin{subfigure}[b]{5.2cm}
		\includegraphics[width=5.2cm]{timeflip.jpeg}
		\caption{TIMEFLIP – Kostka a~aplikace \cite{timeflip}}
		\label{pic:timeflip}
	\end{subfigure}
	\hspace{2cm}
	\begin{subfigure}[b]{7cm}
		\includegraphics[width=7cm]{timebuzzer.jpeg}
		\caption{timeBuzzer – tlačítko a~aplikace \cite{timebuzzer}}
		\label{pic:timebuzzer}
	\end{subfigure}
	\caption{TIMEFLIP a~timeBuzzer}
\end{figure}

Všechny ze zmíněných produktů poskytují veřejné API pro aplikace, se kterými komunikují \cite{timeflip-api} \cite{timeular-api} \cite{timebuzzer-api}. Pokud by tedy bylo potřeba produkty propojit s~vlastní aplikací, která by mohla naměřený čas spravovat, bylo by potřeba, aby aplikace komunikovala s~těmito nástroji. Pořád by byl potřeba prostředník, kterým by byla aplikace daného produktu. Výjimkou je v~těchto produktech pouze \emph{TIMEFLIP}, který poskytuje veřejný protokol pro BLE komunikaci \cite{timeflip-ble-api}. Tento produkt by tedy mohl komunikovat s~novou aplikací napřímo.

Pokud by nová aplikace měla umožňovat budoucí propojení s~jakýmkoli dalším hardwarovým produktem, tak by také měla poskytovat veřejné API, které by takovou komunikaci umožňovalo. Vývoj takových produktů by mohl být předmětem návrhu pro budoucí vylepšení aplikace.

%---------------------------------------------------------------
\subsection{Softwarové}\label{software-tracking-triggers}
%---------------------------------------------------------------

Za softwarové spouštěče měření času lze považovat určité formy integrace a~automatizace v~rámci zařízení, na kterém aplikace běží. V~případě aplikace na platformě iOS to mohou být následující možnosti:
\begin{itemize}
\item\textbf{Automatizace podle polohy:} Aplikace by mohla reagovat na polohu uživatele a~spouštět nebo vypínat časovač na základě informace, kde se uživatel nachází. Uživatel by si mohl například nastavit určitá místa, kde by chtěl, aby se automaticky spustil časovač. Mohl by se zapnout přímo s~konkrétními přednastavenými parametry (projekt, klient, \dots), nebo by se jen mohlo ukázat upozornění, aby si uživatel časovač zapnul sám, s~parametry, které zadá. Podobně by aplikace mohla reagovat i~na opuštění nějakého místa – například pokud uživatel opustí místo, které má označeno jako práci, tak by dostal upozornění, zda si nechce vypnout časovač.
\item\textbf{Automatizace podle času:} Jednoduchá forma automatizace by mohla fungovat na základě času. V~přednastavených časech by se mohl zapnout nebo vypnout časovač s~danými parametry, nebo by aplikace mohla uživatele jen upozornit.
\item\textbf{Automatizace podle kalendáře:} Užitečná by také mohla být integrace s~kalendářem uživatele. Podle naplánovaných událostí by se časovač mohl automaticky zapínat, přepínat, nebo vypínat, s~parametry z~kalendáře. Nebo opět pouze uživatele upozornit, zda si kvůli nějaké události nechce měření času aktualizovat.
\item\textbf{Automatizace podle režimu soustředění:} Systém iOS umožňuje od verze 15.0 uživatelům používat takzvané \emph{Režimy soustředění} \cite{ios-focus-modes}. Uživatel si může konfigurovat různé režimy a~nastavovat, jaké mu v~tomto režimu budou chodit upozornění, jaké aplikace může používat, kdo mu může volat, apod. Od iOS verze 16 také Apple umožňuje vývojářům přizpůsobovat chování aplikace podle toho, jaký režim soustředění je zapnutý \cite{ios-focus-modes-adjustment}. Toho by se dalo vhodně využít například tak, že by se mohl automaticky spouštět časovač (nebo by uživatel mohl být upozorněn) podle toho, do jakého režimu soustředění se uživatel právě přepnul.
\item\textbf{Automatizace pomocí zkratek:} Apple nabízí uživatelům s~iOS verze 13.0 nebo novější aplikaci \emph{Zkratky} \cite{ios-shortcuts-app}, která umožňuje tvorbu vlastních automatizujících procesů \cite{ios-shortcuts}. Tento způsob automatizace umožňuje uživatelům i~vývojářům velkou míru přizpůsobitelnosti – zkratky lze propojovat s~dalšími aplikacemi, s~hlasovým asistentem a~s~mnoha různými akcemi. Všechny předchozí způsoby automatizace lze také nějakým způsobem vytvořit pomocí zkratek. \cite{ios-shortcuts-developer}
\end{itemize}

%---------------------------------------------------------------
\section{Existující systémy pro měření a~správu odpracovaného času}\label{existing-tracking-solutions}
%---------------------------------------------------------------

Systémů pro měření a~správu odpracovaného času existuje mnoho. Existuje dokonce několik různých článků o~tom, které systémy jsou nejlepší a~jaké je jejich srovnání \cite{forbes-tracking-apps-article} \cite{zapier-tracking-apps-article}. Následující výběr příkladů je učiněn podle těchto článků, podle osobní preference a~podle počtu stažení na iOS platformě. U~každého příkladu jsou také rozebrány možnosti importu dat, od kterých se mohou odvíjet požadavky na export dat z~nové aplikace.

%---------------------------------------------------------------
\subsection{Clockify}
%---------------------------------------------------------------

Clockify je jednoduchá a~rozšířená aplikace primárně určená pro měření a~správu odpracovaného času. Mobilní aplikace má více než 1 milion stažení \cite{clockify-app-magic} a~celý systém používají miliony lidí \cite{clockify-customers}. 

V článku Forbes je hodnocena jako celkově nejlepší aplikace pro měření času pro rok 2024. Má široké spektrum funkcionalit – umožňuje měření a~správu času pro různé projekty, klienty a~zařízení. Mimo měření času nabízí také sledování prezence pro mzdy a~účetnictví, optimalizaci produktivity zaměstnanců, měření vykazovatelného času a~sdílení pokroku na projektech s~klienty. \cite{forbes-tracking-apps-article}

Clockify nabízí aplikace pro mnoho platforem: Desktopovou aplikaci pro Mac, Windows a~Linux, doplněk webového prohlížeče pro Chrome, Firefox a~Edge, mobilní aplikaci pro iOS a~Android a~nakonec sdílené Kioskové řešení, na kterém si uživatelé mohou zapisovat příchody, přestávky a~další. \cite{clockify-apps}

V~základní variantě je používání aplikace zdarma. Clockify ale nabízí i~placené plány, které nabízí přidané funkce, jako audity naměřených časů, používání přestávek a~další. Tyto placené plány mohou stát od \$3.99 měsíčně za uživatele až po \$11.99 měsíčně za uživatele. \cite{clockify-pricing} 

\begin{figure}[p]
	\centering
	\includegraphics[width=0.9\textwidth]{clockify-timer.jpg}
	\caption{Časovač na měření času v~aplikaci Clockify \cite{clockify-features}}
	\label{fig:clockify-timer}
	\vspace{1cm}
	\includegraphics[width=0.9\textwidth]{clockify-report.jpg}
	\caption{Reporty v~aplikaci Clockify \cite{clockify-features}}
	\label{fig:clockify-report}
\end{figure}

Clockify nabízí dobře zdokumentované API pro své služby \cite{clockify-api}. Napojení na funkce systému lze tedy jednoduše udělat i~z~vlastního řešení. Do Clockify lze také data importovat z~CSV tabulek, jsou-li data vhodně naformátována \cite{clockify-import-timesheets}.

Mezi nevýhody systému Clockify Forbes uvádí, že projekty nejdou označit jako dokončené, a~že shrnující reporty mohou být zpočátku matoucí. Celkově Clockify ale vnímá jako nejlepší řešení pro pokrytí celé škály funkcionalit.

Na obrázku \ref{fig:clockify-timer} je vidět hlavní obrazovka ve webové verzi aplikace – časovač. Uživatel si zde může časovač ručně zapnout a~vypnout, přidělit projekt, přidat značky, označit vykazovatelnost a~přidat popisek. Také má možnost přidat odpracovaný čas ručně, tedy ne pomocí časovače. K~tomu stačí jen napsat čas začátku a~konce. Pod ovládacím panelem časovače potom uživatel vidí seznam svých již zadaných časových záznamů, které zde může libovolně upravovat.

Dále je na obrázku \ref{fig:clockify-report} vidět obrazovka určená pro reporty. Uživatel si zde může vybrat časový úsek, pro který chce report vypracovat, a~může si nastavit filtry, podle kterých chce časové vstupy filtrovat (tým, klient, projekt, úkol, značka, stav a~popisek). Po aplikování filtru uživatel následně uvidí report své práce vyhovující zadaným parametrům, který si může exportovat ve formátech PDF, CSV a~Excel.

%---------------------------------------------------------------
\subsection{Toggl Track}
%---------------------------------------------------------------

Toggl track je další poměrně rozšířená aplikace pro měření a~správu času. Na platformě iOS má více než 1 milion stažení \cite{toggl-track-app-magic}. 

Forbes tuto aplikaci považuje za nejlepší pro malé týmy, protože nabízí plán zdarma pro týmy do 5 uživatelů a~nabízí neomezený počet klientů a~projektů. \cite{forbes-tracking-apps-article}

Aplikace nabízí určitou formu automatické detekce toho, že uživatel nemá zapnutý časovač, i~když pracuje, nebo automatickou detekci naopak toho, že uživatel už nepracuje, ale časovač má zapnutý. Aplikace nabízí detailní denní, týdenní a~měsíční reporty.

Toggl Track také pokrývá více platforem: Webovou aplikací, mobilní aplikací pro iOS a~Android a~desktopovou aplikací pro Windows a~Mac. \cite{toggl-track}

\begin{figure}[h]
	\centering
	\includegraphics[width=\textwidth]{toggl-track.png}
	\caption{Aplikace Toggl Track \cite{toggl-track}}
\end{figure}

Toggl Track také nabízí možnost importu dat z~CSV tabulky, ve velmi podobném formátu jako Clockify \cite{toggl-track-import-csv}. a~stejně jako Clockify nabízí vlastní otevřené API \cite{toggl-track-api}.

%---------------------------------------------------------------
\subsection{Deputy}\label{deputy}
%---------------------------------------------------------------

Deputy je další aplikací, která umožňuje měření a~správu odpracovaného času. Jejím hlavním účelem je však plánování směn zaměstnanců. Je uvedena jako jeden z~příkladů proto, aby šlo nahlédnout na problém měření času i~z~jiného úhlu, než striktně měření času primárně za účelem vykazování, fakturování nebo analýzy efektivity. Je také velice rozšířenou aplikací – počty stažení na mobilních telefonech překračují 5 milionů \cite{deputy-app-magic}.

Právě plánování směn zaměstnanců je hlavním důvodem, proč Forbes tuto aplikaci ve svém seznamu zmiňuje. Aplikace totiž nabízí plánování neomezeného počtu směn za měsíc. Uživatelé se mohou přihlásit k~aplikacím pro plánování a~pro prezenci zvlášť. I~neplacený plán umožňuje automatické plánování směn, což šetří čas manažerům. Toto plánování umožňuje i~započítání obědových pauz, přestávek a~dalšího, přímo do plánu směn, podle příslušných zákonů.  \cite{forbes-tracking-apps-article}

Deputy nabízí aplikaci pro měření odpracovaného času pro mobilní platformy iOS a~Android. 

\begin{figure}[h]
	\centering
	\includegraphics[width=\textwidth]{deputy.png}
	\caption{Aplikace Deputy \cite{deputy-time-tracking-app}}
\end{figure}

Stejně jako předchozí systémy, Deputy nabízí možnost importu dat z~CSV tabulky \cite{deputy-import-csv}. Vzhledem k~trochu jinému byznysovému modelu této aplikace ale tyto data musí na rozdíl od předchozích systémů obsahovat dodatečná data, jako místo, pauzy na oběd, a~další. a~ani u~této aplikace nechybí otevřené API \cite{deputy-api}.

%---------------------------------------------------------------
\section{Vývoj mobilních aplikací pro systém iOS}
%---------------------------------------------------------------

Tato sekce se věnuje několika klíčovým tématům souvisejícím s~vývojem aplikací pro zařízení s~operačním systémem iOS. Tato témata shrnuje a~poskytuje tím základ pro diskuzi o~vývoji aplikací pro tento systém.

%---------------------------------------------------------------
\subsection{Historie a~vývoj iOS platformy}
%---------------------------------------------------------------

Historie operačního systému iOS začala v~roce 2007, kdy byl představen a~začal se prodávat první mobilní telefon od společnosti Apple, kterým byl iPhone. V~tomto telefonu byl od výroby nainstalován operační systém, který ještě v~tu dobu nenesl název \emph{iOS}, ale \emph{iPhone OS}. Název \emph{iOS} se začal používat až později od verze systému 4. Přestože z~dnešního pohledu chybělo v~systému mnoho funkcí, se kterými si dnes mobilní telefony spojujeme (například možnost instalace aplikací třetích stran), tak v~tu dobu to byl velký posun. Věci jako \emph{Multitouch screen}, vizuální znázornění hlasové schránky, nebo integrace s~\emph{iTunes}, byly ohromnou výhodou. Mezi předinstalované aplikace patřil kalendář, fotky, fotoaparát, poznámky, prohlížeč webu \emph{Safari}, e-mailový klient, telefon a~\emph{iPod} (který se později rozdělil na aplikace pro hudbu a~aplikace pro videa).

\begin{figure}[h]
	\centering
	\includegraphics[width=10cm]{iphone.png}
	\caption{První model mobilního telefonu iPhone \cite{iphone-review}}
\end{figure}

O~rok později, v~roce 2008, byla představena nová generace mobilního telefonu s~názvem iPhone~3G, se kterým také vznikla nová verze operačního systému iOS (iPhone OS) 2.0. Hlavní novinkou této nové verze byl nový obchod s~aplikacemi třetích stran, \emph{App Store}. Až 500 aplikací bylo v~obchodě k~dispozici v~době, kdy byl zpřístupněn uživatelům.

V~roce 2009 poté s~novým iPhonem 3GS přišla verze iOS (iPhone OS) 3, která přinesla hlavně vylepšení, jako možnost používat kopírování a~vkládání, vyhledávání přes \emph{Spotlight}, podporu MMS a~možnost natáčet videa (do té doby iPhone uměl pouze pořizovat fotky). Také to byla první verze operačního systému, která fungovala i~na nových tabletech od společnosti Apple – na iPadech. První iPad byl představen v~roce 2010.

V~roce 2010 byl představen nový iPhone 4 a~s~ním verze operačního systému 4. Od této verze se začal používat název \emph{iOS}, který nahradil do té doby používaný \emph{iPhone OS}. Toto nové pojmenování mělo sjednocovat názvosloví pro více zařízení – iPhone, iPad a~iPod. iPhone 4 byl poměrně velkou změnou, protože disponoval novým hranatým designem, a~systém iOS 4 přinesl novinky, které tento operační systém pomalu začaly rýsovat do dnešní podoby. Mezi novinky patřil \emph{FaceTime}, \emph{multitasking}, \emph{iBooks}, organizování aplikací do složek, osobní hotspot nebo sdílení dat pomocí \emph{AirPlay} a~\emph{AirPrint}. Také to byla první verze iOS, která nepodporovala všechny dosud vydané iPhony, protože nebyla kompatibilní s~prvním iPhonem.

O~rok později, s~iPhonem 4S, byl představen iOS verze 5. Apple v~ní reagoval na rostoucí trend bezdrátovosti a~\emph{cloud computing} představením několika nových funkcí a~platforem. Mezi ně patřil třeba iCloud nebo synchronizace s~iTunes přes Wi-Fi. Také vznikla platforma \emph{iMessage} a~začalo se používat oznamovací centrum.

Dalším modelem byl iPhone 5, který poprvé změnil velikost displeje a~zvětšil ji z~3,5 palce na 4. S~tímto modelem přišla verze iOS 6, kterou doprovázelo mnoho problémů. V~této verzi Apple představil hlasového asistenta \emph{Siri}, který i~přes to, že ho později předčila konkurence, byl poměrně zásadní novinkou. Dále Apple v~nové verzi představil nové aplikace pro mapy, na které do té doby používal řešení od firmy Google. Tyto mapy od začátku obsahovaly mnoho chyb a~celkově byly považovány za dost nedokončené, což způsobilo ve firmě mnoho problémů.

V~roce 2013, s~novým modelem iPhone 5S, přišla verze iOS 7. Tato verze zásadním způsobem změnila vzhled uživatelského rozhraní, které se na začátku od uživatelů netěšilo příliš dobrému přijetí, ale po několika vylepšeních a~po tom, co si uživatelé na nový vzhled zvykli, problémy ustaly. Mezi nové funkce této verze se řadí sdílení dat s~dalšími uživateli přes \emph{AirDrop}, používání operačního systému v~displejích automobilů přes \emph{CarPlay}, ovládací centrum, nebo nový způsob odemykání zařízení pomocí otisku prstu – \emph{Touch ID}.

\begin{figure}[h]
	\centering
	\includegraphics[width=10cm]{ios7.jpg}
	\caption{Nový vzhled iOS verze 7 \cite{ios-7-design}}
\end{figure}

S~dalším rokem a~novým modelem iPhone 6 přišla verze iOS 8. iPhone 6 opět zásadně změnil svůj vzhled, ale v~operačním systému se moc zásadních změn nedělo. Apple se v~této verzi soustředil hlavně na nové funkce. Mezi ně se řadí \emph{Apple Music}, placení přes \emph{Apple Pay}, cloudové úložiště \emph{iCloud Drive}, sdílení práce mezi Apple zařízeními přes \emph{Handoff}, rodinné sdílení a~další.

Rok 2015 přinesl iPhone 6S a~iOS 9. V~této verzi se Apple soustředil hlavně na optimalizaci a~stabilitu, příliš nových funkcí představeno nebylo – pouze nasvícení pro noční používání \emph{Night shift}, režim nízké spotřeby a~možnost používat veřejný testovací beta program.

Dalším modelem byl v~roce 2016 iPhone 7 s~iOS~10. Opět nepřinesl moc nových funkcí, mimo například možnosti mazat původní nainstalované aplikace.

O~rok později poté s~iPhonem~8 přišel iOS~11. V~této verzi se Apple soustředil převážně na nové funkcionality pro iPad, které se jeho použití snažily přiblížit k~plnohodnotnému počítači. Nově přibyly funkce jako podpora tužky \emph{Apple Pencil}, podpora více otevřených oken a~pracovních ploch nebo prohlížeč souborů. Apple ale na závěr představil ještě další nový iPhone – iPhone X. Jednalo se o~další velký redesign, kdy se výrazně zvětšila plocha displeje, zmizelo přední tlačítko a~vzniklo ověření uživatele pomocí obličeje – \emph{Face ID}.

V~roce 2018 Apple představil iPhone~XS a~levnější variantu XR, se kterými přišel iOS~12, kde se jednalo opět jen o~nepatrná vylepšení. 

S~dalším rokem přišel iPhone~11 a~iOS~13. Tento systém už se rozdělil – pro zařízení iPad nyní vznikl oddělený \emph{iPadOS}. V~iOS~13 Apple představil možnost tmavého režimu pro celý systém, nové možnosti bezpečnosti a~soukromí a~s~tím související možnost přihlášení přes Apple.

V~roce 2020 s~iPhonem~12 a~iOS~14 Apple přidal několik menších změn a~vylepšení, jako \emph{Widgety} na domovské obrazovce.

Podobně tomu bylo i~v~roce 2021, kdy Apple s~novým iPhonem~13 a~iOS~15 přidával velké množství menších vylepšení, primárně do vlastních aplikací.

S~rokem 2022 přišel iPhone~14 a~iOS~16, který přinesl sadu vylepšení a~nových funkcí pro zamčenou obrazovku, spolu s~dalšími vylepšeními.

Poslední znatelný update iOS přišel v~roce 2023 s~iPhonem~15 a~iOS~17. a~jak už bylo v~posledních letech zvykem, tak tato aktualizace přinášela větší množství menších vylepšení a~aktualizací, které ale už nijak zásadně neměnily systém a~interakci s~ním. \cite{history-of-ios}

Za těchto 17 let vývoje se operační systém iOS dostal do stavu, kdy ho v~současnou chvíli používá 1,46 miliard aktivních uživatelů. Mobilních telefonů iPhone se od roku 2007 prodalo 2,3 miliard. \cite{iphone-user-statistics}

%---------------------------------------------------------------
\subsection{Vývojové nástroje a~prostředí}\label{ios-dev-tools}
%---------------------------------------------------------------

Jak již bylo zmíněno v~předchozí sekci, obchod s~aplikacemi \emph{App Store} byl uživatelům dostupný od roku 2008. Ve stejnou dobu také Apple zpřístupnil vývojářům iPhone SDK ve svém vývojovém prostředí \emph{Xcode}. Toto vývojové prostředí od Applu již existovalo od roku 2003, kdy vzniklo jako rebranding původního vývojového prostředí \emph{Project Builder}. S~příchodem App Storu Apple umožnil všem vývojářům v~Xcodu vyvíjet aplikace pro iPhone OS od verze 2.0 pomocí zmíněného iPhone SDK. 

Vývojové prostředí Xcode je jednotné integrované vývojové prostředí, které slouží pro vývoj aplikací pro všechny platformy společnosti Apple. Lze stáhnout zdarma z~obchodu \emph{Mac App Store} nebo ze stránek společnosti Apple. Instalace vývojového prostředí obsahuje i~instalaci dalších pomocných nástrojů, jako je simulátor zařízení (iPhone, iPad, \dots), nástroje příkazové řádky, a~další. \cite{xcode-history}

Pro nativní vývoj iOS aplikací se dříve používal programovací jazyk \emph{Objective-C} \cite{objc} a~pro tvorbu uživatelského rozhraní knihovna \emph{UIKit}. V~roce 2014 Apple přidal podporu svého nového programovacího jazyku \emph{Swift} \cite{swift} a~v~roce 2019 přidal podporu nové deklarativní knihovny pro tvorbu uživatelského rozhraní \emph{SwiftUI} \cite{swiftui}.

%---------------------------------------------------------------
\subsection{Architektura aplikací}\label{app-architecture}
%---------------------------------------------------------------

Jako u~každého softwarového projektu, architektura je důležitou součástí jeho návrhu. K~architekturám mobilních aplikací existuje mnoho přístupů. Například u~Android aplikací sám Google doporučuje, jaké architektonické vzory by měly být používány a~doporučuje architektury pro tyto aplikace \cite{android-app-arch}. Apple přímo žádné architektury pro aplikace cílící na iOS platformu nedoporučuje. 

Dokumentace pro iOS platformu ale obsahuje mnoho ukázek kódu věnujících se konkrétním tématům (například \cite{swift-ui-tutorial-complex-interfaces}), kde je vždy nějaký náznak architektury připraven. V~těchto ukázkách se vyskytuje architektura \emph{Model-View-Controller} (zkráceně MVC). Na rozdíl od tradiční MVC architektury se ta používaná Applem mírně liší, jelikož tradiční MVC architektura není moc dobře aplikovatelná na moderní iOS vývoj. Jak lze vidět na obrázcích \ref{fig:mvc}, v~tradičním MVC by mělo \emph{View} být beze stavu a~měl by ho pouze překreslovat \emph{View Controller}. Toto je sice možné v~iOS aplikaci implementovat, ale nedává to moc smysl, protože všechny tyto 3 entity jsou hluboce provázané, což dramaticky snižuje jejich opětovné použití. Při použití v~iOS aplikaci je \emph{Controller} mediátorem mezi \emph{View} a~\emph{Model}, kteří o~sobě navzájem nic nevědí. Jelikož je ale \emph{View Controller} zpravidla velice provázán s~\emph{View}, tak vzniká potřeba psát masivní \emph{View Controller}. Proto se v~iOS vývoji často referuje na MVC jako na \emph{Massive View Controller}.

\begin{figure}[h]
	\centering
	\begin{subfigure}[b]{0.35\textwidth}
		\centering
		\includegraphics[width=6cm]{traditional-mvc.png}
		\caption{Tradiční MVC architektura}
	\end{subfigure}
	\hspace{1cm}
	\begin{subfigure}[b]{0.45\textwidth}
		\centering
		\includegraphics[width=7cm]{apple-mvc.png}
		\caption{MVC architektura v~iOS aplikacích}
	\end{subfigure}
	\caption{Architektura Model-View-Controller \cite{ios-architecture-patterns}}
	\label{fig:mvc}
\end{figure}

Architektura MVC je ale v~ukázkách kódu od Applu používána pravděpodobně převážně proto, že se hodí pro výstižné a~krátké ukázky kódu – obsahuje totiž minimální \emph{overhead}. Pro potřeby větších projektů začíná být nedostačující. Nelze nijak testovat logiku prezentování, a~\emph{View} také nelze pomocí jednotkových testů otestovat. Jediné, co lze testovat, je \emph{Model}. a~velkou nevýhodou je právě zmíněný masivní \emph{View Controller}, který pro složitější obrazovky ztrácí přehlednost.

Velmi rozšířenou architekturou v~iOS vývoji je \emph{Model-View-ViewModel} (zkráceně MVVM). \emph{View} a~\emph{Model} zastávají stejnou funkci jako v~MVC, a~mediátorem je zde \emph{View Model}. \emph{View} je v~tomto případě konkrétní obrazovka a/nebo \emph{View Controller}. \emph{View Model} je nezávislý na knihovně uživatelského rozhraní, což umožňuje jeho lepší testovatelnost. Drží stav obrazovky, vyvolává změny pro \emph{Model} a~aktualizuje se podle něj. Vizuální reprezentace architektury MVVM lze nahlédnout v~obrázku \ref{fig:mvvm}.

\begin{figure}[h]
	\centering
	\includegraphics[width=10cm]{mvvm.png}
	\caption{Architektura MVVM \cite{ios-architecture-patterns}}
	\label{fig:mvvm}
\end{figure}

Další rozšířenou architekturou v~iOS aplikacích je \emph{Viper}. Tato architektura nepochází ze skupiny MV(X). \emph{Viper} se snaží vzít rozdělovaní odpovědností o~krok dále – obsahuje 5 vrstev:
\begin{itemize}
\item\textbf{View:} Zastává stejnou funkcionalitu jako v~MV(X), tedy \emph{View} a/nebo \emph{View Controller}.
\item\textbf{Interactor:} Obsahuje byznysovou logiku související s~daty, jako tvorba instancí nebo načítání dat ze serveru.
\item\textbf{Presenter:} Obsahuje byznysovou logiku UI, ale je nezávislý na UI knihovně. Volá metody \emph{Interactoru}.
\item\textbf{Entities:} Čisté objekty dat.
\item\textbf{Router:} Odpovědný za přechody mezi \emph{Viper} moduly.
\end{itemize}
Tato architektura je poměrně volná v~tom, co bude konkrétní \emph{Viper} modul reprezentovat. Může to být jedna samotná obrazovka, ale může to být celá část aplikace. Lze si všimnout několika rozdílů od architektur ze skupiny MV(X):
\begin{itemize}
\item Model (interakce s~daty) je přesunut do \emph{Interactoru} pomocí \emph{Entities}.
\item Pouze povinnosti UI reprezentace \emph{Controlleru}/\emph{View Modelu} je přesunuta do \emph{Presenteru}, ale ne byznysová logika s~daty.
\item \emph{Viper} explicitně adresuje odpovědnost navigace, kterou řeší \emph{Router}.
\end{itemize}
Vizualizace architektury \emph{Viper} lze nahlédnout v~obrázku \ref{fig:viper}. \cite{ios-architecture-patterns}

\begin{figure}[h]
	\centering
	\includegraphics[width=12cm]{viper.png}
	\caption{Architektura \emph{Viper} \cite{ios-architecture-patterns}}
	\label{fig:viper}
\end{figure}

Existuje také rozšíření architektury MVVM o~navigační logiku, takzvaná MVVM-C, kde písmeno C představuje \emph{Flow Coordinator}. Tato vrstva je odpovědná za navigační tok, vytváří instance \emph{View} a/nebo \emph{View Controller} a~prezentuje je, předává data mezi těmito instancemi a~obsluhuje akce uživatele. Struktura \emph{Flow Coordinatoru} je vhodně navržena tak, aby se dala také dobře testovat. \cite{ios-mvvm-c}

Jednotlivé architektury mají také své aktualizované varianty pro nově používanou deklarativní UI knihovnu \emph{SwiftUI}. Všechny ze zmíněných architektur se dají vhodně použít i~s~touto knihovnou. Ve \emph{SwiftUI} už se jednotlivé obrazovky nerozdělují na \emph{View} a~\emph{View Controller}, ale zůstává zde pouze \emph{View}. \emph{SwiftUI} ale ze své podstaty jako deklarativní knihovna přináší nové možnosti, jak k~architektuře aplikace přistupovat.

Jednou z~novějších architektur, které využívají výhody deklarativních UI knihoven, jako je \emph{SwiftUI}, je \emph{Model-View-Intent} (zkráceně MVI). \emph{View} a~\emph{Model} reprezentují stejné vrstvy jako doposud a~\emph{Intent} reprezentuje nějaký úmysl vyvolat určitou akci. Tím může být například kliknutí uživatele. Konkrétní \emph{Intent} poté pomocí nějaké byznysové logiky aktualizuje stav, podle kterého se následně aktualizuje \emph{View}. \cite{swiftui-mvi}

Všechny výše popsané architektury pracují s~daty pomocí nějakého \emph{Modelu} (\emph{Viper} pomocí \emph{Interactoru}). Práce s~daty ale obvykle není jednoduchá logika, která by si zasloužila pouze zmínku o~tom, že se o~ní stará nějaký \emph{Model}. Může se totiž jednat o~poměrně komplikovanou doménovou a~byznysovou logiku, stahování dat ze serveru, cache dat, práce s~databází, a~další. V~tomto ohledu je namístě diskutovat o~nějaké obecnější architektuře aplikace, ne pouze o~prezentační logice, tedy primárně o~logice uživatelského rozhraní. Obecným zvykem je aplikace rozdělovat do třívrstvé architektury. Tyto vrstvy se mohou nazývat různě, ale obvykle jde o~následující:
\begin{itemize}
\item\textbf{Prezentační vrstva:} Logika uživatelského rozhraní, interakce s~uživatelem, obrazovky, \dots
\item\textbf{Doménová/byznysová vrstva:} Byznysová a~doménová logika.
\item\textbf{Datová vrstva:} Práce s~daty (komunikace se serverem, databáze, \dots)
\end{itemize}
Rozšířená architektura, vhodná pro iOS aplikace, která definuje strukturu aplikace, je \emph{Clean Architecture}. Tato architektura je navržena pro dobrou testovatelnost, rozdělení odpovědnosti, a~vyhovění dalším zašlým návrhovým vzorům týkajících se návrhu softwaru. V~této architektuře se jednotlivé vrstvy skládají z~následujících částí:
\begin{itemize}
\item\textbf{Prezentační vrstva:} \emph{View}, \emph{View Controllery}, \emph{View Modely}, a~další, podle dané architektury (MVC, MVVM, Viper, \dots). Tato vrstva má závislost na doménové vrstvě a~volá její \emph{Use Cases}.
\item\textbf{Doménová vrstva:} Rozhraní \emph{Use Cases} a~jejich implementace, rozhraní \emph{Repositories}, které \emph{Use Cases} používají a~doménové objekty. \emph{Use Cases} obsahují byznysovou logiku aplikace a~datovou logiku nechávají na \emph{Repositories}.
\item\textbf{Datová vrstva:} Implementace \emph{Repositories}, rozhraní \emph{Providers} a~implementace \emph{Providers}. Implementace repozitářů jsou nezávislé na konkrétních knihovnách třetích stran, obsahují logiku s~daty. Implementace \emph{Providers} už jsou závislé na konkrétních knihovnách. Jednotlivé \emph{Providers} se mohou starat například o~práci s~konkrétní databází nebo s~konkrétní knihovnou pro serverovou komunikaci.
\end{itemize}
V~\emph{Clean Architecture} se také pracuje s~\emph{Dependency Injection}, což řeší logiku toho, které konkrétní implementace budou dosazeny jednotlivým rozhraním (\emph{Use Cases, Repositories, Providers}). \cite{ios-clean-arch-mvvm}

%---------------------------------------------------------------
\section{Cross-platformní a~multi-platformní možnosti}\label{crossplatform-multiplatform}
%---------------------------------------------------------------

V~současném prostředí mobilního vývoje se rozhodnutí ohledně vhodné platformy pro tvorbu aplikací stává klíčovým faktorem pro úspěch projektu. Existuje široká škála přístupů, které vývojáři mohou zvolit, od tradičních nativních řešení až po moderní multi-platformní a~cross-platformní frameworky. Tato sekce se zaměřuje na analýzu těchto možností s~ohledem na tvorbu mobilní aplikace pro zaznamenávání odpracovaného času. Zhodnocuje jejich výhody, nevýhody a~vhodnost pro konkrétní aplikaci. Tento přístup umožní lépe pochopit, jakým směrem se vydat při návrhu a~implementaci projektu.

Pro rozdíl mezi cross-platformním a~multi-platformním vývojem neexistuje přesná definice a~v~různých zdrojích lze nalézt různé interpretace (např. \cite{cross-multiplatform-alternative-intepretation-one} a~\cite{cross-multiplatform-alternative-intepretation-two}). V~kontextu vývoje pro mobilní aplikace si budeme tyto pojmy vykládat následovně:
\begin{itemize}
\item Cross-platformní vývoj představuje řešení, ve kterých sdílený kód běží na koncových platformách přímo přes nějakou formu abstrakce nebo interpretace.
\item Multi-platformní vývoj představuje řešení, ve kterých se sdílený kód přímo kompiluje do nativního kódu specifického pro každou platformu.
\end{itemize}

%---------------------------------------------------------------
\subsection{Nativní vývoj}
%---------------------------------------------------------------

Nativní vývoj pro platformu iOS znamená vytváření mobilních aplikací přímo v~jazyce \emph{Swift} nebo \emph{Objective-C} s~využitím oficiálních nástrojů poskytovaných společností Apple, jako je Xcode IDE a~iOS SDK. Tento přístup umožňuje vytvořit aplikaci, která je optimalizovaná pro konkrétní operační systém a~využívá všech funkcí a~výhod, které iOS nabízí.

Výhody nativního vývoje pro iOS spočívají především v~plné integraci s~ekosystémem Apple, což zajišťuje vysokou kvalitu, rychlost a~stabilitu aplikací. Vývojáři mají přístup ke kompletní sadě nástrojů, dokumentace a~podpory přímo od výrobce platformy, což usnadňuje vývoj a~řešení potíží. Díky nativnímu přístupu je možné vytvářet aplikace s~vysokou výkonností a~možnostmi, které jsou na míru prostředí iOS.

Nevýhody nativního vývoje spočívají v~tom, že vyžaduje znalost specifických jazyků a~nástrojů pro každou platformu (\emph{Swift}/\emph{Objective-C} pro iOS, \emph{Kotlin}/\emph{Java} pro Android), což může zvýšit náklady na vývoj a~časovou náročnost. Navíc nativní přístup vyžaduje oddělený vývoj pro každou platformu, což může být neefektivní pro projekty s~omezeným rozpočtem nebo krátkým časovým rámcem.

Nativní vývoj je ideální pro projekty, které se zaměřují na plné využití možností iOS platformy, jako jsou výkonné aplikace nebo aplikace s~náročnějším uživatelským rozhraním. Také je vhodný pro aplikace, které potřebují maximální stabilitu a~bezpečnost, například bankovní aplikace nebo aplikace pro zpracování citlivých údajů. Pro vývojáře, kteří chtějí mít plnou kontrolu nad každým aspektem aplikace a~využít všech funkcí, které iOS nabízí, je nativní vývoj nejlepší volbou.

%---------------------------------------------------------------
\subsection{Cross-platformní vývoj}
%---------------------------------------------------------------

Cross-platformní vývoj se zaměřuje na tvorbu mobilních aplikací, které mohou běžet na více než jedné platformě (např. iOS a~Android) s~využitím jediného kódu a~jednoho vývojového prostředí. Tento přístup umožňuje vývojářům sdílet co nejvíce kódu mezi různými platformami, čímž snižuje náklady a~zjednodušuje správu aplikací pro různé zařízení.

Mezi hlavní výhody cross-platformního vývoje patří efektivita a~rychlost vývoje díky sdílení kódu mezi platformami. Vývojáři mohou využít frameworky jako \emph{Flutter} \cite{flutter-cross-platform}, \emph{React Native} \cite{react-native-cross-platform} nebo \emph{Xamarin} \cite{xamarin-cross-platform}, které umožňují psát kód v~populárních jazycích (např. \emph{JavaScript}, \emph{TypeScript}, \emph{Dart}, \emph{C\#}) a~následně ho spouštět na různých platformách. Tento přístup také usnadňuje aktualizace a~údržbu aplikací, protože změny se projeví na všech podporovaných platformách současně.

Nevýhody cross-platformního vývoje se mohou projevit ve snížené flexibilitě a~omezení přístupu k~některým pokročilým funkcím a~knihovnám specifickým pro danou platformu. Rovněž může docházet k~omezení rychlosti nebo výkonu aplikace v~porovnání s~nativními aplikacemi. Další nevýhodou může být závislost na externích frameworcích a~jejich aktualizacích.

Cross-platformní vývoj je ideální pro projekty, které vyžadují rychlou dostupnost na více platformách s~omezenými zdroji. Hodí se pro aplikace s~jednodušším uživatelským rozhraním, obsahově orientované aplikace (např. novinky, e-commerce), nebo pro firemní aplikace, které nevyžadují specifické funkce jednotlivých platforem. Tento přístup je také vhodný pro malé a~střední projekty, kde je důležitá rychlá a~efektivní tvorba aplikace pro více platforem.

%---------------------------------------------------------------
\subsection{Multi-platformní vývoj}
%---------------------------------------------------------------

Multi-platformní vývoj se liší od cross-platformního vývoje tím, že přímo kompiluje zdrojový kód do nativního kódu specifického pro každou platformu, místo aby spoléhal na vrstvu abstrakce nebo interpretaci. To znamená, že výsledná aplikace běží jako nativní aplikace bez vrstvy prostřednictvím frameworku. Typickým příkladem multi-platformního vývoje je použití jazyka jako \emph{Kotlin} pro Android a~\emph{Kotlin/Native} pro iOS, které se kompilují do nativního kódu pro obě platformy. Tato technologie se nazývá \emph{Kotlin Multiplatform} \cite{kotlin-multiplatform}.

Výhody multi-platformního vývoje zahrnují možnost sdílet větší část kódu mezi různými platformami a~zároveň dosahovat výkonnosti a~funkčnosti nativních aplikací. To znamená, že vývojáři mohou využívat specifické funkce a~knihovny pro každou platformu, aniž by se museli spoléhat na obecné abstraktní vrstvy. Tento přístup také umožňuje lepší optimalizaci výkonu a~přístup k~pokročilým funkcím operačních systémů.

Nevýhody multi-platformního vývoje mohou zahrnovat větší složitost a~náročnost vývoje oproti čistě cross-platformním frameworkům. Každá platforma může vyžadovat jiné postupy a~úpravy, ačkoli základní kód je sdílen. Některé specifické funkce nebo optimalizace pro konkrétní platformu mohou být obtížnější dosáhnout pomocí multi-platformního přístupu než s~nativním vývojem.

Multi-platformní vývoj je vhodný pro projekty, které vyžadují vysokou výkonnost a~přístup k~nativním funkcím a~knihovnám, ale zároveň potřebují sdílet co nejvíce kódu mezi platformami. Ideální je pro rozsáhlejší projekty nebo aplikace, které potřebují plnou integraci s~operačním systémem, ale zároveň chtějí minimalizovat duplicitu práce a~zjednodušit správu a~údržbu kódu. Multi-platformní vývoj je také vhodný pro situace, kdy je důležitá konzistence a~shoda funkcí mezi různými verzemi aplikace na různých platformách.

%---------------------------------------------------------------
\section{Backendová řešení pro mobilní aplikace}\label{backend}
%---------------------------------------------------------------

Backendová část mobilních aplikací hraje klíčovou roli v~poskytování dat, zpracování požadavků a~správě uživatelských účtů a~obsahu. Tato sekce se zaměřuje na analýzu různých backendových řešení a~technologií, které jsou vhodné pro podporu mobilních aplikací. Při výběru správného backendového řešení je důležité pochopit úlohu, kterou backend hraje v~kontextu mobilního prostředí a~identifikovat faktory, které ovlivňují výběr a~implementaci.

Backendová část mobilní aplikace zajišťuje komunikaci mezi klientem (mobilní aplikací) a~serverem, zpracování dat, autentizaci uživatelů, a~další potřebné funkce. Tento centrální prvek infrastruktury zabezpečuje efektivní a~spolehlivé fungování mobilních aplikací, přičemž umožňuje sdílení dat mezi různými zařízeními a~poskytuje uživatelům personalizovaný a~interaktivní zážitek.

Při výběru backendového řešení pro mobilní aplikaci je důležité zvážit několik faktorů. Patří mezi ně škálovatelnost a~výkon serveru, podpora pro bezpečnostní standardy a~autentizaci, možnosti správy uživatelských dat, a~také kompatibilita s~konkrétními požadavky a~technologiemi použitými ve frontendové části aplikace. Dále je důležité zvážit náklady na provoz, údržbu a~rozvoj backendového systému v~průběhu životního cyklu aplikace.

Existuje široká škála backendových technologií a~frameworků, které lze použít při vývoji mobilních aplikací. Od tradičních serverových platforem a~frameworků až po moderní cloudová řešení a~služby. Každá možnost má své vlastní výhody a~nevýhody. Důkladná analýza těchto možností je zásadní k~správnému výběru backendové architektury pro konkrétní mobilní aplikaci.

%---------------------------------------------------------------
\subsection{Backend as a~Service (BaaS) a~jiná delegovaná řešení}\label{baas}
%---------------------------------------------------------------

Tato podsekce se zaměřuje na možnosti, kdy jsou části backendové infrastruktury mobilní aplikace outsourcovány na externí poskytovatele služeb, jako je \emph{Firebase}, \emph{AWS Amplify} nebo \emph{Parse}. Tento přístup umožňuje vývojářům rychle nasadit backendovou část aplikace bez nutnosti spravovat a~udržovat vlastní serverovou infrastrukturu.

Jednou z~hlavních výhod BaaS je urychlení vývoje aplikace a~snížení nákladů a~komplexity provozování backendu. Poskytovatelé BaaS nabízejí hotová řešení pro autentizaci uživatelů, ukládání dat, notifikace, analýzu chování uživatelů a~řadu dalších funkcí, což umožňuje vývojářům zaměřit se více na samotnou funkcionalitu aplikace a~méně na infrastrukturální detaily.

Další výhodou je škálovatelnost a~výkon poskytovaných služeb, který může být optimalizován poskytovatelem a~automaticky přizpůsobován podle potřeb aplikace. Tento přístup je obzvláště užitečný pro malé a~střední projekty s~omezenými zdroji nebo pro projekty, které potřebují rychle nasadit MVP (Minimum Viable Product) bez investice do vlastní infrastruktury.

Nevýhodou použití BaaS může být omezení v~možnostech škálování a~přizpůsobení, zejména u~složitějších aplikací nebo projektů se specifickými požadavky na infrastrukturu. Dále může být problémem závislost na externím poskytovateli služeb a~jejich možná změna podmínek nebo dostupnosti.

Jako příklad BaaS lze uvést \emph{Firebase} od společnosti Google \cite{firebase}. Firebase poskytuje širokou škálu služeb, včetně realtime databáze, autentizace, cloudového úložiště, notifikací, analýzy a~mnoha dalších. Tato platforma je oblíbená mezi vývojáři pro svou jednoduchost a~širokou nabídku poskytovaných funkcí, což umožňuje rychlý vývoj a~nasazení mobilních aplikací s~minimálním úsilím na správu backendové infrastruktury.

Dalšími příklady BaaS služeb jsou \emph{AWS Amplify} od společnosti Amazon \cite{aws-amplify} a~\emph{Parse} \cite{parse}, který je open-source frameworkem pro tvorbu aplikací. Tyto služby nabízejí podobné funkce jako \emph{Firebase} a~umožňují vývojářům využít hotová řešení pro backendovou část svých mobilních aplikací bez nutnosti psát a~spravovat vlastní kód pro serverovou stranu.

%---------------------------------------------------------------
\subsection{Vývoj vlastního backendu}
%---------------------------------------------------------------

Tato sekce se zaměřuje na možnosti vytvoření a~implementace vlastního backendového řešení pro podporu mobilních aplikací. Tento přístup umožňuje vývojářům plnou kontrolu nad backendovou infrastrukturou a~její přizpůsobení specifickým požadavkům aplikace.

Jednou z~hlavních výhod vývoje vlastního backendu je možnost plného přizpůsobení infrastruktury a~funkcí podle konkrétních potřeb mobilní aplikace. Vývojáři mají kontrolu nad škálovatelností, bezpečností a~výkonem backendu, což je zvláště důležité pro aplikace s~vyššími nároky na bezpečnost, správu dat nebo specifické obchodní požadavky.

Další výhodou je snížená závislost na externích poskytovatelích služeb a~jejich změnách v~podmínkách či dostupnosti. Vlastní backend umožňuje také integraci s~existujícími systémy a~infrastrukturou v~organizaci, což může být podstatné pro firemní aplikace nebo projekty s~komplexními integračními požadavky.

Nevýhodou vývoje vlastního backendu může být zvýšená náročnost a~časová zátěž vývoje a~údržby. Vývojáři si musí sami implementovat všechny potřebné funkce backendu, včetně autentizace, ukládání dat, správy uživatelů a~dalších. To může vést ke zvýšeným nákladům a~časovému zpoždění při nasazení aplikace na trh.

Mezi nejpoužívanější řešení pro vývoj vlastního backendu patří frameworky jako \emph{Node.js} \cite{node-js} s~frameworky \emph{Express} \cite{node-js-express} nebo \emph{NestJS} \cite{nest-js} pro \emph{JavaScript}/\emph{TypeScript}, \emph{Ruby on Rails} \cite{ruby-on-rails} pro \emph{Ruby}, \emph{Django} \cite{django} pro \emph{Python} nebo \emph{Spring Boot} \cite{spring-boot} pro \emph{Javu}. Tyto frameworky nabízejí komplexní sadu nástrojů pro rychlý vývoj a~nasazení backendové aplikace s~podporou různých funkcí, včetně routování, databázového přístupu, autentizace a~dalších.

Každý z~těchto frameworků má své výhody a~nevýhody. Například \emph{Node.js} s~\emph{Express} je velmi populární pro svou rychlost a~flexibilitu, zatímco \emph{Spring Boot} je oblíbený pro svou robustnost a~škálovatelnost v~prostředí \emph{Java} \cite{spring-boot-vs-node-js}. Výběr správného frameworku závisí na preferencích vývojářů, technologických požadavcích a~cílech aplikace.

%---------------------------------------------------------------
\subsection{Databáze}
%---------------------------------------------------------------

Implementace databáze je dalším důležitým prvkem vývoje vlastního backendu pro mobilní aplikace, neboť poskytuje úložiště pro data, která aplikace zpracovává a~uchovává. Existuje několik možností pro implementaci databází v~rámci backendového prostředí, které se liší podle typu databáze a~potřeb aplikace.

Mezi nejpoužívanější řešení patří relační databáze, jako je \emph{Oracle Database} \cite{oracle-database} nebo \emph{PostgreSQL} \cite{postgresql}, a~také \emph{noSQL} databáze, jako je \emph{MongoDB} \cite{mongodb} nebo \emph{Redis} \cite{redis}. Relační databáze jsou založené na modelu relačního datového skladu s~použitím \emph{SQL} (Structured Query Language) pro manipulaci s~daty. Tyto databáze jsou vhodné pro aplikace, které vyžadují komplexní transakční operace a~silné zajištění integrity dat. Na druhou stranu \emph{noSQL} databáze jsou navrženy tak, aby byly flexibilnější a~lépe škálovatelné pro různé typy dat. Jsou ideální pro aplikace, které mají různorodá data a~vyžadují rychlý a~flexibilní přístup k~nim.

Výhody relačních databází zahrnují silnou konzistenci dat, vysokou integritu a~možnost provádět komplexní dotazy pomocí SQL. Na druhou stranu \emph{noSQL} databáze nabízejí vyšší škálovatelnost, flexibilitu datového modelu a~lepší výkon pro určité aplikace s~velkým objemem dat. \cite{fit-lecture-no-sql}

Výběr správného typu databáze závisí na specifických požadavcích a~charakteristikách projektu. Pro aplikace s~potřebou silné konzistence a~transakcí jsou vhodné relační databáze, zatímco pro aplikace s~velkým objemem různorodých dat a~vyššími požadavky na škálovatelnost jsou vhodnější \emph{noSQL} databáze.

Výběr implementace databázového řešení lze také outsourcovat na externí poskytovatele služeb BaaS. Jak bylo zmíněno v~sekci \ref{baas}, poskytovatelé jako \emph{Firebase} \cite{firebase} nebo \emph{AWS Amplify} \cite{aws-amplify} již implementují databázové řešení, které lze v~rámci těchto platforem využít. Existují také řešení, která poskytují pouze databázi jako externí řešení. Tato řešení jsou někdy nazývána DBaaS (Database as a~Service) \cite{mongodb-dbaas}. Příkladem tohoto řešení je například \emph{MongoDB Atlas} \cite{mongodb-atlas}, právě pro databázi \emph{MongoDB}.

V implementaci vlastního backendu pro mobilní aplikace je důležité pečlivě zvážit výběr databázového řešení, aby byly splněny požadavky na výkon, škálovatelnost a~bezpečnost aplikace. Zvážení výhod a~nevýhod jednotlivých databázových systémů pomůže zajistit optimální implementaci a~správu dat pro mobilní aplikaci pro zaznamenávání odpracovaného času.

%---------------------------------------------------------------
\section{Závěr analýzy}
%---------------------------------------------------------------

V~této kapitole byly probrány všechny informace potřebné pro navazující návrh mobilní aplikace pro zaznamenávání odpracovaného času. 

Co se týče domény problému, byla provedena rešerše spouštěčů měření času, ze které mohou být vhodně navrženy požadavky pro aplikaci na propojení s~nimi. Dále byla také provedena rešerše existujících systémů pro zaznamenávání odpracovaného času, ze které lze také vhodným způsobem navrhnout požadavky na integraci s~těmito systémy.

Dále se kapitola věnovala vývoji mobilních aplikací pro systém iOS, různým možnostem cross-platformního a~multi-platformního vývoje a~možnostem, jak přistupovat k~backendovým řešením podporující mobilní aplikace. Tyto informace by měly být plně postačující pro vhodný návrh funkcionalit aplikace, jejího uživatelského rozhraní, vnitřní architektury a~nástrojů k~tomu potřebným.













































%---------------------------------------------------------------
\chapter{Návrh}
%---------------------------------------------------------------

Tato kapitola se věnuje návrhu aplikace, tedy technologické architektury celé platformy a jednotlivých implementací, návrhu funkcionalit a uživatelskému rozhraní aplikace. V~následujících sekcích budou rozebírány jednotlivé navržené funkcionality a jejich rozhraní. Nejprve je ale potřeba si stanovit nějaký vzhledový styl aplikace.

Pro návrh uživatelského rozhraní aplikace byl použit nástroj \emph{Figma} \cite{figma}, dále byly použity zdroje z~šablony \emph{Apple Design Resources} \cite{apple-design-resources}.

%---------------------------------------------------------------
\section{Vzhledový styl aplikace}
%---------------------------------------------------------------

Aplikace cílí na platformu iOS, což bude důležitou součástí jejího návrhu. Apple definuje rozsáhlou příručku pro návrh uživatelského rozhraní pro platformu iOS \cite{apple-design-guidelines-ios} a návrh aplikace se touto příručkou bude v~mnoha ohledech řídit.

Každá aplikace má nějaký svůj vzhledový styl, který definuje základní barvy, které bude aplikace používat, vzhledy tlačítek, textových polí, fontů a dalšího. Následující definice těchto prvků vychází převážně z~osobní preference, která se soustředí spíše na jednoduchost a ne příliš velkou výraznost rozhraní. Cílem tedy bude se přiblížit systémovému vzhledu platformy iOS a přidat vlastní mírný vzhledový jazyk.

%---------------------------------------------------------------
\subsection{Barvy}
%---------------------------------------------------------------

Základní návrh barev aplikace lze nahlédnout v~obrázku \ref{fig:colors}. Barvy jsou navrženy tak, aby vždy vznikl dostatečný kontrast mezi barvou pozadí a barvou popředí.

\begin{figure}[h]
	\centering
	\includegraphics[width=\textwidth]{colors.png}
	\caption{Vzhledový styl aplikace – Barvy}
	\label{fig:colors}
\end{figure}

%---------------------------------------------------------------
\subsection{Fonty}
%---------------------------------------------------------------

Základní návrh fontů lze nahlédnout v~obrázku \ref{fig:fonts}. Daný font bude vždy používat systémovou rodinu fontů, tedy obvykle \emph{San Francisco} (SF). Jednotlivé velikosti jsou pouze referenční, protože aplikace by měla podporovat dynamické fonty a reflektovat tak škálování uživatele. Daná velikost je tedy velikost pro výchozí nastavení škálování textu.

\begin{figure}[h]
	\centering
	\includegraphics[width=\textwidth]{fonts.png}
	\caption{Vzhledový styl aplikace – Fonty}
	\label{fig:fonts}
\end{figure}

%---------------------------------------------------------------
\subsection{Prvky}
%---------------------------------------------------------------

Návrh prvků rozhraní vychází z~již definovaných barev a fontů, lze jej nahlédnout v~obrázku \ref{fig:elements}.

\begin{figure}[h]
	\centering
	\includegraphics[width=\textwidth]{elements.png}
	\caption{Vzhledový styl aplikace – Prvky}
	\label{fig:elements}
\end{figure}

Na všechny ostatní prvky, jako prvky navigace, seznamy, alerty, a další, bude využito systémových prvků. Tím bude nejlépe vyhověno vzhledové příručce firmy Apple, pouze budou upravné některé barvy těchto prvků, aby ladily k~vzhledu aplikace.

%---------------------------------------------------------------
\subsection{Název a ikona}
%---------------------------------------------------------------

Návrh chytlavého názvu bývá obvykle složitá věc. Pro tuto aplikaci byl zvolen název \emph{Trackee} (anglicky [tra·ki]), který je odvozen z~anglického pojmu \emph{Time tracking}, což představuje měření odpracovaného času. Koncovka \emph{-ee} je také poslední dobou častou volbou pro názvy různorodých aplikací, jako \emph{Spendee} \cite{spendee}, \emph{Fondee} \cite{fondee} a další. Pod tímto názvem není registrována žádná ochranná známka \cite{upd-database}, ani není veden žádný záznam u~správce české domény \cite{cz-nic-trackee}.

Ikona aplikace také neprocházela nijak složitým procesem návrhu, byl pouze použit systémový symbol časovače na pozadí s~barvami aplikace popředí a pole. Ikonku lze nahlédnout v~obrázku \ref{fig:app-icon}.

\begin{figure}[h]
	\centering
	\includegraphics[width=5cm]{trackee.png}
	\caption{Ikona aplikace}
	\label{fig:app-icon}
\end{figure}

%---------------------------------------------------------------
\section{Funkcionality aplikace a jejich uživatelské rozhraní}
%---------------------------------------------------------------

Aplikace bude primárně sloužit pro zaznamenávání odpracovaného času. Je tedy potřeba, aby každý uživatel měl možnost si vytvářet vlastní záznamy a další data, která budou propojena pouze s~ním, a ke kterým bude mít přístup pouze on. Toto obvyklý případ užití mobilní aplikace, který ze své podstaty vyžaduje nějakou formu vytvoření uživatelského účtu, se kterým budou data propojena, a jeho autentizace. Nejobvyklejším způsobem autentizace je autentizace pomocí e-mailu a hesla. Tento způsob je i~poměrně jednoduchý z~hlediska implementace a spousta poskytovatelů BaaS (Backend as a Service, vizte \ref{baas}) tento způsob autentizace implementuje. 

%---------------------------------------------------------------
\subsection{Přihlášení a registrace}
%---------------------------------------------------------------

Přihlašovací obrazovka bude obsahovat pouze nadpis, pole pro vyplnění e-mailu, hesla, primární tlačítko pro přihlášení a sekundární tlačítko pro registraci. Obrazovka pro registraci, která se otevře po kliku na tlačítko pro registraci, poté bude od uživatele potřebovat také jen e-mail a heslo, které je ale zvykem napsat dvakrát, aby se snížila šance, že se v~něm vyskytl překlep. Obrazovky přihlášení a registrace lze nahlédnout v~obrázku \ref{fig:onboarding}. Úspěšné přihlášení a registrace uživatele přesměruje na hlavní obrazovku aplikace.

\begin{figure}[h]
    \centering
    \begin{subfigure}[b]{0.4\textwidth}
		\centering
		\includegraphics[width=6cm]{login.png}
		\caption{Přihlášení}
		\label{fig:login}
	\end{subfigure}
	\hspace{2cm}
	\begin{subfigure}[b]{0.4\textwidth}
		\centering
		\includegraphics[width=6cm]{register.png}
		\caption{Registrace}
	\end{subfigure}
	\caption{Onboarding}
	\label{fig:onboarding}
\end{figure}

%---------------------------------------------------------------
\subsection{Lišta karet a časovač}
%---------------------------------------------------------------

Navigace mezi hlavními obrazovkami aplikace bude řešena pomocí lišty karet, jelikož se jedná o~častý a doporučený způsob, jak navigovat mezi vzájemně exkluzivními částmi obsahu \cite{apple-guidelines-tabbars}. Hlavní obrazovkou bude přehled, na kterém bude uživatel moct ovládat časovač, a kde uvidí historii svých časových záznamů, seřazenou od nejnovějších po nejstarší. Jelikož pro uživatele je nejjednodušší dosáhnout na ovládací prvky, které jsou ve spodní části displeje, bude ovládání časovače umístěno ve spodní části obrazovky, a časové záznamy se budou řadit nad ním. Návrh této obrazovky lze nahlédnout na obrázku \ref{fig:timer}.

\begin{figure}[h]
    \centering
    \begin{subfigure}[b]{0.4\textwidth}
		\centering
		\includegraphics[width=6cm]{timer.png}
		\caption{Časovač}
		\label{fig:timer}
	\end{subfigure}
	\hspace{2cm}
	\begin{subfigure}[b]{0.4\textwidth}
		\centering
		\includegraphics[width=6cm]{project-selection.png}
		\caption{Výběr projektu}
		\label{fig:project-selection}
	\end{subfigure}
	\caption{Hlavní obrazovka}
	\label{fig:timer-and-project-selection}
\end{figure}

Ovládací panel pro časovač může mít různé varianty, jak bude vypadat, podle toho, v~jakém je stavu. Návrh počítá se dvěma možnostmi, jak půjde přidávat časové záznamy do historie:
\begin{itemize}
\item Pomocí časovače – uživatel zapne časovač, když bude chtít začít měření, a poté ho vypne, když bude měření chtít ukončit, čímž se automaticky uloží záznam se zadanými vlastnostmi.
\item Ručně – uživatel ručně zadá začátek a konec záznamu a poté časový záznam uloží.
\end{itemize}
Ovládací panel půjde přepínat mezi těmito dvěma stavy pomocí vedlejšího tlačítka. Hlavní ovládací tlačítko bude vypínat/zapínat časovač, pokud bude přepnut do stavu časovače, nebo bude přidávat ruční záznam, pokud bude v~ručním stavu. V~ovládacím panelu časovače bude také možné vybrat projekt a popis, které budou k~danému záznamu přiděleny. Klik na volbu projektu otevře novou obrazovku, která umožní vyhledávání v~projektech a výběr projektu, jak lze vidět na obrázku \ref{fig:project-selection}. Různé stavy časovače (časovač/manuální, zapnutý/vypnutý, vyplněný/nevyplněný, atd.) lze nahlédnout v~obrázku \ref{fig:timer-control-variants}.

\begin{figure}[h]
	\centering
	\includegraphics[width=\textwidth]{timer-control-variants.png}
	\caption{Různé stavy ovladače pro časovač}
	\label{fig:timer-control-variants}
\end{figure}

Na obrázku \ref{fig:timer} lze dále v~horní části obrazovky vidět souhrn časových záznam za tento den a týden. Uživatel tak uvidí, kolik času již odpracoval v~daný den i~týden. Časové záznamy v~historii budou také seskupeny podle dnů – každý den bude mít nadpis s~datem a součtem odpracovaného času za ten den. Jednotlivé záznamy také půjdou mazat pomocí posuvného gesta.

Vzhledem k~tomu, že v~historii se může časem nacházet mnoho záznamů, měla by tato obrazovka podporovat stránkování, tedy funkci, že nebude ze zdroje načítat všechny záznamy najednou, ale pouze nějaký kus (stránku), a postupně může načítat další, pokud si to uživatel bude přát. Pokud se během načítání objeví chyba, tak se na této obrazovce ovládací panel časovače ani historie záznamů vůbec nezobrazí – zobrazí se pouze popis chyby a tlačítko pro opakování pokusu o~načtení. Pokud uživatel zatím žádné záznamy mít nebude, tak nebude potřeba zobrazovat nějakou explicitní formu prázdné obrazovky – pouze bude ve spodní části obrazovky ovládací panel a nad tím prázdno.

U~obrazovky pro výběr projektu bude také explicitní chybový stav, který ukáže popis chyby včetně tlačítka pro opakování. Na této obrazovce už ale bude potřeba definovat i~prázdný stav, aby se zobrazila nějaká instrukce, že uživatel nemá vytvořené žádné projekty a musí si je vytvořit na místě k~tomu určeném (bude navrženo dále). Prázdná data lze ale ještě rozdělit do dvou kategorií – prázdná data kvůli tomu, že uživatel žádné projekty nemá, nebo prázdná data kvůli tomu, že jeho vyhledávání neodpovídá žádný projekt. Pro tyto dva stavy je potřeba použít rozdílné texty pro uživatele.

%---------------------------------------------------------------
\subsection{Profil uživatele}
%---------------------------------------------------------------

Poslední kartou v~navigační liště aplikace bude karta s~Profilem. Uživatel zde bude mít základní přehled a akce týkající se jeho účtu, jak lze nahlédnout na obrázku \ref{fig:profile-overview}. Uživatel se odsud dostane do seznamu klientů a seznamu profilů, dále si může pomocí tlačítka svůj účet smazat, nebo se odhlásit, což ho vrátí zpět na přihlašovací obrazovku \ref{fig:login}. Mazání účtu je nevratná akce, která vymaže spoustu dat spojených s~uživatelem, je tedy potřeba alespoň ukázat ověřující dialog, který lze nahlédnout na obrázku \ref{fig:profile-delete}. 

\begin{figure}[h]
    \centering
    \begin{subfigure}[b]{0.4\textwidth}
		\centering
		\includegraphics[width=6cm]{profile.png}
		\caption{Přehled}
		\label{fig:profile-overview}
	\end{subfigure}
	\hspace{2cm}
	\begin{subfigure}[b]{0.4\textwidth}
		\centering
		\includegraphics[width=6cm]{profile-delete.png}
		\caption{Smazání účtu}
		\label{fig:profile-delete}
	\end{subfigure}
	\caption{Profil}
	\label{fig:profile}
\end{figure}

V~horní části obrazovky se také nachází uživatelův e-mail, jako indikace toho, na kterém účtě je uživatel přihlášen.

Pokud uživatel klikne na tlačítko klientů, zobrazí se mu seznam jeho klientů, jak lze vidět na obrázku \ref{fig:client-list}. V~tomto seznamu může v~klientech vyhledávat, otevřít detail klienta, nebo vytvořit nového, pomocí tlačítka vpravo nahoře v~navigační liště. Při kliknutí na konkrétního klienta, s~cílem zobrazit jeho detail, i~při kliknutí na volbu tvorby nového klienta, se zobrazí stejná obrazovka detailu, kterou lze nahlédnout na obrázku \ref{fig:new-client}. V~případě zobrazení detailu již existujícího klienta se akorát změní název v~navigační liště, předvyplní se hodnoty klienta a zobrazí se navíc tlačítko pro smazání klienta.

Při tvorbě nebo úpravě klienta uživatel může zvolit jeho název. V~budoucnu je možné přidat další parametry, které by mohly být ke klientovi přiděleny. Kliknutím na tlačítko \emph{Uložit} v~navigační liště se upravený klient uloží a uživatel bude odnavigován zpět na seznam klientů. V~případě, že uživatel klikne na tlačítko zrušit, bude také odnavigován zpět na seznam, ale všechny změny budou zahozeny.

Seznam projektů by měl mít také definované stavy pro chybu a prázdná data. V~případě chyby se zobrazí popis chyby a tlačítko pro opakování, v~případě, že uživatel nemá žádné klienty, se zobrazí tato informace a instrukce k~tomu, aby si nějakého klienta vytvořil. Opět je také potřeba rozlišit mezi tím, zda se žádní klienti nezobrazují proto, protože žádní nejsou, nebo protože žádní nevyhovují vyhledávánému výrazu.

Detail klienta bude mít také explicitní chybový stav, ale prázdný stav zde potřeba není, jelikož existující klient musí mít data vždycky, a nový klient určitě žádná nemá, tudíž budou pole prázdná.

\begin{figure}[h]
    \centering
    \begin{subfigure}[b]{0.4\textwidth}
		\centering
		\includegraphics[width=6cm]{clients.png}
		\caption{Seznam}
		\label{fig:client-list}
	\end{subfigure}
	\hspace{2cm}
	\begin{subfigure}[b]{0.4\textwidth}
		\centering
		\includegraphics[width=6cm]{new-client.png}
		\caption{Nový klient}
		\label{fig:new-client}
	\end{subfigure}
	\caption{Klienti}
	\label{fig:clients}
\end{figure}

Seznam projektů a detail projektu funguje stejným způsobem, jako u~klientů. Kliknutí na tlačítko projektů v~profilu otevře seznam, odkud lze otevřít detail/tvorbu nového projektu. Seznam projektů lze nahlédnout na obrázku \ref{fig:project-list} a detail projektu na obrázku \ref{fig:new-project}.

Detail projektu narozdíl od klienta obsahuje více informací. Každý projekt musí patřit k~nějakému klientovi, tudíž je potřeba tohoto klienta zvolit, k~čemuž bude sloužit další obrazovka pro výběr klienta, která bude vypadat stejně, jako seznam klientů \ref{fig:client-list}, ale funkčně bude stejná, jako výběr projektu v~časovači \ref{fig:project-selection}. Dále je možnost nastavit jméno projektu, a poté nepovinný údaj o~typu projektu, což bude definovaný seznam hodnot, ze kterého půjde volit (\emph{Práce}, \emph{Škola} a další).

Chybové a prázdné stavy budou fungovat stejným způsobem, jako u~klientů.

\begin{figure}[h]
    \centering
    \begin{subfigure}[b]{0.4\textwidth}
		\centering
		\includegraphics[width=6cm]{projects.png}
		\caption{Seznam}
		\label{fig:project-list}
	\end{subfigure}
	\hspace{2cm}
	\begin{subfigure}[b]{0.4\textwidth}
		\centering
		\includegraphics[width=6cm]{new-project.png}
		\caption{Nový projekt}
		\label{fig:new-project}
	\end{subfigure}
	\caption{Projekty}
	\label{fig:projects}
\end{figure}

%---------------------------------------------------------------
\subsection{Integrace}
%---------------------------------------------------------------

Jedním z~cílů práce je stanovit požadavky pro integraci spouštěčů měření času, dále požadavky na integraci aplikace s~existujícími systémy pro měření času, a tyto požadavky v~aplikaci implementovat. Tyto dva typy integrací lze rozdělit do skupin \emph{import} (integrace se spouštěči) a \emph{export} (integrace s~existujícími systémy).

%---------------------------------------------------------------
\subsubsection{Import}
%---------------------------------------------------------------

V~sekci \ref{tracking-triggers} byly popsány teoretické možnosti, s~jakými spouštěči by aplikace šla propojit.

Prvním typem spouštěčů měření času byly fyzické spouštěče, tedy nějaké fyzické formy ovladače. Z~uvedených příkladů v~analýze byl pouze jeden, který by uměl uskutečnit napojení na aplikaci napřímo, a to \emph{TIMEFLIP} \cite{timeflip}, který poskytuje protokol pro BLE komunikaci. Implementace BLE komunikace s~hardwarovým produktem je ale poměrně komplexní záležitost, která byla začleněna nad rámec rozsahu této práce, která se už takto věnuje návrhu a implementaci celé platformy pro měření odpracovaného času. Implementace tohoto typu komunikace tedy může být podnětem pro budoucí rozšíření aplikace.

Důležité ale je, aby na takové rozšíření byla aplikace dobře připravená, a aby minimálně poskytovala veřejné API umožňující se z~jakéhokoli budoucího konfigurovatelného hardwarového řešení na aplikaci napojit.

Dalším typem spouštěčů byly softwarové spouštěče, tedy nějaké formy automatizace. V~sekci \ref{software-tracking-triggers} zabývající se tomuto tématu bylo rozebráno několik typů automatizace (podle polohy, času, režimu soustředění, a podobně). Také bylo ale zmíněno systémové řešení pomocí aplikace \emph{Zkratky} \cite{ios-shortcuts-app}, které dokáže automatizaci všech těchto typů vytvořit. \emph{Zkratky} poskytují společné systémové API, které je velmi rozšířené a Apple se snaží vývojáře přesvědčit, aby nějakou část funkcionality aplikace na toto API napojili. Potenciál této platformy je velký, protože její API umožňuje propojování různých procesů navzájem, aplikace si můžou přeposílat výsledky jednotlivých procesů, navazovat na ně a pracovat s~nimi. 

Aplikace se mohou na toto API napojit pomocí definic takzvaných \emph{App Intents}. Apple v~jejich dokumentaci \cite{ios-app-intents} poskytuje podrobné informace k~tomu, jak tyto \emph{App Intents} definovat, jak pro ně vytvářet parametry, a další. Tyto struktury reprezentují nějakou akci aplikace, kterou uživatel může spustit. Nejpoužívanější (podle očekávání) automatizovatelné funkce aplikace budou zapínání časovače, vypínaní časovače a případně rušení časovače. Bylo by tedy vhodné pro tyto tři funkce vytvořit \emph{App Intents}, které uživatelům umožní vytvářet zkratky a automatizace pro jejich spuštění. V~případě zapínání časovače se také naskytuje možnost použití dvou nepovinných parametrů, a to projekt a popis, který bude uživatel chtít k~záznamu přiřadit.

Implementace integrace se systémovými zkratkami umožní širokou a potenciálně velmi rozšiřovatelnou míru integrace, protože se jedná o~společné systémové API, na které lze napojit jakoukoli aplikaci, a spousta aplikací nějakou formu napojení na zkratky implementuje. U~systémových aplikací to dokonce platí pro všechny – pokrytí automatizovatelných funkcí je zde velmi široké.

%---------------------------------------------------------------
\subsubsection{Export}
%---------------------------------------------------------------

V~sekci \ref{existing-tracking-solutions} byly popsány populární systémy \emph{Clockify}, \emph{Toggl Track} a \emph{Deputy}, které poskytují funkce pro zaznamenávání odpracovaného času. Všechny tyto systémy poskytují možnost importu dat z~CSV souborů, a u~\emph{Clockify} a \emph{Toggl Track} šlo o~velmi podobné formáty.

Možnost exportu z~aplikace do CSV souboru by tedy byla silným nástrojem, pomocí kterého by se záznamy z~aplikace mohly nejen importovat do těchto systémů, ale byla by uživatelům poskytnuta volná možnost, co s~exportovanými daty dělat. V~různých programech by si je mohli dle svého uvážení analyzovat, vizualizovat, a další. Možnost exportu dat do CSV souboru by tedy v~aplikaci neměla chybět. Ideálně by mělo jít o~formát, který půjde importovat jak do systému \emph{Clockify}, tak do systému \emph{Toggl Track}. Systém \emph{Deputy} cílí na trochu jinou cílovou skupinu, napojení na něj by proto nebylo tolik relevantní, jak již bylo popsáno v~sekci \ref{deputy}.

Všechny zmíněné systémy také poskytovaly veřejné API pro napojení na jejich struktury. Zde už se nepůjde spolehnout na nějaké obecné řešení, které bude pasovat na více systémů, ale bude se potřeba na každý systém napojit zvlášť. V~rámci implementace by mohla být vytvořena integrace alespoň s~jedním systémem, a zároveň by implementace mohla být připravena pro rozšíření o~napojení na další systémy.

Rozhraní pro integraci s~dalšími systémy bylo navrženo následovně. V~pořadí druhá, tedy prostřední, karta v~navigační liště bude sloužit pro vytváření integrací. Aby aplikace mohla v~budoucnu podporovat různé typy integrací a zároveň aby si uživatel mohl svá nastavení integrací ukládat, bude hlavní obrazovku integrací představovat seznam již vytvořených integrací, který lze nahlédnout na obrázku \ref{fig:integration-list}.

\begin{figure}[h]
    \centering
    \begin{subfigure}[b]{0.4\textwidth}
		\centering
		\includegraphics[width=6cm]{integrations.png}
		\caption{Seznam}
		\label{fig:integration-list}
	\end{subfigure}
	\hspace{2cm}
	\begin{subfigure}[b]{0.4\textwidth}
		\centering
		\includegraphics[width=6cm]{new-integration.png}
		\caption{Nová integrace}
		\label{fig:new-integration}
	\end{subfigure}
	\caption{Integrace}
	\label{fig:integrations}
\end{figure}

V~tomto seznamu budou prvky jednotlivých vytvořených integrací, které budou mít přidělenou ikonku podle toho, o~jaký typ integrace se jedná (CSV, \emph{Clockify}, a další). Vpravo nahoře v~navigační liště bude tlačítko pro přidání nové integrace, které otevře dialog s~možnostmi, jaký typ integrace chce uživatel vytvořit, jak lze nahlédnout na obrázku \ref{fig:new-integration}. Tento seznam bude mít definovaný chybový stav, kde se místo seznamu zobrazí popis chyby a tlačítko pro opakování, a prázdný stav, který zobrazí jen informaci o~tom, že si uživatel žádné integrace zatím nevytvořil.

Vytvoření nové integrace, nebo otevření detailu existující integrace, otevře obrazovku pro detail integrace, jak lze vidět na obrázku \ref{fig:clockify-integration}. Obrázek \ref{fig:clockify-integration} reprezentuje detail \emph{Clockify} integrace, u~které je potřeba nastavit název integrace, API klíč pro napojení na \emph{Clockify} účet a případně nějaké další parametry pro exportování dat. Tento detail se bude lišit pro různé typy integrací podle toho, jaké parametry budou pro exportování dat potřeba. Například u~exportu do CSV dat bude potřeba jen název a nic dalšího. Tato obrazovka bude potřebovat definic chybového stavu, který bude opět zobrazovat popis chyby a tlačítko pro opakování. Prázdný stav zde nemá smysl. Obrazovky pro vytvoření nové integrace, nebo pro úpravu existující integrace, se budou lišit akorát v~možnosti odstranění integrace – u~existující integrace se přidá tlačítko, které to umožní.

Kliknutí na tlačítko exportu dat otevře další obrazovku, která lze vidět na obrázku \ref{fig:export-data}. Tato obrazovka bude sloužit pro manuální export dat v~zadaném období. U~některých typů integrací totiž bude dávat smysl nabídnout automatický export dat, například u~\emph{Clockify} integrace, kde je možné každý nově vytvořený záznam rovnou automaticky exportovat do \emph{Clockify} účtu. Ale například u~exportu do CSV souboru taková automatizace nedává smysl, tento typ integrace bude tedy sloužit pouze pro manuální export. Tato obrazovka je poměrně jednoduchá, vyžaduje po uživateli pouze zadání od kdy a do kdy chce časové záznamy exportovat. Pokud zadá validní interval a klikne na tlačítko pro export, aplikace data exportuje. V~případě \emph{Clockify} integrace se pokusí data odeslat na \emph{Clockify} API, v~případě CSV integrace nabídne uživateli výsledný soubor, který si s~ním poté může dělat co chce – uložit do souborů, odeslat e-mailem, otevřít v~jiné aplikaci, atd. U~této obrazovky není potřeba definovat žádné chybové nebo prázdné stavy, protože samotná obrazovka žádná data nepotřebuje.

\begin{figure}[h]
    \centering
    \begin{subfigure}[b]{0.4\textwidth}
		\centering
		\includegraphics[width=6cm]{clockify-integration.png}
		\caption{Clockify integrace}
		\label{fig:clockify-integration}
	\end{subfigure}
	\hspace{2cm}
	\begin{subfigure}[b]{0.4\textwidth}
		\centering
		\includegraphics[width=6cm]{export-data.png}
		\caption{Export dat}
		\label{fig:export-data}
	\end{subfigure}
	\caption{Detail integrace}
	\label{fig:integration-detail}
\end{figure}

%---------------------------------------------------------------
\section{Architektura}
%---------------------------------------------------------------

V~analýze v~sekcích \ref{app-architecture}, \ref{crossplatform-multiplatform} a \ref{backend} byly popsány různé architektury pro mobilní aplikace, backendová řešení a databáze. Tato sekce se bude věnovat návrhu architektury celé platformy a nástrojů či jiných produktů, které budou potřeba pro realizaci aplikace.

%---------------------------------------------------------------
\subsection{Architektura platformy}
%---------------------------------------------------------------

V~první řadě je potřeba se rozhodnout, jakou architekturu bude mít celá platforma, tedy jaký přístup bude zvolen pro vývoj mobilní aplikace, jaké řešení bude zvoleno pro implementaci backendu a jaká bude zvolena implementace databáze. Kombinace těchto řešení lze seřadit na stupnici, která má dva konce:
\begin{itemize}
\item Outsourcování co největší části platformy na externí poskytovale (BaaS, DBaaS, atd.). Tento přístup by byl nejjednodušší z~hlediska komplexní a časové náročnosti implementace, ale ze své podstaty by byl nejvíce omezující ve flexibilitě a možnosti budoucích rozšíření. V~případě, že by bylo v~budoucnu rozhodnuto, že se například změní poskytovatel databáze, tak by bylo značně náročnější implementaci pro takovou změnu upravit. Zejména v~případě, kdyby na jednom externím poskytovateli záviselo více částí infrastruktury.
\item Vlastní implementace všech částí platformy. Toto by vyžadovalo vlastní implementaci nativní aplikace, backendu i~databáze. V~případě, že by bylo v~budoucnu rozhodnuto, že se aplikace rozšíří například o~Android aplikaci, musela by být implementována tato aplikace celá, ale zase by se jen napojila na existující backend. V~tomto ohledu je toto řešení nejvíce flexibilní. Předchozí příklad eventuální změny implementace databáze by byl mnohem jednodušší, pouze by se změnil zdroj dat v~implementaci backendu. Tento přístup by byl ale zásadně náročnější na komplexnost a časovou náročnost implementace. 
\end{itemize}

Vhodné řešení pro implementaci této aplikace bude pravděpodobně ležet někdy mezi těmito dvěma extrémy. Vzhledem k~tomu, že zadání aplikace vyžaduje implementaci aplikace pouze pro platformu iOS, nabízí se lákavé řešení implementovat pouze tuto nativní aplikaci a veškerý zbytek infrastruktury opravdu delegovat na nějaký BaaS. Z~osobních preferencí a kvůli solidní konkurenceschopnosti řešení a jeho obstání jako práce z~dílny oboru softwarového inženýrství se ale toto řešení zdá jako poněkud méně šťastné. Jako jedním z~vedlejších cílů této práce byla zvolena široká možnost budoucích rozšíření této aplikace. Potenciál, kam dále by se implementace dala rozšířit, je široký – aplikace by mohla být rozšířena o~již zmíněnou Android aplikaci, ale třeba také o~webovou aplikaci, nebo o nějakou administrátorskou variantu. Zmíněné řešení s~delegováním celé infrastruktury mimo nativní aplikaci na BaaS by jednoduché řešení těchto rozšíření moc neumožňovala, protože každé další napojení by muselo být implementováno úplně odděleně a muselo by se nějakým způsobem na BaaS napojit, čímž by navíc vznikla silná závislost na poskytovateli BaaS služby.

Zajímavou technologií, která byla zmíněna v~sekci \ref{crossplatform-multiplatform} je \emph{Kotlin Multiplatform} \cite{kotlin-multiplatform}. Z~osobního pohledu tato technologie činí ideální kompromis mezi maximálním využití funkcí jednotlivých platforem a snahy sdílení co největší části kódu. Tato technologie poskytuje nástroje pro sdílení kódu mezi aplikacemi pro různé platformy – iOS, Android, Web a další. Využití této technologie, přestože se realizace bude věnovat pouze implementaci pro platformu iOS, poskytne mnohem jednodušší možnosti pro budoucí rozšíření o~další platformy. Ve sdíleném kódu bude potřeba jen přidat nové moduly podle cílových platforem a bude možnost využít co největší část sdíleného kódu, který je společný pro všechny platformy. Použití této technologie je také osobní preferencí kvůli sympatiím s~jejím používáním, lehkou znalostí práce s~ní a možnost se s~touto technologií více seznámit napřímo.

Rozhodnutí využít technologie \emph{Kotlin Multiplatform} ovlivní i~rozhodnutí při volbě technologie pro implementaci backendu. Pro jazyk \emph{Kotlin} totiž existuje knihovna \emph{Ktor} \cite{ktor}, která poskytuje silné nástroje pro serverovou komunikaci. Poskytuje nástroje jak pro implementaci backendu, tak pro implementaci klienta a jejich vzájemnou komunikaci. Přestože tato knihovna není tak rozšířená a známá jako například \emph{Spring Boot} nebo \emph{Django}, které byly zmíněny v~sekci \ref{backend}, tak se ale hodí pro přehlednost a jednotu zdrojového kódu platformy, kde by celý backend i~multiplatformní část mohla být napsána v~jednom jazyce, v~jednom projektu, v~jednom společném repozitáři.

Rozhodnutí využít nástroje spojené s~multiplatformním vývojem v~\emph{Kotlinu} by se daly využít i~dále. Ve světě vývoje Android aplikací je poslední dobou častou volbou implementace uživatelského rozhraní pomocí knihovny \emph{Jetpack Compose} \cite{compose-ui}. Společnost \emph{JetBrains}, tvůrce technologie \emph{Kotlin Multiplatform}, vyvíjí technologii \emph{Compose Multiplatform} \cite{compose-multiplatform}, která dále umožňuje dokonce implementaci uživatelského rozhraní ve sdíleném kódu, které si narozdíl od cross-platformních řešení stále drží výhody multi-platformního přístupu, tedy lepšího využití možností dané platformy. Tato technologie je ale ve fázi \emph{Alpha} a není tolik pokročilá. Z~toho důvodu zůstala volba pro implementaci uživatelského rozhraní na nativním moderním přístupu s~pomocí knihovny \emph{SwiftUI} \cite{swiftui}.

Zatím tedy padly následující volby – nativní uživatelské rozhraní s~pomocí jazyka \emph{Swift} a knihovny \emph{SwiftUI}, multi-platformní implementace sdíleného kódu s~pomocí technologie \emph{Kotlin Multiplatform}, backend s~pomocí jazyka \emph{Kotlin} a knihovny \emph{Ktor}. Nyní ještě zbývá volba pro implementaci databáze.

V~tomto směru už bylo rozhodnuto, že využití externího DBaaS poskytovatele bude na místě. Toto řešení zjednoduší implementaci aplikace o~technologii vlastní databáze, ale stále si udrží možnost eventuální volby, kdy by se poskytovatel databáze mohl změnit. Jelikož bude mít platforma vlastní backend, nebude problém technologii databáze změnit z~externího poskytovatele například na technologii \emph{PostgreSQL}, protože už to bude jen otázka změny zdroje a migrace dat, zatímco implementace byznysové logiky a dalšího, která bude na backendu, zůstane stejná.

Jako dobrá volba externího poskytovatele BaaS se jeví \emph{Firebase} \cite{firebase} od firmy \emph{Google}. Jedná se o~rozšířenou technologii používanou mnoha aplikacemi. Tato technologie zároveň umožní obsluhu jiných funkcionalit aplikace, jako například autentizace. Není potřeba mezi implementací databáze a autentizace tvořit nějakou závislost, přijde-li tedy v~budoucnu požadavek poskytovatele autentizace nebo databáze změnit, může tak být učiněno nezávisle na sobě, tedy klidně pouze jedno z~toho, nebo obojí.

Tato volba architektury platformy dodržuje požadavek na velkou míru flexibility a rozšiřovatelnosti, zároveň se snaží zjednodušit náročnost implementace a eliminovat duplikace kódu, bude-li aplikace rozšířena o~další platformy. Schéma této architektury lze nahlédnout na obrázku \ref{fig:architecture}. V~tomto obrázku jsou také znázorněny možnosti případného rozšíření aplikace. Z~této celkové architektury platformy lze poté navrhovat architekturu jednotlivých částí.

\begin{figure}[h]
	\centering
	\includegraphics[width=\textwidth]{architecture.png}
	\caption{Architektura platformy}
	\label{fig:architecture}
\end{figure}

%---------------------------------------------------------------
\subsection{Architektura nativní aplikace}
%---------------------------------------------------------------

Architektura nativní aplikace bude vycházet z~veřejné šablony pro projekt mobilní aplikace pro platformy iOS a Android s~použitím technologie \emph{Kotlin Multiplatform}, nazvané \emph{Matee KMP DevStack} \cite{matee-devstack}. Důvody k~použití této platformy jsou osobní preference a znalost tohoto projektu. Alternativou by bylo například použití generátoru šablon projektů \emph{Kotlin Multiplatform Wizard} \cite{kmp-wizard}, ale \emph{DevStack} společnosti \emph{Matee} obsahuje velkou řadu užitečných nástrojů pro uživatelské rozhraní a další užitečné komponenty.

\emph{Matee DevStack} používá architekturu \emph{Clean Architecture} popsanou v~sekci \ref{app-architecture}. Technologie \emph{Kotlin Multiplatform} do této architektury zapadá tak, že nativní aplikace implementuje celou prezentační vrstvu, a multi-platformní část implementuje co největší možnou část doménové a datové vrstvy. Multi-platformní část je do nativní aplikace přidána jako knihovna, která je závislostí doménové vrstvy. Nativní aplikace má tedy přístup hlavně k~doménovým modelům a \emph{Use Cases}.

Nativní aplikace využívá \emph{modularizaci}, která představuje další vlastnost architektury, ve které je aplikace rozdělená do několika modulů, v~tomto případě podle funkcionalit. Prezentační vrstva aplikace tedy bude disponovat moduly podle funkcí jako \emph{Onboarding} (přihlášení/registrace), \emph{Timer} (časovač, historie záznamů), \emph{Integrations} (nastavení integrací), \emph{Profile} (profil uživatele) a poté například modulem \emph{UIToolkit}, který definuje společné komponenty pro prezentační vrstvu.

Doménová vrstva nativní aplikace bude obsahovat moduly \emph{SharedDomain}, což je modul, který definuje společné atributy domény aplikace, ale zde bude mít hlavní funkci jako poskytovatel sdílené knihovny, která bude výstupem multi-platformní části. Dále bude vrstva obsahovat modul \emph{Utilities}, který definuje užitečné nástroje pro práci s~doménovými či sdílenými funkcemi a objekty.

Datová vrstva obvykle obsahuje implementace repozitářů, v~modulech podle funkcionalit, a \emph{Providers}, také v~modulech podle funkcionalit. Jelikož většinu této vrstvy bude implementovat sdílený kód, bude tato vrstva obsahovat jen ty části, které sdílený kód implementovat nemůže. Což jsou obvykle platformní záležitosti, jako obsluha notifikací, GPS polohy, a další. Mohou to být ale i~další implementace, pro které zkrátka multi-platformní část nemá podporu, jako třeba různé funkce poskytovatele \emph{Firebase}. Tyto části kódu budou fungovat tak, že v~multiplatformní části kódu se budou nacházet rozhraní pro funkcionality, které musí implementovat každá platforma sama. V~multiplatformní části se bude pracovat s~těmito rozhraními, a nativní aplikace bude zodpovědná za to, že rozhraním poskytne implementace. Nativní aplikace tedy bude implementovat pouze tu část, kterou sdílená část implementovat nemůže, a výstupy těchto implementací bude vracet do sdíleného kódu.

Zmíněné moduly jsou v~nativní aplikaci ve skutečnosti balíčky typu \emph{Swift Package}, které jsou spravovány nástrojem \emph{Swift Package Manager} \cite{spm}. Nativní aplikace dále obsahuje aplikační vrstvu, která obsahuje základní části aplikace, jako \emph{AppDelegate}, \emph{Info.plist}, ikonu aplikace, nebo například rodičovský \emph{Flow Controller} aplikace, obvykle \emph{AppFlowController}, ze kterého se vytváří všechny další navigační toky. Architekturu nativní aplikace lze nahlédnout na obrázku \ref{fig:app-architecture}.

\begin{figure}[h]
	\centering
	\includegraphics[width=\textwidth]{app-architecture.png}
	\caption{Architektura nativní aplikace}
	\label{fig:app-architecture}
\end{figure}







































%---------------------------------------------------------------
\chapter{Realizace}
%---------------------------------------------------------------

Tato kapitola se zabývá samotnou realizací platformy pro měření odpracovaného času. Popisuje nástroje použité během vývoje, strukturu zdrojového kódu a~nakonec implementaci jednotlivých navržených funkcionalit aplikace.

%---------------------------------------------------------------
\section{Nástroje pro vývoj}\label{development-tools}
%---------------------------------------------------------------

Jak bylo navrženo v~sekci \ref{dev-tools}, pro nativní aplikaci bylo použito vývojové prostředí \emph{Xcode}. Verze tohoto IDE, na kterém byla implementace vyvíjena, byla 15.2. S~touto verzí IDE se pojí verze programovacího jazyku \emph{Swift} 5.9.2. Aplikace byla vyvíjena na operačním systému \emph{macOS Sonoma} (14.4). Nativní aplikace bude definovat minimální iOS verzi cílové platformy na verzi 17.0, jelikož tato verze umožňuje použití nejnovějších nástrojů platformy iOS. Podle dat z~února 2024 už tuto nejnovější verzi operačního systému používalo 66\% ze všech aktivních uživatelů mobilních telefonů iPhone. Mezi zařízeními představených v~předchozích čtyřech letech je to dokonce 76\% \cite{ios-17-adoption}.

Vývojové prostředí pro multi-platformní část a~backend také dodržuje návrh ze sekce \ref{dev-tools}, bylo tedy použito \emph{IntelliJ IDEA}, verze 2024.1. Pro jazyk \emph{Kotlin} byla použita verze 1.9.10. 
 a~
\emph{Git} repozitář byl zálohován nástrojem \emph{GitHub} \cite{github} v~repozitáři \cite{trackee-app-github}. Během vývoje nebyly použity žádné větve ani další nástroje pro týmovou spolupráci v~nástroji \emph{Git}, jelikož nikdo další na projektu nespolupracoval. Veškerý vývoj tedy probíhal přímo v~hlavní větvi \emph{main}.

%---------------------------------------------------------------
\subsection{Firebase}
%---------------------------------------------------------------

V~nástroji \emph{Firebase}, který byl vybrán v~návrhu v~sekci \ref{platform-architecture}, byl vytvořen projekt pro aplikaci Trackee. Byly vytvořeny tři prostředí – Alpha, Beta a~Produkce. V~projektu zatím byly nastaveny funkce pro autentizaci a~\emph{Firestore} databázi. 

\emph{Firebase} autentizace podporuje různé formy autentizace, jako e-mail s~heslem, ale třeba také přihlášení přes Google, přes Apple, přes Facebook a~podobně. Rozšíření možností přihlášení by mohlo být podnětem pro budoucí vývoj aplikace, v~této fázi je zatím implementováno pouze přihlášení přeš e-mail a~heslo. 

Při tvorbě \emph{Firestore} databáze \emph{Firebase} nabízí výběr, kde se má instance databáze nacházet (západní Evropa nebo Severní Amerika). Byla vybrána nejbližší možnost, tedy západní Evropa.

%---------------------------------------------------------------
\subsection{Nasazení backendu}\label{backend-deployment}
%---------------------------------------------------------------

Během vývoje byla aplikace backendu spouštěna a~testována na lokálním stroji. V~místních sítích se tedy stačilo připojovat pomocí IP adresy počítače, na kterém aplikace backendu běžela, přes protokol HTTP, například: \texttt{http://192.168.88.70:8080}

Pro umožnění používání aplikace kdekoli, kdykoli a~bez potřeby na lokálním stroji neustále ručně spouštět aplikaci backendu, bylo vhodné vybrat nějaké řešení, které by umožnilo nasazení aplikace backendu. K~tomu existuje řada externích poskytovatelů. Pro potřeby tohoto projektu byl zvolen nástroj \emph{Railway} \cite{railway}, jelikož zdarma nabízí počáteční kredit v~hodnotě pěti amerických dolarů, což by mělo na nějakou dobu vystačit pro potřeby vývoje aplikace. Tomuto nástroji pouze stačí propojení s~\emph{GitHub} repozitářem, nastavení skriptu pro spuštění a~je hotovo. Nástroj poté při každé aktualizaci repozitáře sestaví projekt, spustí backend a~nasadí ho na URL adrese \texttt{https://trackee-app-production.up.railway.app}. 

\emph{Railway} také podporuje automatické uspávání aplikace, které může snížit využívání kreditu v~době, kdy aplikace není využívána \cite{railway-app-sleeping}. Pokud se tedy aplikace nějakou dobu nepoužívá, je potřeba počítat s~několika sekundovou prodlevou při prvním požadavku.

Jednou nevýhodou nástroje \emph{Railway} ale je, že v~bezplatné variantě nabízí umístění stroje, na kterém běží instance, pouze ve státě Oregon ve Spojených státech amerických. Jelikož je to v~podstatě na druhé straně Země, vzniká tím poměrně znatelná prodleva mezi klientem a~serverem. A~v~kombinaci s~tím, že při nastavení nástroje \emph{Firebase} bylo zvoleno umístění instance v~Evropě (a toto umístění není možné po nastavení nástroje změnit), vznikají tím hlavně prodlevy při komunikaci s~databází, protože nyní musí backend každý požadavek na čtení nebo zápis dat posílat zhruba 8 tisíc kilometrů daleko. Rozdíl mezi používáním instance backendu běžící pomocí nástroje \emph{Railway} a~používáním lokální instance je znatelný – například přihlášení a~načtení hlavní obrazovky včetně plné počáteční stránky historie záznamů (10 záznamů) trvá 5-7 sekund, zatímco při použití lokální instance toto trvá 2-3 sekundy.

%---------------------------------------------------------------
\subsection{Nasazení aplikace a~Testflight}\label{testflight}
%---------------------------------------------------------------

Vývojové prostředí \emph{Xcode} umožňuje nahrání vyvíjené aplikace do připojeného mobilního zařízení. Dané zařízení ale musí být fyzicky připojeno k~počítači, na kterém IDE běží, musí být ve vývojářském režimu a~musí být napojené na vývojářský účet, pod který aplikace spadá. Jedná se tedy o~poměrně komplikované řešení, je-li účelem instalace aplikace mezi další uživatele, například mezi testery.

Nástrojem, jak řešit nasazení aplikace mezi širší počet uživatelů, ale nikoli zatím jako finální aplikaci mířenou na reálné uživatele, je \emph{Testflight} \cite{testflight}. Tento nástroj je určen pro beta testování vyvíjených aplikací. Umožňuje nahrávání archivovaných (sestavených) aplikací pomocí nástroje \emph{App Store Connect} \cite{app-store-connect} a~jejich následnou distribuci mezi vybrané uživatele, kteří si tuto aplikaci nainstalují. Jsou dva způsoby, jak lze aplikace přes \emph{Testflight} distribuovat:
\begin{itemize}
\item\textbf{Interní testování} – vývojář pozve jednotlivé uživatele pomocí jejich e-mailové adresy k~testování aplikace a~ti si ji mohou stáhnout. Tito vývojáři musí mít účet \emph{Apple ID} a~musí být přidání mezi interní vývojáře aplikace.
\item\textbf{Veřejné testování} – vývojář může do testování přizvat kohokoli, dokonce stačí jenom takzvaný \emph{public link}, přes který si uživatelé mohou aplikaci nainstalovat. Aby aplikace mohla být testována pomocí veřejného testování, musí být schválena firmou \emph{Apple} v~rámci procesu \emph{App Review}, který aplikace podstupují i~tehdy, pokud chtějí zamířit mezi reálné uživatele do obchodu \emph{App Store}.
\end{itemize}

Aplikace Trackee umožňuje oba typy testování, již prošla schvalovacím procesem \emph{App Review} a~může si ji tak nainstalovat kdokoli, kdo bude pozvaný přes e-mailovou adresu, nebo pomocí \emph{public linku}: \texttt{https://testflight.apple.com/join/cTRdRkBc}

%---------------------------------------------------------------
\section{Architektura a~struktura projektu}\label{project-structure}
%---------------------------------------------------------------

Jako šablona celého projektu sloužil \emph{DevStack} společnosti \emph{Matee} \cite{matee-devstack}, ze kterého byla odstraněna většina implementovaných funkcionalit a~jejich balíčků, kromě těch, které se hodilo znovu využít (onboarding, profil, \dots). Ze šablony byly také ponechány nástroje na síťovou komunikaci (síťový klient pomocí knihovny \emph{Ktor} \cite{ktor}), interoperabilitu a~další. V~šabloně také byly ponechány některé nástroje, které zatím projektem nejsou využívány, ale v~budoucnu při dalším rozvoji aplikace by se mohly hodit, jako nástroje pro práci s~lokálním úložištěm \emph{SQLDelight}, \emph{UserDefaults} nebo \emph{Keychain}, nebo \emph{Providers} pro obsluhu GPS, obsluhu notifikací, a~další. Jelikož Aplikace Trackee neimplementuje Android klienta, je celý jeho modul ze šablony odstraněn.

Aplikace implementuje tři prostředí – Alpha, Beta a~Produkce. Zvykem bývá, že Alpha má vlastní instanci databáze a~serveru, která slouží pouze pro testování a~ladění, například při nasazování nových verzí backendu, a~podobně. Beta na tom je pak obvykle podobně, ale buď duplikuje data z~produkčního prostředí (ale přímo ho nemůže ovlivňovat), nebo je na produkční prostředí přímo napojena, ale nabízí přidané ladící možnosti. Produkční prostředí pak poskytuje databázi a~server, které jsou využívány reálnými uživateli, kde je priorita orientována spíše na rychlost, než na možnosti ladění. Během vývoje aplikace Trackee nebylo potřeba využívat více instancí databáze nebo serveru, protože na projektu nepracuje více lidí, nevznikají konflikty a~není zatím využíván reálnými uživateli. Prostředí se tedy liší jenom v~drobnostech, jako například v~tom, že Alpha a~Beta nabízí rozšířené možnosti zachycování a~přepisování požadavků na backend, nebo poskytuje detailnější popisy chybových hlášek, které mohou obsahovat více technických informací. 

Multi-platformní část a~backend používají automatizační sestavovací nástroj \emph{Gradle} \cite{gradle}, což je populární nástroj pro flexibilní konfiguraci a~automatizaci sestavování softwarových projektů. Nativní iOS aplikace používá pro sestavení nástroj \emph{xcodebuild}, který je součástí \emph{Xcode} IDE. Multi-platformní knihovna je do nativní iOS aplikace propagována jako knihovna typu \emph{xcframework}, což je \emph{multiplatformní binární framework} \cite{xcframework}. \emph{Gradle} zkompiluje multi-platformní kód do této \emph{xcframework} reprezentace a~zkopíruje ji do iOS projektu.

Implementace API backendové části aplikace dodržuje návrhové vzory \emph{REST} \cite{rest-api}, jedná se tedy o~\emph{RESTful API}. Pro možnosti rozšiřovatelnosti aplikace nabízí \emph{Swagger OpenAPI} dokumentaci \cite{swagger-open-api} na URL adrese \texttt{<backend-host-url>/openapi}, v~případě nasazení v~\emph{Railway} aplikaci tedy na URL adrese \texttt{https://trackee-app-production.up.railway.app/openapi}. Zdrojem této dokumentace je soubor \texttt{openapi/documentation.yaml} ve složce \texttt{resources} ve zdrojovém kódu aplikace backendu.

Všechny části implementace (nativní aplikace, multi-platformní část a~backend) jsou součástí jednoho repozitáře (\emph{monorepo}), který má následující strukturu:
\begin{itemize}
\item\texttt{backend} – modul obsahující projekt backendu
\item\texttt{build-logic} – společný modul ostatních modulů, obsahující pluginy a~nástroje pro sestavení multi-platformní knihovny
\item\texttt{gradle} - složka pro soubory nástroje \emph{Gradle}
\item\texttt{ios} – složka obsahující projekt nativní iOS aplikace
\item\texttt{other} – nástroje projektu
\item\texttt{shared} – modul obsahující sdílený kód pro technologii \emph{Kotlin Multiplatform}
\item\texttt{twine} – složka obsahující lokalizační soubor
\end{itemize}
Jednotlivé části implementace poté dodržují architekturu navrženou v~sekci \ref{platform-architecture}.

%---------------------------------------------------------------
\subsection{Lokalizace}
%---------------------------------------------------------------

Aplikace byla implementována a~lokalizována pro tři jazyky – čeština, slovenština a~angličtina. Zdroje pro lokalizace všech textů (tlačítka, popisky, instrukce, \dots) jsou uloženy v~souboru \texttt{strings.txt} ve složce \texttt{twine}, který je připraven ve formátu pro nástroj \emph{Twine} \cite{twine}, který tento soubor zpracovává a~tvoří z~něj lokalizační soubory pro cílové platformy (pro iOS jsou to soubory \texttt{Localizable.strings}). Nastavení tohoto nástroje pro tyto jazyky je již v~šabloně \emph{DevStack} připraveno.

%---------------------------------------------------------------
\subsection{Automatické generování kódu}
%---------------------------------------------------------------

Šablona také používá několik pomocných nástrojů pro automatické generování kódu. Jedním takovým nástrojem je \emph{SwiftGen} \cite{swiftgen}, který zde slouží k~několika účelům:
\begin{itemize}
\item Generování struktur pro lokalizaci: \emph{SwiftGen} podle šablon generuje statické struktury z~lokalizačních souborů \texttt{Localizable.strings} pro jednodušší použití v~kódu, které zamezí použití špatných klíčů lokalizací.
\item Generování struktur pro obrázky: \emph{SwiftGen} je v~šabloně nastaven takovým způsobem, aby ze zdrojů \texttt{Images.xcassets} generoval statické struktury, které umožní jednodušší referování obrázků ze zdrojového kódu.
\item Generování struktur pro barvy: Funguje stejným způsobem, jako obrázky.
\end{itemize}
Nástroj \emph{SwiftGen} je v~šabloně přidán jako \emph{plugin} modulu \emph{UIToolkit} v~prezentační vrstvě.

Nativní iOS aplikace také v~šabloně využívá makra \emph{swift-spyable} \cite{swift-spyable} \emph pro generování \emph{Use Case Mocks}, tedy statických náhrad pro \emph{Use Cases} používaných pro testování. Toto je ale používáno jen pro nativní \emph{Use Cases}, které aplikace Trackee neobsahuje. Pro sdílené \emph{Use Cases} je využíváno vlastních rodičovských tříd \texttt{UseCaseResultMock}, \texttt{UseCaseResultNoParamsMock}, atd.

%---------------------------------------------------------------
\subsection{Dependency Injection}
%---------------------------------------------------------------

Pro \emph{Dependency Injection} v~nativní aplikaci je využíváno knihovny \emph{Factory} \cite{factory}, v~multi-platformní části a~na backendu knihovny \emph{Koin} \cite{koin}. V~aplikaci Trackee jsou sice všechny \emph{Use Cases} a~\emph{Repositories} v~multi-platformní části, ale \emph{Dependency Injection} v~prezentační vrstvě nativní aplikace získává implementace sdílených \emph{Use Cases} knihovnou \emph{Factory}, aby bylo dosazování sjednocené s~případnými nativními \emph{Use Cases}. \emph{Dependency Injection} v~nativní aplikaci tedy funguje tak, že v~modulu \emph{DependencyInjection} v~aplikační vrstvě se registrují implementace všech závislostí, včetně sdílených \emph{Use Cases}, které ale poskytuje knihovna \emph{Koin}.

%---------------------------------------------------------------
\section{Implementace jednotlivých funkcionalit}
%---------------------------------------------------------------

Tato sekce se věnuje implementaci jednotlivých funkcionalit navržených v~sekci \ref{features}. Implementaci každé funkcionality rozebírá z~hlediska její realizace v~backendové části, v~multi-platformní části a~v~nativní aplikaci. 

U~jednotlivých funkcionalit mohou být uvedeny ukázky výpisů kódu, většinou tak, aby pro každou funkcionalitu byl ukázán příklad zdrojového kódu z~jiné části aplikace. Cílem je poskytnout pár příkladů pro představu toho, jakým stylem je zdrojový kód implementován, a~ne zbytečně ukazovat výpisy kódů pro věci, které jsou velmi podobné implementaci, která již byla ukázána v~jiné části. Pro důkladnější studium kódu samozřejmě slouží zdrojový kód sám, který je součástí přiloženého média této práce. Text této práce slouží mimo jiné jako jeho dokumentace, která v~něm má usnadnit orientaci.

%---------------------------------------------------------------
\subsection{Přihlášení a~registrace}\label{onboarding-impl}
%---------------------------------------------------------------

Přihlášení a~registrace je základní funkcionalitou, kterou aplikace potřebuje, aby mohla držet informace o~konkrétním uživateli. V~návrhu platformy (obrázek \ref{fig:architecture}) byla autentizace navržena tak, že bude probíhat přes backend. Ale vzhledem k~tomu, že \emph{Firebase} autentizace nabízí více možností autentizace, jako přihlášení přes Apple, Google, a~podobně, tak byl v~implementaci zvolen přístup, že autentizace bude probíhat na straně klienta, a~ne backendu, protože jí celou obstará \emph{Firebase} autentizace. Bylo by sice možné provádět autentizace přes e-mail a~heslo tak, že by klient poslal danou kombinaci e-mailu a~hesla na backend, který by uživatele přes \emph{Firebase} přihlásil, klientovi vrátil jeho \emph{access token}\footnote{\emph{Access token} je v~autentizační terminologii řetězec znaků, se kterým klient posílá požadavky na server, aby prokázal svou identitu. \cite{access-token}}, který by ho poté dále používal při následujících požadavcích. V~případě rozšíření o~další formy přihlášení, jako je přihlášení přes Apple, by v~tomto přístupu musel \emph{Firebase} provést autentizaci u~klienta, získat mezistavový \emph{token} pro daného poskytovatele přihlášení, poslat ho na backend, který by ho poté ověřil s~autentizací \emph{Firebase}, získal \emph{access token}, a~poslal ho klientovi. Vzhledem k~tomu, že takto by se autentizační API muselo volat jak u~klienta, tak na backendu, byl zvolen přístup, že celý tento proces se bude dít u~klienta pomocí knihovny pro \emph{Firebase} autentizaci, a~na backend se bude ve všech možnostech přihlášení posílat jen \emph{access token}. V~případě, že by v~budoucnu padlo rozhodnutí tuto metodiku změnit a~implementovat autentizační procesy pouze na backendu, musela by se tedy implementovat metodika popsaná výše.

%---------------------------------------------------------------
\subsubsection{Backend}
%---------------------------------------------------------------

Jak již bylo popsáno, backendová část nebude implementovat celý autentizační proces, ale jen bude ověřovat platnost \emph{access tokenu}. To bylo naimplementováno pomocí kofigurace bezpečnosti knihovny \emph{Ktor}. Pro všechny \emph{Routes}, tedy \emph{Endpoints}, které jsou definovány vnořeně ve funkci \texttt{authenticate}, se bude ověřovat, zda požadavek ve hlavičce obsahuje \emph{Bearer token} (\emph{access token}), ověří se přes \emph{Firebase} autentizaci pomocí funkce \texttt{verifyIdTokenAsync}, poté se získá uživatel pomocí \texttt{uid} a~dále budou moct \emph{Routes} pracovat přímo s~informacemi o~přihlášeném uživateli. Průběh verifikace \emph{access tokenu} lze vidět v~ukázce kódu \ref{code:access-token-verification}. V~ukázce kódu \ref{code:ktor-security-config} lze poté vidět, jak bude backend získávat objekt úspěšně ověřeného uživatele. Pokud je uživatel úspěšně ověřen, ale nemá záznam v~databázi, bude mu tento záznam automaticky vytvořen – toto se bude dít při registraci uživatele.

\begin{listing}
\caption{Průběh verifikace \emph{access tokenu}}\label{code:access-token-verification}
\begin{minted}{Kotlin}
class FirebaseAuthProvider(config: Config) : AuthenticationProvider(config) {

    // ...
    
    override suspend fun onAuthenticate(context: AuthenticationContext) {
        val authHeader = context.call.request.parseAuthorizationHeader() 
            as? HttpAuthHeader.Single ?: throw AuthException.Unauthorized

        val token = try {
            firebase.verifyIdTokenAsync(authHeader.blob).await()
        } catch (e: FirebaseAuthException) {
            throw AuthException.InvalidToken(e)
        }

        val user = firebase.getUserAsync(token.uid).await()

        context.principal(FirebasePrincipal(user))

        val principal = context.mapPrincipal(user)
        if (principal != null) context.principal(principal)
    }
    
    // ...
}

fun AuthenticationConfig.firebase(
    name: String? = null,
    configure: FirebaseAuthProvider.Config.() -> Unit
) {
    val provider = FirebaseAuthProvider.Config(name).apply(configure).build()
    register(provider)
}
\end{minted}
\end{listing}

\begin{listing}
\caption{Konfigurace bezpečnosti knihovny \emph{Ktor}}\label{code:ktor-security-config}
\begin{minted}{Kotlin}
fun Application.configureSecurity() {
    
    // ...
    
    authentication {
        firebase {
            firebase = firebaseAuth
            mapPrincipal { userRecord ->
                try {
                    val user = userRepository.readUserByUid(userRecord.uid)
                    UserPrincipal(user)
                } catch(e: UserException.UserNotFound) {
                    val user = userRepository.createUser(userRecord.uid)
                    UserPrincipal(user)
                }
            }
        }
    }
}
\end{minted}
\end{listing}

Žádné \emph{Routes} pro funkcionalitu přihlášení a~registrace být definovány nemusí, jelikož tento proces bude probíhat přes \emph{Firebase} autentizaci. V~ukázce kódu \ref{code:ktor-security-config} je ale vidět, že po úspěšném ověření \emph{access tokenu} se aplikace pokusí získat objekt uživatele z~databáze, případně nového uživatele vytvořit. Tyto funkce poskytuje \texttt{UserRepository}, která bude dále definovat řadu dalších funkcích souvisejících v~práci s~daty, které se týkají uživatele. Definice rozhraní tohoto repozitáře je součástí doménové vrstvy, implementaci poté definuje datová vrstva. Implementace repozitáře poté bude potřebovat pro přístup k~datům \emph{Source}, v~tomto případě \texttt{UserSource}. Implementace tohoto \emph{Source} už poté přímo poskytuje komunikaci s~\emph{Firestore} databází.

V~ukázce kódu \ref{code:be-create-user} je vidět implementace toho, jak se v~databázi vytváří nový uživatel. Lze nahlédnout, že tato metoda vrací objekt úspěšně vytvořeného uživatele typu \texttt{FirestoreUser}. Jako parametr funkce ale bere typ \texttt{User}. \texttt{User} je totiž doménový objekt reprezentující uživatele, v~databázi se používá databázový model nazvaný \texttt{FirestoreUser}, a~mezi těmito objekty existují převáděcí funkce (\texttt{toFirestore()}, \texttt{toDomain()}). Další backendovou reprezentací uživatele je pak ještě \texttt{UserDto}, která se zase používá při komunikaci s~klientskou aplikací. Pro tuto reprezentaci také existují převáděcí funkce (\texttt{toDto()}, \texttt{toDomain()}). V~backendové části je potřeba používat všechny tyto tři typy reprezentací pro jeden typ objektu, protože pro interní práci s~objekty slouží doménový objekt (tedy například \texttt{User}) a~pro externí práci s~objekty slouží DTO objekty pro jednotlivé části infrastruktury (tedy například \texttt{FirestoreUser} a~\texttt{UserDto}).

\begin{listing}
\caption{Vytvoření nového uživatele v~\texttt{UserSourceImpl}}\label{code:be-create-user}
\begin{minted}{Kotlin}
internal class UserSourceImpl : UserSource {

    private val db = GoogleFirestoreClient.getFirestore()
    
    // ...
    
    override suspend fun createUser(user: User): FirestoreUser {
        db
            .collection(SourceConstants.Firestore.Collection.USERS)
            .document(user.uid)
            .set(user.toFirestore())
            .await()

        return db
            .collection(SourceConstants.Firestore.Collection.USERS)
            .document(user.uid)
            .get()
            .await()
            .toObject(FirestoreUser::class.java) 
                ?: throw UserException.UserNotFound(user.uid)
    }
    
    // ...
}
\end{minted}
\end{listing}

%---------------------------------------------------------------
\subsubsection{Multi-platformní část}
%---------------------------------------------------------------

Pro potřeby přihlášení a~registrace poskytuje multi-platformní knihovna následující \emph{Use Cases}:
\begin{itemize}
\item\texttt{LoginWithCredentialsUseCase} – Poskytuje přihlášení pomocí e-mailu a~hesla. Tuto funkcionalitu ale implementuje \emph{Firebase} autentizace na straně nativní aplikace, multi-platformní kód tedy musí provolat sdílený \texttt{AuthProvider}, který implementuje nativní aplikace pomocí \emph{Firebase} knihovny.
\item\texttt{RegisterUseCase} – Registruje uživatele přes \emph{Firebase} autentizaci, stejným způsobem jako přihlášení.
\item\texttt{LogoutUseCase} – Odhlásí uživatele pomocí \emph{Firebase} autentizace. Tento \emph{Use Case} sice bude potřeba až v~profilu uživatele, ale je součástí \emph{auth} modulu multi-platformní části, je zde tedy zmíněn.
\item\texttt{IsLoggedInUseCase} – Zjistí, zda je uživatel do aplikace přihlášený, nebo ne. Tento \emph{Use Case} se zavolá při čistém spuštění aplikace, aby zjistila, zda má ukázat přihlašovací obrazovku, nebo uživateli rovnou ukázat časovač. Funguje tak, že se zeptá \emph{Firebase} autentizace, zda má uložený \emph{access token}.
\end{itemize}

API \emph{Firebase} knihovny je potřeba provolávat pomocí sdíleného \texttt{AuthProvider} zpět do nativní aplikace, protože \emph{Firebase} oficiálně nenabízí SDK pro \emph{Kotlin Multiplatform}, ale pouze SDK pro nativní aplikace (\emph{firebase-ios-sdk} \cite{firebase-ios-sdk} a~\emph{firebase-android-sdk} \cite{firebase-android-sdk}). Existuje sice \emph{firebase-kotlin-sdk} \cite{firebase-kotlin-sdk}, která podporuje \emph{Kotlin Multiplatform}, ale nejedná se o~oficiální SDK a~nemá 100\% pokrytí celého \emph{Firebase} API.

Žádné z~těchto \emph{Use Cases} neprovolávají \emph{Endpoints} na backendu, ovlivňují ale to, jak budou ostatní požadavky vypadat, konkrétně obsah parametru pro autorizaci v~hlavičce požadavků.

%---------------------------------------------------------------
\subsubsection{Nativní aplikace}
%---------------------------------------------------------------

Implementace přihlášení a~registrace dodržuje vzhledový návrh ze sekce \ref{feature-onboarding}. Realizace viditelná na skutečném zařízení lze vidět na obrázku \ref{fig:onboarding-impl}, kde lze i~vidět jak se rozhraní přizpůsobí vysunuté klávesnici. Na obrázku \ref{fig:login-error-impl} je také vidět, jakým způsobem ze zobrazí \emph{Toast}\footnote{\emph{Toast} je UI komponentou z~\emph{DevStack} šablony sloužící pro dočasné zobrazovaní informací nebo chybových hlášek na obrazovce.} s~chybovou hláškou.

\begin{figure}[h]
    \centering
    \begin{subfigure}[b]{0.4\textwidth}
		\centering
		\includegraphics[width=6cm]{login-error-impl.png}
		\caption{Chyba při přihlášení}
		\label{fig:login-error-impl}
	\end{subfigure}
	\hspace{2cm}
	\begin{subfigure}[b]{0.4\textwidth}
		\centering
		\includegraphics[width=6cm]{register-impl.png}
		\caption{Registrace}
	\end{subfigure}
	\caption{Realizace přihlášení a~registrace}
	\label{fig:onboarding-impl}
\end{figure}

%---------------------------------------------------------------
\subsection{Časovač}
%---------------------------------------------------------------

Tato funkcionalita pokrývá veškeré operace s~ovládáním časovače, získáváním historie časových záznamů a~vytváření nových záznamů. Jedná se o~hlavní funkci aplikace.

%---------------------------------------------------------------
\subsubsection{Backend}
%---------------------------------------------------------------

Backendová část platformy bude muset spravovat veškerá data, která se týkají časovače a~časových záznamů. Backend je ale rozdělený podle funkcionalit jiným způsobem, než nativní aplikace a~multi-platformní část, které jsou rozděleny podle funkcionalit z~hlediska uživatele. Moduly backendu jsou rozděleny podle toho, čeho se data v~modulu týkají, tedy například klienti, projekty, integrace a~uživatel. Jelikož pro data časovače budou potřeba data uživatele, u~kterého se ukládá aktuální nastavení časovače a~historie časových záznamů, ale také data projektů a~klientů, bude implementace funkcionality časovače zasahovat do více modulů. Největší část funkcionality časovače bude obsluhovat modul uživatele. 

Obrazovka časovače bude potřebovat následující data – souhrn odpracovaných hodin v~aktuální den a~týden, historii časových záznamů, aktuální nastavení časovače a~přehled projektů, bude-li si uživatel chtít vybrat projekt, který časovači přidělí.

V~první řadě je tedy potřeba implementovat získávání historie časových záznamů, protože to bude potřeba i~pro výpočet odpracovaných hodin v~aktuálním dnu a~týdnu. V~návrhu v~sekci \ref{feature-timer} bylo popsáno, že načítání historie časových záznamů by mělo podporovat stránkování, jelikož může teoreticky obsahovat velké množství záznamů, a~pokud by se mělo velké množství načítat celé najednou, mohlo by to trvat dlouho a~mohlo by se jednat o~velké množství dat, která by ale uživatel nejspíš ani všechna vůbec nepotřeboval. \texttt{UserRepository} tedy definuje mimo jiné funkce pro získávání záznamů, které lze nahlédnout v~ukázce kódu \ref{code:be-read-entries-interface}. Tyto funkce potřebují parametr \texttt{uid} identifikující uživatele, pro kterého mají být záznamy čteny, a~dále přijímají nepovinné parametry pro specifikaci toho, od kdy a~do kdy mají být záznamy čteny, a~kolik maximálně záznamů se ve stránce má nacházet. Pokud bude mít některý z~těchto nepovinných parametrů hodnotu \texttt{null}, nebude žádné omezení na záznamy klást. Rozdíl mezi funkcemi, které vrací stránky objektu \texttt{TimerEntry} a~stránky objektu \texttt{TimerEntryPreview} je ten, že objekt \texttt{TimerEntryPreview} obsahuje navíc celé zdrojové objekty klienta a~projektu, který k~záznamu patří, zatímco \texttt{TimerEntry} obsahuje pouze jejich identifikátory. V~obrazovce pro zobrazení historie záznamů budou tyto zdrojové objekty potřeba, protože při jejich vizualizaci (při jejich \emph{preview}) bude potřeba zobrazit jméno klienta a~projektu.

\begin{listing}
\caption{Funkce pro získávání časových záznamů v~\texttt{UserRepository}}\label{code:be-read-entries-interface}
\begin{minted}{Kotlin}
interface UserRepository {

    // ...
    
    suspend fun readEntries(
        uid: String,
        startAfter: Instant?,
        limit: Int?,
        endAt: Instant?
    ): Page<TimerEntry>
    
    // ...
    
    suspend fun readEntryPreviews(
        uid: String,
        startAfter: Instant?,
        limit: Int?,
        endAt: Instant?
    ): Page<TimerEntryPreview>
    
    // ...
}
\end{minted}
\end{listing}

V~ukázce kódu \ref{code:be-read-entries-source} lze poté vidět, jakým způsobem aplikace získává záznamy přímo z~\emph{Firestore} databáze. Omezení ve smyslu od kdy do kdy záznamy číst a~kolik maximálně jich přečíst se implementuje pomocí funkcí \texttt{startAfter(start)}, \texttt{endAt(end)} a~\texttt{limit(limit)}, které přímo nabízí \emph{Firestore} API. \texttt{UserRepositoryImpl} používá tuto \emph{Source} funkci i~pro načtení \emph{preview} objektů, objekty klienta a~projektu si pak pomocí identifikátoru načte sama a~připojí je.

\begin{listing}
\caption{Funkce pro získávání časových záznamů v~\texttt{UserSourceImpl}}\label{code:be-read-entries-source}
\begin{minted}{Kotlin}
internal class UserSourceImpl : UserSource {

    // ...
    
    override suspend fun readEntries(
        uid: String,
        startAfter: Instant?,
        limit: Int?,
        endAt: Instant?
    ): Page<FirestoreTimerEntry> {
        val entriesCollection = db
            .collection(SourceConstants.Firestore.Collection.ENTRIES)
            .document(uid)
            .collection(SourceConstants.Firestore.Collection.ENTRIES)
            .orderBy(
                SourceConstants.Firestore.FieldName.STARTED_AT, 
                Query.Direction.DESCENDING
            )

        val snapshot = entriesCollection
            .startAfter(startAfter?.toTimestamp() ?: Timestamp.now())
            .endAt(endAt?.toTimestamp() ?: Timestamp.MIN_VALUE)
            .limit(limit ?: Int.MAX_VALUE)
            .get()
            .await()

        val data = snapshot.documents.map { 
            it.toObject(FirestoreTimerEntry::class.java) 
        }

        val remainingCount = entriesCollection
            .startAfter(data.lastOrNull()?.startedAt ?: Timestamp.MIN_VALUE)
            .count()
            .get()
            .await()
            .count

        return Page(
            data = data,
            isLast = remainingCount == 0.toLong()
        )
    }
    
    // ...
}
\end{minted}
\end{listing}

Při načítání historie záznamů na obrazovce časovače pak bude potřeba použít parametry \texttt{startAfter}, abychom definovali datum a~čas, od kterého další záznamy načítat, pokud načítáme další stránky, a~\texttt{limit}, který určí velikost stránky. Při načítání všech záznamů v~daném období se pak využije parametrů \texttt{startAfter} a~\texttt{endAt}, například když bude potřeba zjistit souhrn odpracovaných hodin v~aktuální den nebo týden. Při výpočtu tohoto souhrnu bude tedy \texttt{UserRepositoryImpl} sčítat doby trvání všech záznamů v~daném období.

Pro obsluhu časovače pak \texttt{UserRepository} poskytuje funkce pro načtení dat časovače a~pro jejich aktualizaci. Při výběru projektu bude uživatel potřebovat vidět všechny své projekty a~názvy jejich klientů, pro což bude sloužit opět \emph{preview} objekt \texttt{ProjectPreview}. \texttt{ClientRepository} tedy nabízí funkce pro získání klienta podle identifikátoru, a~\texttt{UserRepository} nabízí funkce pro získání projektu podle identifikátoru klienta a~projektu, ale také funkci pro získání všech projektů daného uživatele.

Pro komunikaci s~klientem se také využívá vlastních struktur pro reprezentaci chyb. Například modul \texttt{user} používá výjimky ze skupiny \texttt{UserExceptions}, která dědí z~třídy \texttt{BaseException}. V~případě, že některé volání vrátí výjimku s~tímto rodičem, tak ji komunikace umí zakódovat do DTO reprezentace, která je známá pro klienta. Ten může potom rozlišovat mezi různými známými chybami. V~případě, že se nejedná o~známou chybu, zakóduje ji komunikace jako obecnou chybu s~HTTP status kódem 500. Implementace tohoto chování lze nahlédnout ve výpisu kódu \ref{code:be-error-handling}.

\begin{listing}
\caption{Obsluha chyb na backendu}\label{code:be-error-handling}
\begin{minted}{Kotlin}
fun Application.configureRouting(isDebug: Boolean) {
    install(StatusPages) {
        exception<BaseException> { call, baseException ->
            call.respond(
                status = baseException.code,
                baseException.toDto(isDebug)
            )
        }

        exception<Throwable> { call, cause ->
            call.respond(
                status = HttpStatusCode.InternalServerError,
                message = ErrorDto(
                    type = "InternalError",
                    message = "Internal Server Error",
                    debugMessage = if (isDebug) cause.message else null
                )
            )
        }
    }
    
    // ...
}
\end{minted}
\end{listing}

%---------------------------------------------------------------
\subsubsection{Multi-platformní část}
%---------------------------------------------------------------

Multi-platformní část aplikace už rozděluje své moduly podle funkcionalit z~hlediska uživatele, modul \texttt{timer} bude tedy poskytovat všechny \emph{Use Cases} pro potřeby časovače:
\begin{itemize}
\item\texttt{AddTimerEntryUseCase} – Vytvoří nový časový záznam podle parametrů.
\item\texttt{DeleteTimerEntryUseCase} – Smaže časový záznam podle identifikátoru.
\item\texttt{GetProjectsUseCase} – Získá \emph{preview} objekty všech projektů, které má uživatel přiřazeny.
\item\texttt{GetTimerDataPreviewUseCase} – Získá \emph{preview} objekt pro aktuální nastavení časovače. Čistá data o~aktuálním stavu časovače totiž opět obsahují jen identifikátory klienta a~projektu, ale při vizualizaci dat jsou potřeba jejich názvy.
\item\texttt{GetTimerEntriesUseCase} – Získává stránky \emph{preview} objektů časových záznamů, umí tedy pracovat se všemi parametry pro omezení stránky. Zároveň časové záznamy seskupuje do seznamu objektů typu \texttt{TimerEntryGroup}, což je skupina, která obsahuje datum, seznam všech záznamů patřící k~tomuto datu a~součet odpracovaných hodin všech těchto záznamů. Toho se využije při vizualizaci v~nativní aplikaci, kde se nad každou skupinou ukáže datum a~časový souhrn pro daný den.
\item\texttt{GetTimerSummariesUseCase} – Získá souhrny časových záznamů pro aktuální den a~týden.
\item\texttt{UpdateTimerDataUseCase} – Aktualizuje aktuální nastavení časovače.
\end{itemize}

Všechny tyto \emph{Use Cases} automaticky počítají s~tím, že pracují s~daty aktuálně přihlášeného uživatele, žádné parametry pro jeho identifikaci tedy nepotřebují.

Jednotlivé \emph{Use Cases} už komunikují s~backendem, a~to v~nejnižší \emph{infrastructure} vrstvě, která požadavky posílá pomocí knihovny \emph{Ktor}. Příklad toho, jak probíhá komunikace, lze nahlédnout v~ukázce kódu \ref{code:kmp-read-entries-source-impl}.

\begin{listing}
\caption{Funkce pro získávání časových záznamů v~\texttt{RemoteTimerSource}}\label{code:kmp-read-entries-source-impl}
\begin{minted}{Kotlin}
internal class RemoteTimerSource(
    private val client: HttpClient
) : TimerSource {
    override suspend fun readEntries(
        startAfter: String?,
        limit: Int?,
        endAt: String?
    ): Result<PageDto<TimerEntryDto>> =
        runCatchingCommonNetworkExceptions {
            val res = client.get("user/entries") {
                url {
                    startAfter?.let { parameters.append("startAfter", it) }
                    limit?.let { parameters.append("limit", it.toString()) }
                    endAt?.let { parameters.append("endAt", it) }
                }
            }
            res.body<PageDto<TimerEntryDto>>()
        }
        
    // ...
}
\end{minted}
\end{listing}

Multi-platformní část pracuje s~objekty v~DTO reprezentaci, kterou definuje backend. Poté si je také převádí pomocí vlastních funkcí do vlastních doménových reprezentací.

Také lze ve výpisu kódu \ref{code:kmp-read-entries-source-impl} nahlédnout, že celé API volání je obaleno do pomocné funkce \texttt{runCatchingCommonNetworkExceptions}, jejíž implementace lze nahlédnout ve výpisu \ref{code:kmp-run-catching-common-network-exceptions}. Tato funkce odchytává všechny výjimky, které při komunikaci s~backendem mohou vzniknout, a~převádí je do vlastních reprezentací, se kterými poté umí pracovat nativní aplikace.

\begin{listing}
\caption{Odchytávání výjimek při komunikaci s~backendem}\label{code:kmp-run-catching-common-network-exceptions}
\begin{minted}{Kotlin}
internal suspend inline fun <R : Any> runCatchingCommonNetworkExceptions(
    block: () -> R
): Result<R> =
    try {
        Result.Success(block())
    } catch (e: ResponseException) {
        val body = e.response.body<ErrorDto>()

        val error = when (body.type) {
            "Unauthorized" -> BackendError.NotAuthorized(e.response.toString(), e)
            "ProjectNotAssignedToUser" -> BackendError.ProjectNotAssignedToUser(
                body.message,
                e
            )
            "MissingProject" -> BackendError.MissingProject(body.message, e)
            "ProjectNotFound" -> BackendError.ProjectNotFound(body.message, e)
            "ClientNotFound" -> BackendError.ClientNotFound(body.message, e)
            "ClockifyProjectNotFound" -> BackendError.ClockifyProjectNotFound(
                body.message, 
                e
            )
            "ClockifyInvalidApiKey" -> BackendError.ClockifyInvalidApiKey(e)
            "ClockifyUnknownError" -> BackendError.ClockifyUnknownError(e)
            "ClockifyWorkspaceNotFound" -> BackendError.ClockifyWorkspaceNotFound(
                body.message,
                e
            )
            else -> ErrorResult(message = body.message, throwable = e)
        }

        Result.Error(error)
    } catch (e: Throwable) {
        val error = when (e::class.simpleName) {
            "UnknownHostException" -> CommonError.NoNetworkConnection(e)
            "HttpRequestTimeoutException", "ConnectTimeoutException",
            "SocketTimeoutException" -> CommonError.Timeout(e)
            "CancellationException" -> CommonError.Cancelled(e)
            else -> handlePlatformError(e)
        }
        Result.Error(error)
    }
\end{minted}
\end{listing}

%---------------------------------------------------------------
\subsubsection{Nativní aplikace}
%---------------------------------------------------------------

Realizace časovače dodržuje vzhled navržený v~sekci \ref{feature-timer}. Na obrázku \ref{fig:timer-empty-impl} lze nahlédnout obrazovka časovače, jak bude vypadat, pokud uživatel zatím nemá žádná data, jako například nově registrovaný uživatel. Na obrázku \ref{fig:fetch-more-impl} lze zase nahlédnout, jak vypadá tlačítko pro načtení dalších záznamů, pokud se uživatel posune v~načtených datech úplně nahoru. Na obrázku \ref{fig:description-edit-impl} lze nahlédnout stav, kdy si uživatel chce změnit popisek časovače, a~na obrázku \ref{fig:time-selection-impl} zase obrazovka pro ruční vybrání času, pokud chce uživatel přidat záznam manuálně. Na obrázku \ref{fig:project-selection-impl} lze poté vidět výběr z~projektů a~možnost v~nich vyhledávat.

\begin{figure}[p]
    \centering
    \begin{subfigure}[b]{0.4\textwidth}
		\centering
		\includegraphics[width=6cm]{timer-empty-impl.png}
		\caption{Prázdná data}
		\label{fig:timer-empty-impl}
	\end{subfigure}
	\hspace{2cm}
	\begin{subfigure}[b]{0.4\textwidth}
		\centering
		\includegraphics[width=6cm]{fetch-more-impl.png}
		\caption{Načtení další stránky}
		\label{fig:fetch-more-impl}
	\end{subfigure}
	\caption{Realizace časovače}
	\label{fig:timer-impl}
\end{figure}

\begin{figure}[p]
    \centering
    \begin{subfigure}[b]{0.4\textwidth}
		\centering
		\includegraphics[width=6cm]{description-edit-impl.png}
		\caption{Úprava popisu}
		\label{fig:description-edit-impl}
	\end{subfigure}
	\hspace{2cm}
	\begin{subfigure}[b]{0.4\textwidth}
		\centering
		\includegraphics[width=6cm]{time-selection-impl.png}
		\caption{Manuální výběr času}
		\label{fig:time-selection-impl}
	\end{subfigure}
	\caption{Realizace ovládání časovače}
	\label{fig:timer-control-impl}
\end{figure}

\begin{figure}[p]
    \centering
    \begin{subfigure}[b]{0.4\textwidth}
		\centering
		\includegraphics[width=6cm]{project-selection-impl.png}
		\caption{Vybraný projekt}
		\label{fig:selected-project-impl}
	\end{subfigure}
	\hspace{2cm}
	\begin{subfigure}[b]{0.4\textwidth}
		\centering
		\includegraphics[width=6cm]{project-select-search-impl.png}
		\caption{Vyhledávání}
		\label{fig:project-select-search-impl}
	\end{subfigure}
	\caption{Realizace výběru projektu}
	\label{fig:project-selection-impl}
\end{figure}

Jeden problém, který musí nativní aplikace řešit, je situace, když má zobrazit nově načtenou navazující stránku časových záznamů. Můžou nastat dvě situace – buď nová stránka bude obsahovat data skupiny, která má prázdný průnik s~poslední skupinou již načtených dat (tedy již načtená data neobsahují data nějakého dne, který by neměl plně načtené všechny záznamy), nebo bude tento průnik neprázdný (tedy nová data obsahují data, která se musí spojit s~daty nějakého dne, který je již z~části načtený). Nově načtenou stránku tedy nelze vždy obyčejně připojit za již načtená data, ale musí probíhat kontrola, zda se nějaká skupina z~již načtených dat nemá spojit s~nějakou skupinou nových dat. Multi-platformní část také musí poskytovat data o~tom, zda je jednotlivá skupina již plně načtená. Jediný případ, kdy o~tomto faktu může výstup multi-platformního \emph{Use Case} lhát, je ten, pokud se jedná o~poslední skupinu v~načtených datech, jejíž konec ale přesně končí v~místě, kde končí i~záznamy dalšího dne, což ale \emph{Use Case} nemá jak zjistit. Tento případ musí ručně detekovat nativní aplikace při načtení nové stránky a~opravit skupinu, u~které to mohlo nastat, tak, že jí označí za plně načtenou. Implementace této logiky lze nahlédnout ve výpisu kódu \ref{code:view-mode-entry-group-overlap}.

\begin{listing}
\caption{Obsluha nově načtené navazující stránky záznamů}\label{code:view-mode-entry-group-overlap}
\begin{minted}{Swift}
// Fetch more groups
let fetchedGroups: [TimerEntryGroup] = try await getTimerEntriesUseCase.execute(
    params: params
)

var newGroups: [TimerEntryGroup] = []

// If there is an overlapping group
if let firstOfCurrent = currentGroups.first,
   let lastOfNew = fetchedGroups.last,
   firstOfCurrent.date == lastOfNew.date {
    // Append already fetched groups without the last one
    newGroups.append(contentsOf: fetchedGroups.dropLast())
    
    // Calculate total interval of the overlapping group
    var interval: KotlinLong? {
        if lastOfNew.interval == nil 
            && firstOfCurrent.interval == nil 
        { return nil }
        
        let lastOfNewInterval = lastOfNew.interval?.int ?? 0
        let firstOfCurrentInterval = firstOfCurrent.interval?.int ?? 0
        
        return KotlinLong(
            value: (lastOfNewInterval + firstOfCurrentInterval).int64
        )
    }
    
    // Append union of the overlapping group
    newGroups.append(TimerEntryGroup(
        date: firstOfCurrent.date,
        interval: interval,
        entries: lastOfNew.entries + firstOfCurrent.entries,
        isFullyLoaded: lastOfNew.isFullyLoaded
    ))
    
    // Append the fetched groups without the first one
    newGroups.append(contentsOf: currentGroups.dropFirst())
} else {
    // No overlapping groups, just append
    newGroups.append(contentsOf: fetchedGroups)
    
    // Fix the `isFullyLoaded` flag if necessary
    if let firstOfCurrent = currentGroups.first, !firstOfCurrent.isFullyLoaded {
        firstOfCurrent.isFullyLoaded = true
    }
    newGroups.append(contentsOf: currentGroups)
}

// Display the updated data
state.listData = .data(newGroups)
\end{minted}
\end{listing}

Veškerá ostatní logika spočívá v~přímočarém používání \emph{Use Cases} podle toho, co je úmyslem uživatele. \emph{View Model} této obrazovky si mimo to musí pamatovat ručně zadaný konec záznamu, jelikož ten není součástí aktuálního nastavení časovače, nebo také musí každou sekundu aktualizovat čas, který se ukazuje na běžících stopkách, ale také tento čas přičítat k~souhrnům aktuálního dne a~týdne.

%---------------------------------------------------------------
\subsection{Profil uživatele}
%---------------------------------------------------------------

Tato funkcionalita pokrývá operace, které může uživatel dělat se svým profilem, se svými klienty a~se svými projekty. 

%---------------------------------------------------------------
\subsubsection{Backend}
%---------------------------------------------------------------

Obsluhu požadavků týkajících se projektů bude řešit modul pro projekty, požadavky týkající se klientů bude řešit modul pro klienty a~požadavky týkající se profilu bude řešit modul pro uživatele.

Co se týče funkcí, co může uživatel dělat se svým profilem, tak se na obrazovce profilu může buď odhlásit, nebo si svůj účet smazat.

Odhlášení je čistě autentizační záležitost, probíhá tedy pouze na straně klienta, kde je součástí modulu \texttt{auth}, jak bylo popsáno v~sekci \ref{onboarding-impl}.

Smazání účtu už je ale z~hlediska dat zajímavější záležitost. Během mazání účtu bude potřeba smazat všechna data, která jsou s~uživatelem propojená, tedy všechny jeho záznamy a~integrace. V~první řadě bude tedy potřeba smazat každý dokument v~kolekci \texttt{/entries/\{uid\}/entries}, protože pro smazání kolekce je potřeba smazat každý dokument, nelze smazat celou kolekci najednou. Poté se bude moct smazat dokument \texttt{/entries/\{uid\}}. Poté se budou muset smazat všechny integrace, tedy všechny dokumenty v~kolekci \texttt{/users/\{uid\}/integrations}, poté všechny informace o~tom, jaké klienty a~projekty má uživatel přiřazené, tedy dokumenty v~kolekci \texttt{/users/\{uid\}/clients} a~ve finále se může smazat dokument uživatele, tedy \texttt{/users/\{uid\}}. Tím budou smazána všechna data související s~uživatelem ve \emph{Firestore} databázi. Poté je ještě potřeba smazat uživatele z~\emph{Firebase} autentizace, aby se na něj už nešlo přihlásit. Celou tuto logiku implementuje funkce \texttt{deleteUser(uid: String)} v~\texttt{UserSourceImpl}.

Další, co bude backend muset implementovat, jsou všechny CRUD (Create, Read, Update, Delete) operace pro klienty a~projekty. Za to mají odpovědnost příslušné \texttt{ProjectRepository}, \texttt{ClientRepository} a~jejich \emph{Sources}. Specifikem projektů je, že pro jejich jednoznačnou identifikaci nestačí pouze identifikátor projektu, ale je potřeba i~identifikátor klienta, ke kterému patří. Také je potřeba implementovat funkce pro přečtení všech klientů pro konkrétního uživatele a~přečtení všech projektů daného uživatele. Poté je také potřeba poskytnout API pro přiřazení klienta nebo projektu k~uživateli, samotné vytvoření je totiž pouze zařadí do kolekcí, ale nepřiřadí je k~žádnému uživateli. Implementace \emph{Route}, která obsluhuje přiřazení projektu k~uživateli, lze nahlédnout ve výpisu kódu \ref{code:be-route-assign-project}. Toto přiřazení řeší speciální \emph{PUT endpoint}, který nepřijímá žádné tělo požadavku.

\begin{listing}
\caption{\emph{Route} pro přiřazení klienta k~uživateli}\label{code:be-route-assign-project}
\begin{minted}{Kotlin}
fun Routing.userRoute() {

    // ...
    
    authenticate {
    
        // ...
        
        route("/user") {
        
            // ...
            
            route("/projects") {
            
                // ...
                
                put("/add") {
                    val user = call.requireUserPrincipal().user
                    val clientId: String by call.request.queryParameters
                    val projectId: String by call.request.queryParameters

                    userRepository.assignProjectToUser(user.uid, clientId, projectId)

                    call.respond(HttpStatusCode.OK)
                }
            }
            
            // ...
            
        }
    }
}
\end{minted}
\end{listing}

Úprava a~mazání projektů a~klientů je také poměrně komplexní záležitost, jelikož je spolu s~nimi potřeba upravit nebo smazat řadu dat, které s~nimi souvisí. Například, pokud se u~upravovaného projektu změní klient, ke kterému patří, bude potřeba změnit všechny časové záznamy, které tento projekt obsahují, aby měly identifikátor nového klienta, dále je potřeba u~všech uživatelů, kteří měli tento projekt přiřazený, toto přiřazení přesunout z~dokumentu původního klienta do dokumentu nového klienta, je potřeba samotný projekt přesunout do kolekce nového klienta, a~ověřit aktuální data časovače u~všech uživatelů, protože tam tento projekt se starým klientem může být. Podobně složité může být například mazání klienta. Spolu s~ním je totiž potřeba smazat u~všech uživatelů případná přiřazení tohoto klienta a~jeho projektů, smazat všechny projekty, které ke klientovi patří, smazání všech časových záznamů všech uživatelů, kteří mají tohoto klienta přiřazeného, a~až poté smazat klienta. Všechnu tuto logiku definují implementace jednotlivých \emph{Sources}, tedy \texttt{ProjectSourceImpl} a~\texttt{ClientSourceImpl}.

%---------------------------------------------------------------
\subsubsection{Multi-platformní část}
%---------------------------------------------------------------

Multi-platformní část aplikace poskytuje \emph{Use Cases} pro všechny funkce popsané výše, tedy:
\begin{itemize}
\item\texttt{AddAndAssignClientUseCase} – Přiřadí klienta k~přihlášenému uživateli.
\item\texttt{AddAndAssignProjectUseCase} – Přiřadí projekt k~přihlášenému uživateli.
\item\texttt{DeleteUserUseCase} – Smaže uživatele podle identifikátoru.
\item\texttt{GetClientsUseCase} – Získá všechny klienty přiřazené k~přihlášenému uživateli.
\item\texttt{GetClientUseCase} – Získá klienta podle jeho identifikátoru.
\item\texttt{GetProjectPreviewUseCase} – Získá \emph{preview} objekt pro projekt podle jeho identifikátoru.
\item\texttt{GetUserEmailUseCase} – Získá e-mail přihlášeného uživatele, který se pak zobrazuje v~obrazovce profilu. Tuto informaci nezískává z~backendu, ale pomocí \emph{Firebase} autentizace. E-mail uživatele není ve \emph{Firestore} databázi vůbec uložen, jedná se pouze o~autentizační nástroj.
\item\texttt{RemoveClientUseCase} – Smaže klienta podle identifikátoru.
\item\texttt{RemoveProjectUseCase} – Smaže projekt podle identifikátoru.
\item\texttt{UpdateClientUseCase} – Aktualizuje klienta.
\item\texttt{UpdateProjectUseCase} – Aktualizuje projekt.
\end{itemize}

%---------------------------------------------------------------
\subsubsection{Nativní aplikace}
%---------------------------------------------------------------

Realizace profilu v~iOS aplikaci dodržuje návrh ze sekce \ref{feature-profile}. Na obrázku \ref{fig:profile-impl} lze nahlédnout přehled profilu a~dialog, který se zobrazí při pokusu o~smazání účtu. Na obrázku \ref{fig:clients-impl} lze poté vidět seznam klientů, které má uživatel přiřazené, a~možnost v~nich vyhledávat. Dále lze na obrázku \ref{fig:client-projet-detail-impl} vidět detaily klienta a~projektu, obrázek \ref{fig:project-detail-impl} navíc ukazuje stav detailu během ukládání změn, během kterého s~prvky nelze interagovat. Poslední obrázek \ref{fig:project-list-client-selection-impl} poté znázorňuje seznam projektů a~výběr klienta k~projektu.

\begin{figure}[p]
    \centering
    \begin{subfigure}[b]{0.4\textwidth}
		\centering
		\includegraphics[width=6cm]{profile-impl.png}
		\caption{Přehled}
		\label{fig:profile-overview-impl}
	\end{subfigure}
	\hspace{2cm}
	\begin{subfigure}[b]{0.4\textwidth}
		\centering
		\includegraphics[width=6cm]{profile-delete-impl.png}
		\caption{Dialog pro smazání účtu}
		\label{fig:profile-delete-impl}
	\end{subfigure}
	\caption{Realizace profilu}
	\label{fig:profile-impl}
\end{figure}

\begin{figure}[p]
    \centering
    \begin{subfigure}[b]{0.4\textwidth}
		\centering
		\includegraphics[width=6cm]{client-list-impl.png}
		\caption{Seznam}
		\label{fig:client-list-impl}
	\end{subfigure}
	\hspace{2cm}
	\begin{subfigure}[b]{0.4\textwidth}
		\centering
		\includegraphics[width=6cm]{client-search-impl.png}
		\caption{Vyhledávání}
		\label{fig:client-search-impl}
	\end{subfigure}
	\caption{Realizace klientů}
	\label{fig:clients-impl}
\end{figure}

\begin{figure}[p]
    \centering
    \begin{subfigure}[b]{0.4\textwidth}
		\centering
		\includegraphics[width=6cm]{client-detail-impl.png}
		\caption{Detail klienta}
		\label{fig:client-detail-impl}
	\end{subfigure}
	\hspace{2cm}
	\begin{subfigure}[b]{0.4\textwidth}
		\centering
		\includegraphics[width=6cm]{project-detail-impl.png}
		\caption{Detail projektu během ukládání}
		\label{fig:project-detail-impl}
	\end{subfigure}
	\caption{Detail klienta a~projektu}
	\label{fig:client-projet-detail-impl}
\end{figure}

\begin{figure}[p]
    \centering
    \begin{subfigure}[b]{0.4\textwidth}
		\centering
		\includegraphics[width=6cm]{project-list-impl.png}
		\caption{Seznam projektů}
		\label{fig:project-list-impl}
	\end{subfigure}
	\hspace{2cm}
	\begin{subfigure}[b]{0.4\textwidth}
		\centering
		\includegraphics[width=6cm]{client-selection-impl.png}
		\caption{Výběr klienta k~projektu}
		\label{fig:client-selection-impl}
	\end{subfigure}
	\caption{Seznam projektů a~výběr klienta k~projektu}
	\label{fig:project-list-client-selection-impl}
\end{figure}

%---------------------------------------------------------------
\subsection{Integrace}
%---------------------------------------------------------------

Tato funkcionalita pokrývá možnosti, jak může uživatel napojovat aplikaci na navržené spouštěče měření času (import) a~jak propojit aplikaci s~existujícími systémy (export), jak bylo navrženo v~sekci \ref{feature-integration}.

%---------------------------------------------------------------
\subsubsection{Backend}
%---------------------------------------------------------------

Aplikace uživateli umožňuje si nastavené integrace vytvářet, ukládat, upravovat a~mazat. Backend bude tedy muset opět implementovat všechny CRUD operace a~získání všech integrací daného uživatele, tentokrát pro objekty integrací. Toto je odpovědností \texttt{IntegrationRepository} a~jejího \emph{Source}. Na rozdíl od uživatelů, klientů nebo projektů, nejsou integrace propojeny s~jinými typy objektů napříč celou databází, jejich úprava je tedy poměrně jednoduchá a~týká se jen vlastních objektů.

Co se týče exportu do CSV souboru, tak celou tuto logiku implementuje backend, který CSV soubor vytvoří a~klientovi ho pošle hotový. Ve výpisu kódu \ref{code:be-route-csv} lze nahlédnout implementace \emph{Route} pro export do CSV souboru, kde lze vidět, že nejprve získá všechny záznamy v~zadaném období, ty uloží do dočasného lokálního souboru, který pošle klientovi, a~poté dočasný soubor smaže.

\begin{listing}
\caption{\emph{Route} pro export do CSV souboru}\label{code:be-route-csv}
\begin{minted}{Kotlin}
fun Routing.integrationRoute() {

    // ...
    
    authenticate {
        route("/integrations") {
        
            // ...
            
            route("/csv") {
                get {
                    val user = call.requireUserPrincipal().user
                    val from = call.request.queryParameters["from"]
                    val to = call.request.queryParameters["to"]

                    val entries = userRepository.readEntryPreviews(
                        uid = user.uid,
                        startAfter = to?.toInstant(),
                        limit = null,
                        endAt = from?.toInstant()
                    ).data
                    val csv = repository.readCsv(entries.reversed())

                    call.respondFile(csv)

                    repository.deleteTempCsvFile(csv.name)
                }
            }
            
            // ...
        }
    }
}       
\end{minted}
\end{listing}

Napojení aplikace na spouštěče měření (import), které byly pro aplikaci Trackee navrženy, je záležitostí klienta, jelikož se jedná o~systémovou aplikaci \emph{Zkratky}. Bude pouze potřeba poskytnout dodatečné API, které umožní samotné zapnutí časovače, jeho zrušení a~jeho vypnutí spolu s~vytvořením nového časového záznamu. Tyto funkce totiž pro potřeby funkcionality časovače implementuje klientská aplikace ručně, která aktualizuje aktuální nastavení časovače, čímž reálně ovlivňuje to, zda časovač běží, od kdy běží, a~podobně. Při tvorbě nového záznamu také volá pouze API pro ruční vytvoření nového záznamu, protože všechna potřebná data zná. Při použití zkratek se jedná pouze o~jednoduchý požadavek, který ale nemá žádné informace o~aktuálním nastavení časovače, takže pro tyto potřeby bude tato logika na backendu. Alternativou by bylo, aby vyvolané \emph{Zkratky} přímo interagovaly s~rozhraním aplikace, ale to by nedávalo moc smysl, protože v~tomto případě je tím rozhraním právě aplikace \emph{Zkratky}.

%---------------------------------------------------------------
\subsubsection{Multi-platformní část}
%---------------------------------------------------------------

Multi-platformní část pokrývá funkcionalitu integrací ve dvou modulech – \texttt{integration} a~\texttt{intent}. Modul \texttt{integration} pokrývá všechny funkce, které aplikace nabízí v~kartě \emph{Integrace} (návrh z~obrázku \ref{fig:integration-list}), a~modul \texttt{intent} nabízí funkce pro \emph{Intents}, což jsou obsluhy jednotlivých zkratek aplikace \emph{Zkratky}, jak bylo popsáno v~sekci \ref{feature-integration-import}.

Modul \texttt{integration} poskytuje tyto \emph{Use Cases}:
\begin{itemize}
\item\texttt{AddIntegrationUseCase} – Přidá novou integraci.
\item\texttt{DeleteIntegrationUseCase} – Smaže integraci podle jejího identifikátoru.
\item\texttt{ExportToCsvUseCase} – Exportuje data ve zvoleném období do CSV souboru.
\item\texttt{GetIntegrationsUseCase} - Získá všechny integrace přihlášeného uživatele.
\item\texttt{GetIntegrationUseCase} – Získá integraci podle jejího identifikátoru.
\item\texttt{UpdateIntegrationUseCase} – Aktualizuje integraci.
\end{itemize}
Modul \texttt{intent} poté poskytuje následující \emph{Use Cases}:
\begin{itemize}
\item\texttt{CancelTimerUseCase} – Zruší běžící časovač, zahodí tedy jeho data. Pokud časovač neběží, neudělá nic.
\item\texttt{StartTimerUseCase} – Spustí časovač, tedy aktualizuje jeho data tak, že bude ve stavu \texttt{active} a~bude měřit od momentu, kdy byl \emph{Use Case} zavolán. Pokud časovač už běží, neudělá nic (tedy počátek měření zůstane nezměněn).
\item\texttt{StopTimerUseCase} – Zastaví časovač a~vytvoří z~jeho dat nový časový záznam. Pokud například chybí zadání vybraného projektu, tak \emph{Use Case} vrátí chybu. Pokud časovač neběží, neudělá nic.
\end{itemize}
U~těchto \emph{Use Cases} je důležité, aby v~případě, že se daný \emph{Intent} snaží dostat časovač do stavu, ve kterém již je, opravdu neudělal nic. Tedy například v~případě opakovaného zapnutí nebyl měněn čas zapnutí podle nového volání. Je tak potřeba proto, protože operace zkratek jsou navrženy tak, aby byly idempotentní\footnote{Operace je idempotentní, pokud jejím opakovaným použitím na nějaký vstup vznikne stejný výstup, jako vznikne jediným použitím dané operace. \cite{idempotence}}. Tyto zkratky totiž pravděpodobně budou součástí nějakých automatizací, které můžou například zapínat časovač podle polohy. Pokud by opakované spuštění zkratky přepisovalo data podlé nových informací, byla by původní data smazána a~uživatel by o~ně přišel. Podnětem pro budoucí vylepšení aplikace může být například to, aby se v~případě, že uživatel vyvolá zkratku pro spuštění časovače s~novými daty, když už časovač běží, časovač automaticky zastavil, uložil nový časový záznam z~dosud běžících dat, a~poté se znovu zapnul pro nová data.

%---------------------------------------------------------------
\subsubsection{Nativní aplikace}
%---------------------------------------------------------------

Uživatelské rozhraní iOS aplikace dodržuje návrh ze sekce \ref{feature-integration}. Na obrázku \ref{fig:integrations-empty-impl} lze vidět kartu integrací ve stavu, když uživatel nemá vytvořené žádné integrace. Na obrázku \ref{fig:new-integration-impl} lze poté vidět rozhraní pro tvorbu nové CSV integrace. Export CSV dat a~jejich vizualizace v~aplikaci \emph{Numbers} lze nahlédnout na obrázku \ref{fig:export-csv-impl}. Pro tvorbu zkratek bude sloužit systémová aplikace \emph{Zkratky}, jejíž hlavní stránka lze vidět na obrázku \ref{fig:shortcuts-impl}, rozhraní pro tvorbu nové zkratky lze pak vidět na obrázku \ref{fig:new-shortcut-impl}. Na obrázku \ref{fig:new-trackee-shortcut} lze poté vidět nabídka možností zkratek pro aplikaace Trackee a~možnost nastavení parametrů zkratky. A~nakonec je na obrázku \ref{fig:automations-impl} vidět možnost tvorby automatizací pro operace s~časovačem.

\begin{figure}[p]
    \centering
    \begin{subfigure}[b]{0.4\textwidth}
		\centering
		\includegraphics[width=6cm]{integrations-empty-impl.png}
		\caption{Prázdný seznam}
		\label{fig:integrations-empty-impl}
	\end{subfigure}
	\hspace{2cm}
	\begin{subfigure}[b]{0.4\textwidth}
		\centering
		\includegraphics[width=6cm]{new-integration-impl.png}
		\caption{Nová CSV integrace}
		\label{fig:new-integration-impl}
	\end{subfigure}
	\caption{Realizace integrací}
	\label{fig:integrations-impl}
\end{figure}

\begin{figure}[p]
    \centering
    \begin{subfigure}[b]{0.4\textwidth}
		\centering
		\includegraphics[width=6cm]{export-csv-sheet-impl.png}
		\caption{Dialog pro exportovaný CSV soubor}
		\label{fig:export-csv-sheet-impl}
	\end{subfigure}
	\hspace{2cm}
	\begin{subfigure}[b]{0.4\textwidth}
		\centering
		\includegraphics[width=6cm]{export-csv-numbers-impl.png}
		\caption{Exportovaný CSV soubor v~aplikaci \emph{Numbers}}
		\label{fig:export-csv-numbers-impl}
	\end{subfigure}
	\caption{Realizace exportu do CSV}
	\label{fig:export-csv-impl}
\end{figure}

\begin{figure}[p]
    \centering
    \begin{subfigure}[b]{0.4\textwidth}
		\centering
		\includegraphics[width=6cm]{shortcuts-impl.png}
		\caption{Přehled zkratek}
		\label{fig:shortcuts-impl}
	\end{subfigure}
	\hspace{2cm}
	\begin{subfigure}[b]{0.4\textwidth}
		\centering
		\includegraphics[width=6cm]{new-shortcut-impl.png}
		\caption{Tvorba nové zkratky}
		\label{fig:new-shortcut-impl}
	\end{subfigure}
	\caption{Systémová aplikace \emph{Zkratky}}
	\label{fig:system-shortcuts-impl}
\end{figure}

\begin{figure}[p]
    \centering
    \begin{subfigure}[b]{0.4\textwidth}
		\centering
		\includegraphics[width=6cm]{new-shortcut-trackee-impl.png}
		\caption{Možnosti zkratek aplikace Trackee}
		\label{fig:new-shortcut-trackee-impl}
	\end{subfigure}
	\hspace{2cm}
	\begin{subfigure}[b]{0.4\textwidth}
		\centering
		\includegraphics[width=6cm]{new-shortcut-params-impl.png}
		\caption{Parametry zkratky}
		\label{fig:new-shortcut-params-impl}
	\end{subfigure}
	\caption{Tvorba nové zkratky pro Trackee}
	\label{fig:new-trackee-shortcut}
\end{figure}

\begin{figure}[p]
    \centering
    \begin{subfigure}[b]{0.4\textwidth}
		\centering
		\includegraphics[width=6cm]{new-automation-impl.png}
		\caption{Tvorba nové automatizace}
		\label{fig:new-automation-impl}
	\end{subfigure}
	\hspace{2cm}
	\begin{subfigure}[b]{0.4\textwidth}
		\centering
		\includegraphics[width=6cm]{automations-list-impl.png}
		\caption{Seznam automatizací}
		\label{fig:automations-list-impl}
	\end{subfigure}
	\caption{Automatizace}
	\label{fig:automations-impl}
\end{figure}

V~sekci \ref{feature-integration-import} bylo také zmíněno, že pro podporu interakce s~aplikací pomocí systémových zkratek je potřeba implementovat takzvané \emph{App Intents}. Tyto \emph{intents} představují nějaký úmysl uživatele, který může pomocí zkratky spustit. \cite{ios-app-intents}

Struktury pro reprezentaci \emph{App Intents} je potřeba definovat v~hlavním \emph{targetu} aplikace, nelze je definovat v~žádném balíčku, protože by pak nebyly detekovány systémem, aby se zobrazily v~aplikaci \emph{Zkratky}. Jednotlivé \emph{intents} musí implementovat protokol (rozhraní) \texttt{AppIntent}, který vyžaduje, aby \emph{intent} obsahoval minimálně název, popis a~funkci \texttt{perform()}, která se zavolá při pokusu o~provedení \emph{intentu}. \texttt{AppIntent} také vyžaduje, aby byly texty definovány jako \texttt{LocalizedStringResource}, nelze tedy použít lokalizaci definovanou v~balíčku \emph{UIToolkit}. Pro potřeby \emph{App Intents} slouží ve složce iOS projektu složka \texttt{Intents}, kde jsou definovány \emph{intenty} a~jejich entity podle funkcionalit aplikace, a~také \emph{Resources}, tedy \texttt{Localizable.strings} soubor, který bude obsahovat texty a~jejich lokalizace pro všechny \emph{App Intents}.

Ve výpisu kódu \ref{code:start-timer-intent} lze nahlédnout implementace \emph{intentu} pro zapnutí časovače. Implementace obsahuje referenci na typ \texttt{StartTimerUseCase}, který poskytuje \emph{Dependency Injection} stejným způsobem, jako ve \emph{View Models}. Dále iplementace obsahuje definici názvu, popisu a~parametrů, které daný \emph{intent} může přijímat. A~nakonec implementace obsahuje definici funkce \texttt{perform()}, která se spustí při spuštění \emph{intentu}.

\begin{listing}
\caption{\emph{App Intent} pro zapnutí časovače}\label{code:start-timer-intent}
\begin{minted}{Swift}
struct StartTimerIntent: AppIntent {
    
    // MARK: - Dependencies
    
    @Injected(\.startTimerUseCase) private var startTimerUseCase
    
    // MARK: - Required
    
    static var title = LocalizedStringResource("start_timer_intent_title")
    
    static var description = IntentDescription("start_timer_intent_description")
    
    // MARK: - Parameters
    
    @Parameter(title: "start_timer_intent_parameter_description")
    var description: String?
    
    @Parameter(title: "start_timer_intent_parameter_project")
    var project: ProjectPreviewEntity?
    
    // MARK: - Perform
    
    func perform() async throws -> some IntentResult {
        let params = StartTimerUseCaseParams(
            body: StartTimerBody(
                clientId: project?.clientId,
                projectId: project?.projectId,
                description: description
            )
        )
        try await startTimerUseCase.execute(params: params)
        return .result()
    }
}       
\end{minted}
\end{listing}

Aby mohl nějaký objekt být parametrem \emph{App Intentu}, musí splňovat požadavky protokolu \texttt{AppEntity}. Tento protokol vyžaduje po objektech, které ho implementují, aby obsahovaly proměnnou \texttt{displayRepresentation} typu \texttt{DisplayRepresentation}, která definuje, jak se daný parametr zobrazí uživateli, tedy nějaký název, případně podnázev a~obrázek. Také po objektech vyžaduje, aby implementovaly statickou proměnnou \texttt{defaultQuery}, což je další objekt, který musí implementovat protokol \texttt{EntityQuery}. Tento protokol slouží pro objekty, které definují, odkud se mají brát všechny možnosti pro výběr konkrétní instance parametru, k~čemuž slouží funkce \texttt{suggestedEntities()} a~\texttt{entities(for identifiers: [String])}.

Pro potřeby \emph{intentu} \texttt{StartTimerIntent} je potřeba entita \texttt{ProjectPreviewEntity}, která představuje objekt projektu, který si uživatel může nastavit jako parametr \emph{intentu}. Tento objekt v~podstatě reflektuje \texttt{ProjectPreview}, který se do této reprezentace umí převést. Obsahuje tedy název, název klienta a~případně typ projektu pro určení obrázku. Identifikátor tohoto objektu má strukturu \texttt{<clientID>-<projectID>}, protože je potřeba, aby měl pouze jeden identifikátor. Implementace \texttt{ProjectPreviewEntityQuery} poté získává objekty projektů přímo přes \texttt{GetProjectsUseCase} a~\texttt{GetProjectPreviewUseCase}. Ostatní \emph{intenty} žádné parametry nepotřebují.

Propagace chyb do aplikace \emph{Zkratky} funguje automaticky, protože když nějaký \emph{Use Case} vrátí chybu, tak vrátí strukturu \texttt{KMPError}, která sama umí poskytnout lokalizovanou chybovou hlášku, která se zobrazuje jak v~aplikaci, tak v~aplikaci \emph{Zkratky}.


































%---------------------------------------------------------------
\chapter{Testování}
%---------------------------------------------------------------

Výslednou aplikaci a~její uživatelské rozhraní je potřeba důkladně otestovat. Tato kapitola popisuje implementaci automatických testů a~scénáře testování s~reálnými uživateli. Výsledky z~testování jsou poté zhodnoceny.

%---------------------------------------------------------------
\section{Automatické testování}
%---------------------------------------------------------------

Pro účely automatizovaného testování se u~Apple platforem používá knihovna \emph{XCTest} \cite{xctest}. Jak již bylo zmíněno, \emph{Clean architecture} ja navržená tak, aby se dala dobře otestovat jednotkovými testy, které jsou popsány v~následující sekci.

Dalšími možnostmi testování jsou UI testy a~testy výkonu. Aplikace Trackee implementaci těchto typů testů neobsahuje. Jejich implementace může být podnětem pro budoucí vylepšení aplikace.

%---------------------------------------------------------------
\subsection{Jednotkové testování}
%---------------------------------------------------------------

V~aplikaci jsou jednotkovými testy pokryty všechny \emph{View Modely}. Pro každý z~nich je v~modulu, ke kterému patří, soubor, který daný \emph{View Model} testuje. Např. pro \texttt{LoginViewModel.swift} bude testovací soubor \texttt{LoginViewModelTests.swift}.

\emph{View Model} testy jsou navržené tak, aby testovaly každý jednotlivý \emph{Intent}, pokud existuje vhodná možnost, jak test provést. Některé intenty vhodně otestovat nelze – například takové, které otevírají nějaké systémové dialogy, které nemají žádný výstup pro \emph{View Model} aplikace. Daný test tedy vždy ověřuje, zda se stav \emph{View Modelu} po aplikování \emph{Intentu} změnil podle očekávání.

Jednotkové testy \emph{View Modelů} využívají \emph{Mocky} pro \emph{Use Cases}, tedy určité zjednodušené náhrady těchto \emph{Use Cases}, u~kterých si každý test může definovat, co bude daný \emph{Use Case Mock} vracet za hodnotu. Toto je klasický přístup k~jednotkovému testování – testujeme zde \emph{View Modely}, nikoli \emph{Use Cases}.

Předpokládejme tedy, že máme \emph{View Model} se strukturou ve výpisu \ref{code:tested-vm-structure}. Struktura testu poté bude vypadat jako ve výpisu \ref{code:vm-test-sctructure}. Struktura obsahuje následující:
\begin{itemize}
\item\texttt{flowController} – \emph{Mock} pro \emph{FlowController}, tedy náhrada reálného controlleru v~aplikaci. Tento \emph{Mock} slouží k~tomu, abychom mohli otestovat, jaká hodnota byla vložena do posledního volání funkce \texttt{handleFlow(\_:)} a~kolikrát byla tato funkce zavolána. Toto slouží k~otestování toho, zda se bude apikace snažit uživatele navigovat do předpokládané destinace.
\item\emph{Mocky} pro \emph{Use Cases} – zde vytváříme instance \emph{Mocků} pro \emph{Use Cases}, které jsou v~daném \emph{View Modelu} používány. Je potřeba to dělat takto ručně, protože kdybychom použili registrování pomocí \texttt{Container.shared.registerUseCaseMocks()}, tak bychom nemohli v~testech měnit hodnoty, které dané \emph{Use Cases} budou vracet.
\item\texttt{createViewModel()} – tato funkce se bude volat na začátku každého testu, aby získal instanci pro otestování. Pokud konstruktor \emph{View Modelu} vyžaduje nějaké parametry, můžou být předány jako parametr této funkce, nebo zde můžou mít nějakou definovanou hodnotu. V~této funkci se také registrují \emph{Mocky} pro \emph{Use Cases} definované výše.
\item Testovací funkce pro jednotlivé intenty – funkce, které mají v~názvu prefix \texttt{test}, tak se automaticky považují ze testovací funkce a~spouštějí se během testu. Každý vhodně otestovatelný intent zde tedy bude mít vlastní funkci, případně více funkcí, pokud lze otestovat úspěch, neúspěch, různé formy úspěchu, a~podobně. Tyto funkce dodržují strukturu given-when-then, tedy rozdělení do tří částí (předdefinované hodnoty a~konstanty, interakce, ověření).
\end{itemize}

\begin{listing}
\caption{Struktura View Modelu pro testování}\label{code:tested-vm-structure}
\begin{minted}{Swift}
final class SomeViewModel: BaseViewModel, ViewModel, ObservableObject {
	
    // ...
	
    @Injected(\.someUseCase) private var someUseCase
	
    // ...
	
    enum Intent {
        case doThis
        case doThat
    }
	
    // ...
}
\end{minted}
\end{listing}

\begin{listing}
\caption{Struktura View Model testu}\label{code:vm-test-sctructure}
\begin{minted}{Swift}
@MainActor
final class SomeViewModelTests: XCTestCase {

    private let flowController = FlowControllerMock<SomeFlow>(
        navigationController: UINavigationController()
    )
    
    private let someUseCaseMock = SomeUseCaseMock(
        executeReturnValue: /* Some default return value */
    )
    
    private func createViewModel(
        someOptionalParam: SomeType
    ) -> SomeViewModel {
        Container.shared.someUseCase.register { self.someUseCaseMock }
	
        return SomeViewModel(
            someParam: someOptionalParam,
            flowController: flowController
        )
    }
    
    // MARK: - Tests
    
    func testDoThis() async {
        // given
        let vm = createViewModel()
        // ... some constants and given values
        let someConstant = ...
    	
        // when
        vm.onIntent(.doThis)
        // ... some interactions
        await vm.awaitAllTasks()
		
        // then
        // ... assertions
        XCTAssertEqual(vm.state.someValue, someConstant)
    }
    
    func testDoThat() async {
        // ...
    }
}
\end{minted}
\end{listing}

Jednotkové testy \emph{View Modelů} testují byznysovou prezentační vrstvy. Pro otestování dalších vrstev je možné v~multi-platformním modulu testovat \emph{Use Cases}, \emph{Repositories} a~\emph{Sources}. V~backendové části je také možné implementovat jednotkové testy pro \emph{Repositories}, \emph{Sources}, ale i~pro \emph{Routes}. Implementace testů pro tyto vrstvy může být podnětem pro budoucí vylepšení aplikace.

%---------------------------------------------------------------
\section{Uživatelské testování}
%---------------------------------------------------------------

Pro účely uživatelského testování byly navrženy scénáře, podle kterých se testeři mají řídit. Tyto scénáře vycházejí z~navržených případů užití v~sekci \ref{features} a~měly by pokrývat většinu funkcionalit aplikace, kterých uživatel může využít. Povinnými účastníky během testování budou moderátor a~uživatel. Moderátor bude uživateli dávat instrukce a~dokumentovat, jak uživatel s~mobilním telefonem interaguje. Uživatel se bude snažit instrukce plnit a~bude při tom nahlas popisovat své akce s~telefonem (co vidí, na co se chystá kliknout, co očekává, že se stane, atd.), aby usnadnil následnou analýzu celé situace. 

Uživatelského testování se zúčastnilo 5 testerů. Záznamy z~testování lze nahlédnout v~\cite{ui-testing-playlist}. Přehled testerů lze nahlédnout v~tabulce \ref{table:testers}. Instrukce, podle kterých se testeři během testu řídili, lze nahlédnout v~příloze \ref{appendix:ui-testing-instructions}. Protokol z~testování lze nahlédnout v~příloze \ref{appendix:ui-testing-protocol}.

Uživatelské testování probíhalo na produkčním prostředí aplikace Trackee, tedy na prostředí, které by mělo cílit na reálné uživatele a~které poskytuje omezené možnosti ladění, jako jsou například omezené popisy neznámých chyb. Aplikace, kterou testeři testovali, také komunikovala s~instancí backendu na lokální síti, a~nikoli s~instancí nasazenou v~aplikaci \emph{Railway} (více informací v~sekci \ref{backend-deployment}). Lokální instance byla zvolena proto, aby byli testeři během testování odstíněni od prodlev, které jsou způsobeny velkou vzdáleností od instance v~aplikaci \emph{Railawy}.

\begin{table}\centering
\begin{tabular}{l|c|c|c}
	Tester		& Pohlaví	& Používaný OS	& Používá aplikace pro měření času	\tabularnewline \hline 
 	F. W.		& Muž		& Android       & ANO (Clockify)	                \tabularnewline \hline
	D. K.		& Muž		& iOS	        & ANO (Clockify)                	\tabularnewline \hline
	D. Ž.		& Muž		& Android       & ANO (Clockify)               		\tabularnewline \hline
	E. Č.		& Žena		& iOS	        & NE 	                	        \tabularnewline \hline
	T. S.		& Muž		& Android       & ANO (Clockify, Toggl Track)	    \tabularnewline \hline
\end{tabular}
\vspace{0.5cm}
\caption[Přehled testerů aplikace]{~Přehled testerů aplikace}\label{table:testers}
\end{table} 

%---------------------------------------------------------------
\subsection{Scénář}
%---------------------------------------------------------------

\begin{description}
\item[Tvorba uživatelského účtu:] Každý nový uživatel mobilní aplikace Trackee si bude muset nejprve vytvořit svůj uživatelský účet. V~tomto scénáři bude uživatel instruován k~tomu, aby aplikaci poprvé spustil, vytvořil si nový uživatelský účet pomocí předepsaného e-mailu a~hesla a~následně se do aplikace pod tímto účtem přihlásil. 
\item[Vytvoření nového klienta:] Po úspěšné registraci bude uživatel instruován, aby si vytvořil 2 nové klienty s~předepsanými názvy.
\item[Vytvoření nového projektu:] Po úspěšném vytvoření klientů bude uživatel instruován, aby vytvořil 2 nové projekty s~předepsanými vlastnostmi.
\item[Spuštění časovače:] Po úspěšném vytvoření projektů dostane uživatel instrukci, aby spustil časovač pro měření odpracovaného času a~přiřadil mu předepsaný projekt a~popis.
\item[Změna začátku časovače:] Uživatel bude instruován k~tomu, aby u~spuštěného časovače změnil začátek na předepsaný čas.
\item[Zastavení časovače:] Uživatel bude instruován, aby ukončil časovač a~uložil časový záznam, který do teď měřil.
\item[Manuální přidání časového záznamu:] Uživatel bude instruován, aby ručně přidal časový záznam s~předepsanými parametry a~časem.
\item[Úprava projektu:] Uživatel bude instruován, aby aktualizoval předepsaný projekt s~novými předepsanými vlastnostmi.
\item[Odhlášení a~přihlášení na testovací účet:] Uživatel ve svém nově vytvořeném účtě nebude mít dostatek klientů, projektů a~časových záznamů, které by odpovídaly dlouhodobějšímu používání aplikace. Pro lepší otestování orientace v~aplikaci bude uživatel instruován, aby se přihlásil na předepsaný testovací účet, který obsahuje větší množství záznamů.
\item[Odstranění časového záznamu:] Uživatel bude instruován, aby odstranil předepsaný časový záznam z~historie.
\item[Export historie do CSV souboru:] Uživatel bude instruován, aby vytvořil integraci pro exportování do CSV souboru a~exportoval data do tabulky pro předepsané časové období. Dále bude instruován, aby tuto tabulku otevřel ve vhodné aplikaci, kde si ji může prohlédnout (např. \emph{Numbers}).
\item[Odstranění klienta:] Budeme předpokládat, že si uživatel v~exportované tabulce všimne, že tam má záznamy patřící předepsanému klientovi, které tam mít nechce. Bude tedy instruován, aby klienta smazal a~znovu vyexportoval CSV soubor, ve kterém zkontroluje, že žádné záznamy patřící k~tomuto klientovi nejsou.
\item[Tvorba automatizace pro spuštění časovače:] Uživatel dostane instrukci, aby pomocí aplikace \emph{Zkratky} vytvořil novou automatizaci, která zapne časovač s~předepsanými parametry, pokud se uživatel objeví na předepsané poloze. Bude také instruován, aby tuto automatizaci zkusil ručně spustit a~v~aplikaci zkontroloval, že časovač opravu běží.
\item[Tvorba automatizace pro vypnutí časovače (odstraněno):] Uživatel bude požádán, aby vytvořil automatizaci, která se ho zeptá, zda nechce zastavit běžící časovač, pokud opustí předepsanou polohu. Zde bude také instruován, aby automatizaci zkusil ručně spustit. Tento scénář byl po několika testech ze seznamu odstraněn, protože bylo zhodnoceno, že je příliš podobný předchozímu scénáři a~zároveň přímo netestuje rozhraní samotné aplikace.
\end{description}

%---------------------------------------------------------------
\subsection{Výsledky}\label{ui-testing-results}
%---------------------------------------------------------------

V~protokolu z~průběhu testování (příloha \ref{appendix:ui-testing-protocol}) byly označeny takzvané \emph{kritické} a~\emph{důležité poznatky}. Kritické poznatky jsou takové, u~kterých bylo zhodnoceno, že se jedná o~závažnou chybu rozhraní aplikace, která může uživateli závažným způsobem zhoršit zkušenost s~jejím používáním. Důležité poznatky jsou potom takové, které nemusí být nutně nějakou chybou v~rozhraní, ale potenciálním podnětem pro zlepšení rozhraní. Tato sekce rozebírá a~analyzuje tyto dva druhy poznatků. Následující sekce postupně rozebírají kritické a~důležité poznatky, seřazené podle toho, jaká jim byla přidělena priorita (první má největší prioritu).

%---------------------------------------------------------------
\subsubsection{Kritické poznatky}
%---------------------------------------------------------------

Kritické poznatky byly v~průběhu testování zaznamenány hned u~prvního testera (tester F. W.). Vzhledem k~jejich závažnosti byly chyby objevené těmito poznatky opraveny okamžitě. Z~tohoto důvodu se už u~dalších testerů tyto stejné chyby objevit nemohly.

\begin{itemize}
\item\textbf{Opakované ukazování neznámé chyby během registrace.} Testerovi F. W. se po zadání údajů pro registraci nedařilo registraci dokončit, protože po klikání na tlačítko \emph{Registrovat} se neustále ukazovala \emph{Neznámá chyba}. Tato chybová hláška neposkytovala žádný popis toho, co konkrétně by mohlo být špatně, ani návrh na to, jak chybu opravit. Tester musel opakovaně zkoušet vyplňovat e-mail v~jiných formátech. Nakonec se mu přihlášení podařilo a~hádal, že chyby asi byly kvůli špatnému formátu e-mailu nebo slabému heslu.
\end{itemize}

%---------------------------------------------------------------
\subsubsection{Důležité poznatky}
%---------------------------------------------------------------

\begin{itemize}
\item\textbf{Obtížná klikatelnost prvků v~navigační liště.} Většina testerů musela opakovaně klikat na tlačítka \emph{Uložit} nebo \emph{Exportovat} v~navigační liště, protože se jim do nich nedařilo trefit.
\item\textbf{Redundantní potvrzení po výběru klienta nebo projektu.} Někteří testeři zmínili, že se jim zdá redundantní klikat na tlačítko \emph{Uložit} po výběru projektu pro ovladač časovače, nebo po výběru klienta pro projekt. Očekávali, že po zvolení projektu nebo klienta se volba uloží automaticky a~aplikace se vrátí o~obrazovku zpět.
\item\textbf{Absence možnosti úpravy časového záznamu.} Několik testerů se při různých fázích scénáře snažilo klikat nebo podržet na časový záznam v~historii, pravděpodobně s~očekáváním, že to bude mít nějakou odezvu, jako otevření detailu. Aplikace ale nic takového neimplementuje, pouze swipe-to-delete.
\item\textbf{Počáteční zmatení nad přepínáním ovladače mezi časovač a~manuální zadávání.} Někteří testeři byli zpočátku zmateni nad rozhraním ovladače pro přepínání mezi klasickým měřením pomocí časovače a~mezi manuálním zadáváním času nového záznamu. Jeden tester zmínil, že by pro manuální zadání času preferoval nějaký dialog místo in-place změny rozhraní časovače, ale poté zmínil, že přepínání nakonec pochopil a~že jde možná o~zvyk.
\item\textbf{Absence swipe-to-delete u~některých seznamů – nekonzistence v~rámci aplikace.} Jeden tester zmínil, že byl zmatený z~toho, že při pokusu o~smazání klienta nefungovalo gesto swipe-to-delete, které fungovalo při mazání časových záznamů, a~používání tohoto gesta tak bylo v~rámci aplikace nekonzistentní.
\item\textbf{Klávesnice při zadávání e-mailové adresy nenabízí v~základním rozmístění kláves tečku.} Při zadávání e-mailové adresy se zobrazí klávesnice, která ve svém výchozím rozmístění neobsahuje klávesu pro tečku, ale ani například klávesu pro zavináč. Uživatel tak musí dodatečně klikat na klávesu pro zobrazení speciálních znaků.
\item\textbf{Malá klikatelná plocha tlačítka pro smazání.} Jeden tester si všiml, že pro kliknutí na tlačítko pro mazání je potřeba kliknout na text tlačítka, nefunguje kliknutí pouze na pozadí tlačítka.
\item\textbf{Neviditelný kurzor při zobrazení zadaného hesla.} Při zadávání hesla aplikace nabízí možnost zobrazit přímo text hesla, které uživatel zadal, místo bezpečnostního zakrytí tečkami. Při tomto zobrazení ale z~pole zmizí kurzor, přestože do pole lze normálně psát i~nadále.
\item\textbf{Jednotlivé karty aplikace si pamatují zanoření.} Někteří testeři byli zmatení nad tím, že při přepnutí do karty profilu byli zanořeni v~obrazovce, do které se proklikli při jednom z~předchozích scénářů. Očekávali, že kliknutí na karty profilu je dostane do přehledu profilu.
\item\textbf{Absence onboardingu nebo možnosti vytvoření počátečních klientů a~projektů.} Pokud účet uživatele neobsahuje žádná data, zpravidla po registraci, tak sice zobrazuje informaci, že uživatel například nemá žádné projekty a~že si je může vytvořit v~profilu, ale nenabízí nějaký jednoduchý proklik, který by toto počáteční zadávání dat usnadnil. Někteří testeři zmínili, že by aplikace mohla obsahovat nějakou formu onboardingu, což je obvykle nějaká úvodní nápověda, v~rámci které si uživatel může například právě vytvořit nějaká počáteční data.
\item\textbf{Nedostatečná vizuální indikace běžícího časovače.} Jeden tester zmínil, že by při běžícím časovači přidal tlačítku pro zastavení červené pozadí, jelikož se jedná o~zvyk u~existujících řešení pro měření a~správu času.
\item\textbf{Ne příliš výstižný popis tlačítka Uložit.} Jeden tester zmínil, že při vytváření nového klienta nebo projektu by upřednostnil textaci tlačítka jako \emph{Vytvořit} místo \emph{Uložit}.
\item\textbf{Ne příliš výstižný popis tlačítka Smazat.} Jeden tester zmínil, že v~potvrzovacím dialogu pro smazání klienta (tedy i~projektu) by očekával text Smazat místo Ano.
\item\textbf{Přednost popisu záznamu před názvem klienta nebo projektu.} Dva testeři zmínili, že sem jim zdá, že popis časového záznamu je důležitější, než název projektu nebo klienta.
\item\textbf{Absence potvrzení před smazáním časového záznamu.} Dva testeři zmínili, že je překvapilo, že pro mazání časového záznamu nebyl žádný potvrzovací dialog.
\item\textbf{Údiv nad zařazením exportu do CSV mezi integrace.} Dva testeři zmínili, že se jim zdá zvláštní, že je export do CSV souboru zařazen mezi integrace.
\item\textbf{Aplikace se po exportu dat nevrací o~obrazovku zpět.} Jeden tester zmínil, že by po úspěšném exportu očekával, že se aplikace vrátí o~obrazovku zpět.
\item\textbf{Absence indikace spuštěné zkratky v~aplikaci.} Dva testeři zmínili, že by po spuštění zkratky očekávali, že se v~aplikaci objeví nějaká indikace, že byla zkratka provedena.
\item\textbf{Při snaze o~úpravu zadaného hesla se maže celé pole.} Někteří testeři byli podráždění tím, že když potřebovali upravit zadané heslo, tak se při pokusu o~smazání jednoho znaku smazalo celé pole.
\item\textbf{Zapnutí časovače v~aplikaci Zkratky nemá automaticky otevřený dialog pro parametry.} Jeden tester zmínil, že při tvorbě zkratky by čekal, že nabídka parametrů bude otevřená automaticky.
\end{itemize}

%---------------------------------------------------------------
\subsection{Zhodnocení}
%---------------------------------------------------------------

Tato sekce rozebírá poznatky z~uživatelského testování, diskutuje nad tím, čím byly způsobeny a~jakým způsobem by případně mělo na jejich základě být upraveno rozhraní aplikace.

Co se týče kritických poznatků, který byl jen jeden, tak jak již bylo zmíněno, byly opraveny ihned po provedení uživatelského testování s~prvním testerem. Jednalo se o~chybu rozhraní, které nezobrazovalo dostatečný popis chyby během registrace. Pokud uživatel zadal nevalidní formát e-mailové adresy nebo příliš slabé heslo, tak se tato chyba uživateli zobrazila pouze jako \emph{Neznámá chyba}, z~čehož by uživatel měl jen velmi nízkou šanci pochopit, v~čem udělal chybu. Tato chyba byla způsobena tím, že aplikace Trackee v~produkčním prostředí neposkytuje technické popisy chyb pro neznámé druhy chyb, tedy takové, které nejsou explicitně detekovány a~není jim přidělena vlastní lokalizovaná textace. To přesně se dělo u~těchto chyb, kdy z~knihovny \emph{Firebase} autentizace přišla nějaká chyba, která sice obsahovala nějaký popis chyby z~knihovny, ale ten aplikace v~produkčním prostředí neukazovala a~z~chyby se tak stala \emph{Neznámá chyba}. Byla tedy přidána explicitní detekce chyb pro špatný formát e-mailu, pro příliš slabé heslo a~pro pokus o~registraci s~již existujícím e-mailem.

Důležité poznatky byly zhodnoceny následovně:

\begin{itemize}
\item\textbf{Obtížná klikatelnost prvků v~navigační liště:} U~této chyby se lze do jisté míry odvolat na systémové rozhraní iOS, protože způsob, kterým jsou v~aplikaci přidávány tlačítka do navigační lišty, přesně dodržuje způsob navržený Applem a~jedná se o~způsob, který sám Apple ve svých aplikacích používá. Do navigační lišty je zkrátka nutné kliknout přesně, jinak aplikace dotek zaregistruje mimo tlačítko. Ale vzhledem k~tomu, že se snad žádnému testerovi nepodařilo kliknout na všechny navigační tlačítka na první pokus, tak se jedná o~tak závažný problém, že by se mu budoucí vývoj měl určitě věnovat i~tak. Mohlo by být vyvinuto nějaké úsilí na pokus rozšíření klikatelné plochy v~navigační liště, a~pokud by tento postup byl při použití systémové navigační lišty obtížný, mohly by být potvrzovací tlačítka z~navigační lišty zcela odstraněny a~nahrazeny primárním tlačítkem přímo vespod obsahu dané obrazovky.
\item\textbf{Redundantní potvrzení po výběru klienta nebo projektu:} Někteří testeři přímo zmínili, že jim toto přijde zbytečné. A~vzhledem k~tomu, že se nejedná o~žádnou nevratnou akci, kterou by uživatel nemohl například při špatné volbě ihned změnit, tak není důvod odmítat změnu rozhraní, ve kterém se při volbě klienta nebo projektu aplikace opravdu automaticky vrátí o~obrazovku zpět a~volbu sama uloží. Ve všech případech, kde se tento přístup používá, se lze na výběr ihned vrátit.
\item\textbf{Absence možnosti úpravy časového záznamu:} Během různých úmyslů se testeři snažili na záznam kliknout, ale nic se nestalo. Možnost vytvoření nějakého rozhraní pro úpravu záznamu je tedy určitě podstatným podnětem pro další vývoj aplikace.
\item\textbf{Počáteční zmatení nad přepínáním ovladače mezi časovač a~manuální zadávání:} Rozhraní ovladače, konkrétně přepínaní mezi stavem časovače a~stavem manuálního zadávání, bylo inspirováno ovladačem časovače v~aplikaci \emph{Clockify} \cite{clockify-ios}. Uživatelům umožňuje ovládat časovač různými způsoby na jednom místě. Uživatelé zmínili, že jakmile způsob přepínání pochopili, tak s~rozhraním problém neměli. Je tedy na pováženou, zda se smířit s~tím, že ovladač časovače vyžaduje nějaké pochopení, které není zcela intuitivní, ale jakmile k~pochopení dojde, umožní snadnější tvorbu záznamů. Protože například použití dialogu pro manuální tvorbu záznamu, jak navrhl jeden tester, by mohlo způsobit, že takový dialog překryje část viditelného rozhraní historie a~tím uživateli znemožní se inspirovat předchozími záznamy, ověřit si, že na sebe časy navazují, a~podobně.
\item\textbf{Absence swipe-to-delete u~některých seznamů – nekonzistence v~rámci aplikace:} Poznámka testera, že to způsobuje nekonzistenci v~rámci aplikace, byla zcela namístě. Podnětem pro další vývoj aplikace by mělo být sjednocení intearkčních prvků ve všech seznamech, aby všude byl implementován detail i~swipe-to-delete, a~všechny se tak chovaly stejným způsobem.
\item\textbf{Klávesnice při zadávání e-mailové adresy nenabízí v~základním rozmístění kláves tečku:} Systém iOS umožňuje různým textovým polím přidělovat různé typy klávesnic, u~pole pro e-mailovou adresu by tohoto tedy mělo být využito, aby se zobrazila klávesnice se specializovaným rozmístěním kláves pro zadávání e-mailu.
\item\textbf{Malá klikatelná plocha tlačítka pro smazání:} U~obrazovek, kde vznikl tento problém, se používá nativní UI komponenta \emph{List}, která pracuje s~tlačítky v~řádcích takovým způsobem, že detekuje kliknutí na ně pouze tehdy, pokud se klikne na text tlačítka, alespoň v~základním použití. Podnětem dalšího vývoje by proto mělo být zkoumání, jak tento problém vyřešit. Buď snahou o~rozšíření klikatelné plochy, nebo nahrazení vlastním primárním tlačítkem mimo nativní seznam.
\item\textbf{Neviditelný kurzor při zobrazení zadaného hesla:} Aplikace používá vlastní řešení pro přepínání mezi viditelným a~neviditelným stavem pole pro hesla, které odstraňuje chybu nativního řešení, které při přepnutí viditelnosti zruší \emph{focus} na pole, což způsobí zmizení klávesnice a~uživatel na pole musí kliknout znovu. Vedlejším produktem tohoto řešení je ale zřejmě tento problém, je tedy potřeba použité řešení zrevidovat a~pokusit se tento problém odstranit.
\item\textbf{Jednotlivé karty aplikace si pamatují zanoření:} U~tohoto poznatku bylo překvapující, že vznikl i~u~uživatelů systému iOS, ve kterém je toto chování při používání několika navigačních karet ve spodní liště zcela standardní. Je opět na pováženou, jestli se snažit toto chování měnit a~určovat tento poznatek za určující, když se jedná o~standardní chování.
\item\textbf{Absence onboardingu nebo možnosti vytvoření počátečních klientů a~projektů:} Prázdné stavy jednotlivých obrazovek by mohly být mimo samotný informující text rozšířeny o~tlačítko pro~přidání prvku, které je jinak pouze v~navigační liště. Implementace úvodní nápovědy poté může být podnětem pro další vývoj aplikace.
\item\textbf{Nedostatečná vizuální indikace běžícího časovače:} Tester, který tento poznatek vznesl, zmínil, že je z~ostatních řešení zvyklý, že při běžícím časovači má ovládací tlačítko červené pozadí, aby dostatečně vizuálně odlišilo stav běžícího časovače. Jestliže se jedná o~běžnou praxi u~měřičů odpracovaného času, není důvod se této praxi vyhýbat a~červené pozadí nepřidat také.
\item\textbf{Ne příliš výstižný popis tlačítka Uložit:} Navigační tlačítko \emph{Uložit} může být v~případě vytváření nového klienta nebo projektu změněno na \emph{Vytvořit}. Tento text je výstižnější a~není důvod ho nepoužít.
\item\textbf{Ne příliš výstižný popis tlačítka Smazat:} Tlačítko \emph{Ano} může být nahrazeno textací \emph{Smazat}, která bude pravděpodobně více intuitivní.
\item\textbf{Přednost popisu záznamu před názvem klienta nebo projektu:} První tester, který tento podnět vznesl, nazval popis jako \emph{title} – jeho význam tedy asi pochopil jako název či nadpis záznamu. Popis ale v~záznamech poskytuje až sekundární roli, je nepovinný. Projekt je naopak u~záznamu povinný. Tento názor ale později vyjádřil i~další tester. Mělo by být tedy zváženo, zda v~rozhraní neupřednostnit popis záznamu před projektem, přestože vůbec nemusí být přítomen.
\item\textbf{Absence potvrzení před smazáním časového záznamu:} Testeři, kteří tento podnět vznesli, měli pravdu v~tom, že se jedná o~destruktivní akci a~pokud ji uživatel udělá omylem, přijde o~data a~nemá možnost je získat zpět. Přidání explicitního dialogu se ale zdá být zase příliš, protože pokud by uživatel chtěl smazat více záznamů, musel by u~každého záznamu potvrzovat dialog, a~zas tolik dat se pod jedním záznamem neskrývá. Ideálním řešením by tedy mohla být přidaná možnost vrátit akci mazání. Po smazání záznamu, které by bylo učiněno stejným způsobem jako do teď, by se mohl zobrazit \emph{Toast} s~informací, že záznam byl smazán, a~s~tlačítkem \emph{Vrátit}, které by akci vrátilo a~smazaný záznam vrátilo zpět.
\item\textbf{Údiv nad zařazením exportu do CSV mezi integrace:} Dva testeři vyjádřili údiv nad tím, že export do CSV je zařazen mezi integrace. Je pravda, že samotný soubor CSV sám o~sobě integraci netvoří. CSV soubor je ale integračním prvkem, který umožňuje integraci v~podstatě s~čímkoli – s~řadou dalších systémů, s~nějakými vizualizačními nástroji, a~podobně. A~vzhledem k~tomu, že ho uživatel právě za účelem integrace bude pravděpodobně využívat, tak má určitě své místo na kartě integrací. Ostatně, oddělit ho od této karty a~hledat mu jiné místo by pravděpodobně nedávalo smysl.
\item\textbf{Aplikace se po exportu dat nevrací o~obrazovku zpět:} Tento podnět vznesl jeden tester. Bylo zhodnoceno, že toto chování by bylo očekávatelné v~případě, kdyby místo obrazovky sloužil pro export dat nějaký typ dialogu. Ale protože se jedná o~standardní obrazovku, jejíž přístup uživatelům umožní například exportovat více časových období za sebou, nebo opětovné exportování se stejnými parametry (které bylo dokonce předmětem jednoho scénáře), bylo rozhodnuto, že automatické vrácení zpět, které by způsobilo ztrátu dat na této obrazovce, by být implementováno nemělo.
\item\textbf{Absence indikace spuštěné zkratky v~aplikaci:} Tento podnět zmínili dva testeři, kteří při návratu do aplikace po spuštění zkratky neviděl žádnou explicitní indikaci toho, že byla zkratka provedena. Zkratky implementované v~aplikaci ale nemají záměr uživatele po jejich provedení přesměrovávat do aplikace, k~tomu slouží jiné, \emph{otevírací} typy zkratek. Zkratky aplikace Trackee slouží primárně pro automatizaci, uživatel nebo automatizace tedy zkratku spustí, aplikace vykoná úkony a~uživatel vůbec nemá potřebu aplikaci otevírat. Tento podnět tedy byl pravděpodobně vznesen kvůli tomu, že instrukce scénáře testerovi přímo říkala, aby po spuštění zkratky zkontroloval, že se akce v~aplikaci provedla, ale toto uživatelé ve skutečnosti vůbec dělat nemusí. Jedná se tedy o~lehce zavádějící instrukci testovacího scénáře, která testery instruuje k~tomu, aby udělali něco, co není klasickým případem užití. Úmyslem tohoto scénáře bylo spíše to, aby testeři pochopili, co vlastně zkratka udělala.
\item\textbf{Při snaze o~úpravu zadaného hesla se maže celé pole:} Několik testerů toto chování podráždilo, ale jedná se o~standardní chování textových polí pro citlivé údaje. Pokud uživatel přestane s~úpravou takového pole, později s~úpravou zase začne a~pokusí se smazat jeden znak, smaže se celý text pole. Jedná se o~bezpečností prvek, není tedy vhodné se toto chování snažit měnit.
\item\textbf{Zapnutí časovače v~aplikaci Zkratky nemá automaticky otevřený dialog pro parametry:} V~dokumentaci pro \emph{App Intents} \cite{ios-app-intents} nebyla nalezena zmínka o~tom, že by existovala nějaká možnost, jak aplikaci \emph{Zkratky} donutit, aby ve výchozím stavu otevřela dialog pro parametry. Toto chování tedy aplikace nemůže ovlivnit.
\end{itemize}

































































%---------------------------------------------------------------
\chapter{Prostředí pro budoucí podporu a~provoz aplikace}
%---------------------------------------------------------------

Během tvorby této práce vzniklo mnoho podnětů pro budoucí vývoj a vylepšení aplikace. Už v~samotném návrhu nebo realizaci byla rozebírána řada možností, které v~této práci realizovány nejsou, ale bylo by je vhodné v~budoucnu implementovat. Během uživatelského testování pak vznikla řada doporučení pro úpravu rozhraní aplikace.

%---------------------------------------------------------------
\section{Zrychlení komunikace a lokální data}
%---------------------------------------------------------------

Prvním tématem, kterému by se budoucí provoz aplikace měl zabývat, je celkové zrychlení manipulace s~aplikací. Jsou dvě zásadní věci, které zpomalují interakci s~aplikací a které vyžadují vylepšení.

%---------------------------------------------------------------
\subsection{Komunikace s~databází}
%---------------------------------------------------------------

Pomalá komunikace s~databází je způsobena hlavně tím, že zvolená řešení pro nasazení backendu a databáze provozují jejich instance navzájem vzdálené 8 000 km od sebe. Důvody, proč tomu tak je, byly popsány v~sekci \ref{development-tools}. Nejlepším řešením tohoto problému by bylo používat takové nástroje, které minimalizují vzdálenost mezi backendem a databází, která ale budou pravděpodobně vyžadovat placené plány. Mezi backendem a databází ale probíhá největší počet požadavků na čtení či zápis, jejich vzájemná komunikace je proto ještě důležitější, než komunikace mezi backendem a samotným klientem.

%---------------------------------------------------------------
\subsection{Lokálních data (cache)}
%---------------------------------------------------------------

Moderní aplikace obvykle využívají nějakou formu lokální \emph{cache}, která může mít mnoho využití, jako například dočasné zastoupení vzdálených dat z~databáze. Vhodné navržení a implementace takové \emph{cache} je ale poměrně komplexní záležitost a proto nemohla být realizována v~rámci této práce. Lokální \emph{cache} by šlo využít mimo jiné takto:
\begin{itemize}
\item{\emph{UseCases}, které vrací různé seznamy dat (záznamy, klienty, projekty, \dots), by místo prostého vrácení jedné hodnoty mohly vracet nějaký \emph{stream} dat (napříkad \emph{KotlinX Flow} \cite{kotlinx-flow}), které by nejprve vrátily data z~\emph{cache} lokální databáze, a hned jak by přišly data ze vzdálené databáze, tak by se data nahradily těmito daty a také by byla aktualizována \emph{cache}. Tímto způsobem by v~podstatě zmizely prodlevy, ve kterých musí uživatel čekat, než se mu zobrazí data a může interagovat s~aplikací. Je však potřeba v~jednotlivých případech počítat s~tím, kdy aplikace nějaká data aktualizuje – tehdy je potřeba na vzdálená data počkat.}
\item{Funkcionality, které nějakým způsobem aktualizují data (ovládání časovače, uložení záznamu, tvorba/úprava klienta nebo projektu, \dots), by nemusely po aplikování změn čekat na to, až aktualizovaná data přijdou, ale mohly by rovnou zobrazit data v~takovém formátu, v~jakém by aplikace předpokládala, že aktualizovaná data budou vypadat (například při vytvoření klienta se klient rovnou zobrazí v~seznamu klientů a až po chvíli jeho přidání potvrdí vzdálená data). Bylo by však třeba v~případě, že aktualizace dat selže, vrátit data do původního stavu, před očekávanou aktualizací.}
\end{itemize}

Implementace aplikace je pro využití lokální \emph{cache} dobře připravena. V~multi-platformní části již obsahuje nástroje pro manipulaci s~lokální databází pomocí knihovny \emph{SQLDelight} \cite{sqldelight}. Lze také v~implementaci nahlédnout, že v~multi-platformní části se jednotlivé \emph{sources} jmenují například \texttt{RemoteIntegrationSource}, což představuje zdroj vzdálených dat (z~backendu) pro integraci. Při použití lokální databáze by se definoval \texttt{LocalIntegrationSource}, který by implementoval obdobné funkcionality, a logika toho, kdy a jak se má který typ \emph{source} použít, by byla obsažena v~konkrétní \emph{repository}, v~uvedeném příkladě \texttt{IntegrationRepository}.

%---------------------------------------------------------------
\section{Úpravy uživatelského rozhraní}
%---------------------------------------------------------------

Z~průběhu uživatelského testování vzešla řada poznatků, ať už velmi důležitých nebo méně důležitých, které poukázaly na nějaké nedostatky či možnosti vylepšení v~uživatelském rozhraní aplikace. Důkladný popis, analýza a doporučení pro další vývoj aplikace, vycházejících z~těchto poznatků, je v~sekci \ref{ui-testing-results}, která by měla sloužit jako podnět budoucího vývoje.

%---------------------------------------------------------------
\section{Rozšíření funkcionalit}
%---------------------------------------------------------------

Funkcionality, které aplikace implementuje, mají také široký potenciál, jak rozšířit jejich možnosti a přinést tak uživatelům další funkce při používání aplikace. Mezi tyto možnosti může patřit:
\begin{itemize}
\item\textbf{Možnost úpravy časového záznamu.} Tato funkce by uživatelům ulehčila práci se záznamy, pokud například špatně zadají nějaké parametry záznamu. Také se jednalo o~jeden z~poznatků uživatelského testování – zájem ze strany uživatelů o~tuto funkci tedy určitě je.
\item\textbf{Více parametrů pro klienty.} U~klientů lze v~realizované aplikaci zadávat pouze název, pro různé budoucí účely se ale mohou hodit další parametry, jako třeba fakturační adresa klienta, různé poznámky, nebo barevné rozlišení.
\item\textbf{Více parametrů pro projekty.} U~projektů se také v~budoucnu mohou hodit další parametry, než ty, které aplikace v~tuto chvíli nabízí. Mezi ty může například patřit cena práce za hodinu u~konkrétního projektu, kterou lze využít během fakturace, nebo možnost vytvořit si vlastní typy projektů. 
\item\textbf{Reporty a shrnutí.} Mnoho existujících řešení pro měření a správu odpracovaného času umožňuje vytvářet nějaké formy reportů nebo vizualizací odpracovaného času. Uživatelé si mohou prohlížet, kolik práce vykonali v~určitých obdobích u~různých klientů a projektů, a podobně. Aplikace tuto funkci v~současné chvíli nemá a tvorbu takových shrnutí umožňuje pouze pomocí jiných nástrojů, například z~CSV dat, která aplikace poskytne.
\item\textbf{Integrace s~dalšími existujícími systémy.} V~analýze v~sekci \ref{existing-tracking-solutions} byly popsány různé existující systémy pro měření a správu odpracovaného času, spolu s~možnostmi integrace s~nimi. Realizace aplikace implementuje základ pro tyto integrace (CSV, \emph{Clockify}), ale potenciál těchto integrací je mnohem větší. Předmětem dalšího vývoje by tedy mohla být nějaká analýza toho, jaké další integrace by bylo vhodné implementovat a tyto integrace poté realizovat.
\item\textbf{Integrace s~hardwarovými spouštěči.} V~analýze byly také popsány možnosti hardwarových spouštěčů (fyzické ovladače apod.), které by šly využít při integraci s~aplikací. Pro propojení s~jedním z~nich by bylo potřeba implementovat BLE komunikaci. Zvážení této funkce také může být předmětem dalšího vývoje.
\item\textbf{Přihlášení přes externí poskytovatele.} \emph{Firebase} autentizace, které aplikace využívá, nabízí možnosti přihlášení přes \emph{Apple}, \emph{Facebook} a \emph{Google}. V~aplikaci je dokonce připraven \emph{provider} pro přihlášení přes \emph{Apple} – \texttt{AppleSignInProvider}.
\end{itemize}

%---------------------------------------------------------------
\section{Nasazení mezi reálné uživatele}
%---------------------------------------------------------------

V~realizaci v~sekci \ref{testflight} bylo popsáno, jakým způsobem byla aplikace nasazována pro účely testování během vývoje. Podobným způsobem se poté realizuje i~nasazení mezi reálné uživatele v~obchodě \emph{App Store}. Pro nahrání aplikace do tohoto obchodu také slouží nástroj \emph{App Store Connect}.

Jak již bylo také zmíněno, aplikace musí před schválením pro nasazení do obchodu \emph{App Store} projít procesem \emph{App Review}, ve kterém \emph{Apple} kontroluje splnění všech pravidel pro aplikace, které do obchodu míří. Aplikace už ale tímto procesem úspěšně prošla už v~rámci veřejného testování. \emph{App Review} pro \emph{App Store} je sice o~něco pečlivější než pro veřejné testování, ale zásadní porušení pravidel pro šíření aplikací v~\emph{App Store} by mělo být odhaleno již v~tomto procesu.

Ostatní nástroje, které jsou aplikací využívány, pak žádnou další konfiguraci pro nasazení mezi reálné uživatele nepotřebují. Backend i~databáze jsou nasazeny pod veřejně přístupnými doménami a aplikace je na ně napojena. Bude pouze potřeba monitorovat využití backendu nebo databáze, které jsou sice prozatím pod neplacenými plány, ale v~budoucnu budou pravděpodobně zapotřebí placené plány. Také bude podle využití potřeba zhodnotit, zda se stále vyplatí využívat funkce pro uspávání backendu, kterou \emph{Railway} nabízí.

%---------------------------------------------------------------
\section{Podpora pro další platformy}
%---------------------------------------------------------------

Návrh architektury platformy (sekce \ref{platform-architecture}) počítal s~možnostmi budoucího rozšíření podpory pro další platformy, jako je Android aplikace, webová aplikace a další. Toto rozšíření může být implementováno s~pomocí technologie \emph{Kotlin Multiplatform}, kterou klientská aplikace používá. Stačí přidat moduly pro nové platformy a jejich aplikace implementovat podobným stylem, jako nativní iOS aplikaci. Detailnější popis struktury celého projektu v~rámci \emph{monorepa} je v~realizaci v~sekci \ref{project-structure}.








































%---------------------------------------------------------------
\chapter*{Závěr}
%---------------------------------------------------------------
\addcontentsline{toc}{chapter}{Závěr}
\markboth{Závěr}{Závěr}

Hlavním cílem práce bylo analyzovat, navrhnout a~implementovat mobilní aplikaci pro systém iOS (frontend i~backend) pro zaznamenávání odpracovaného času, která měla umožnit integraci s~různými spouštěči akcí a~propojení s~dalšími systémy pro zaznamenávání a~spravování odpracovaného času. V~rámci návrhu bylo cílem navrhnout vhodné technické řešení pro implementaci aplikace, funkcionality aplikace a~jejich uživatelské rozhraní a~tohoto návrhu se poté při realizaci aplikace držet. Následně bylo cílem práce realizované řešení vhodně otestovat a~provést uživatelské testování. 

V~rámci analýzy se podařilo poskytnout přehled o~tom, co je problematika měření a~správy odpracovaného času, jaké jsou možnosti propojení s~různými spouštěči akcí, jaké jsou možnosti integrace s~existujícími systémy, které řeší stejnou problematiku, a~o~tom, jaká jsou specifika vývoje pro mobilní platformu iOS. Návrh implementace poté navrhl technická řešení a~přístupy pro realizaci, funkcionality aplikace a~jejich uživatelské rozhraní, čehož se poté držela realizace aplikace. Výsledkem poté byla plně funkční mobilní aplikace, která umožňuje některé formy integrace se spouštěči a~s~dalšími systémy.

Realizovaná aplikace implementuje funkcionality aplikace pro měření a~správu odpracovaného času – umožňuje měření času pomocí časovače nebo pomocí manuálního zadání a~udržuje historii časových záznamů, ke kterým může přidělovat projekt a~popis. Z~hlediska integrace aplikace implementuje základní funkce, jako propojení se systémovými zkratkami, které mají teoreticky neomezené možnosti pro tvorbu automatizací, nebo možnost exportu historie záznamů do CSV souboru, což umožňuje import historie do mnoha existujících systémů.

Vedlejším cílem práce byla flexibilita a~rozšiřovatelnost technické implementace aplikace. Během návrhu a~realizace celé platformy byl kladen důraz na její přehlednost a~rozšiřovatelnost, která je díky použití technologií jako \emph{Kotlin Multiplatform}, nebo díky implementaci vlastního backendu, dobře připravena na rozšíření nejen o~další cílové platformy (Android, Web a~další), ale i~na rozšíření o~další funkcionality, jako možnosti integrace s~dalšími systémy.

Velkým přínosem pro budoucí vývoj aplikace jsou také výsledky uživatelského testování, kterého se zúčastnilo 5 testerů. Většina testerů měla větší zkušenost s~používáním mobilních aplikací pro měření a~správu odpracovaného času, a~poskytli tak relevantní zkušenost cílového uživatele s~používáním aplikace. Z~testování vzešla řada poznatků, které byly popsány, analyzovány a~ze kterých vznikla doporučení, jak by kvůli nim měla být aplikace upravena.

Na závěr aplikace navrhuje prostředí pro budoucí provoz a~podporu aplikace, které navrhuje řadu možností pro budoucí vylepšení a~rozšíření, úpravy rozhraní a~další. Dále také popisuje, jakým způsobem by bylo možné aplikaci finálně nasadit mezi reálné uživatele, nebo jakým způsobem aplikaci rozšířit o~podporu pro další platformy.

Práce také může poskytnout určitou formu inspirace nebo seznámení pro čtenáře, kteří se zajímají o~různé přístupy k~možnostem vývoje mobilních aplikací, nebo o~konkrétní technologie, kterých tato práce v~rámci realizace využila.



































