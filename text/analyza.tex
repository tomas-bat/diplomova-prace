Tato kapitola rozebírá, k~čemu slouží zaznamenávání odpracovaného času. Poté popisuje možnosti, které umožňují měření času spouštět a také se věnuje příkladům řešení, která již pro zaznamenávání odpracovaného času existují. V dalších sekcích se poté obecně zaměřuje na vývoj aplikací pro mobilní platformu iOS včetně možností multiplatformního a cross-platformního vývoje. Na závěr poté zkoumá, jaké existují možnosti pro integraci spouštěčů měření odpracovaného času a exportu dat do dalších aplikací.

%---------------------------------------------------------------
\section{Zaznamenávání odpracovaného času}
%---------------------------------------------------------------

Zaznamenávání odpracovaného času může být potřeba z~několika různých důvodů, buď z~pohledu jednotlivce, nebo z~pohledu spolupráce v~nějakém týmu. Mezi tyto důvody může patřit například:
\begin{itemize}
\item\textbf{Sledování pracovních hodin:} Pomáhá zaměstnancům a podnikům sledovat, kolik času strávili na různých úkolech, projektů a aktivitách během pracovní doby. To může být užitečné pro sledování produktivity, hodnocení efektivity práce a plánování rozpočtu času.
\item\textbf{Fakturace a účtování:} Pro profesionály a firmy, které účtují za své služby na základě odpracovaného času, umožňuje zaznamenávání snadné sledování času stráveného na jednotlivých projektech a klientech pro účely fakturace.
\item\textbf{Analýza produktivity:} Poskytuje data a statistiky o~tom, jaký čas je věnován různým úkolům a projektům. To umožňuje identifikovat trendy v~pracovních návycích, optimalizovat časové plánování a zlepšit efektivitu práce.
\item\textbf{Správa projektů:} Pomáhá organizovat časové údaje spojené s~různými projekty a úkoly, což usnadňuje plánování, delegování a monitorování pokroku.
\item\textbf{Transparentnost a komunikace:} Pro týmy umožňuje transparentně sdílet informace o~čase stráveném na různých aktivitách, což podporuje spolupráci a komunikaci v rámci týmu.
\end{itemize}

Celkově zaznamenávání slouží k~lepšímu řízení času, sledování produktivity a optimalizaci využití pracovního času pro jednotlivce i~organizace.

%---------------------------------------------------------------
\section{Spouštěče měření času}
%---------------------------------------------------------------

Jedním z~cílů této práce je prozkoumat různá řešení pro spouštěče měření času. Nejjednodušším spouštěčem je samozřejmě ruční zapnutí nějaké formy časovače v~samotné aplikaci, která pro měření času slouží. To ale umí kdejaké již existující řešení a přidanou hodnotou této práce by měly být nějaké možnosti, jak spouštění uživateli ulehčit. Tyto možnosti se dají rozdělit do dvou kategorií – fyzické a softwarové.

%---------------------------------------------------------------
\subsection{Fyzické}
%---------------------------------------------------------------

Za fyzické spouštěče měření času lze považovat jakoukoli formu fyzického ovladače, kterou může uživatel nějak ovládat a tím spouštět časovač měření času.

