Tato kapitola rozebírá, k~čemu slouží zaznamenávání odpracovaného času. Poté popisuje možnosti, jak měření času spouštět a~také se věnuje příkladům řešení, která již pro zaznamenávání odpracovaného času existují, a~jaké jsou možnosti propojení s~těmito řešeními. V~dalších sekcích se poté obecně zaměřuje na vývoj aplikací pro mobilní platformu iOS včetně možností multi-platformního a~cross-platformního vývoje.

%---------------------------------------------------------------
\section{Zaznamenávání odpracovaného času}
%---------------------------------------------------------------

Zaznamenávání odpracovaného času může být potřeba z~několika různých důvodů, buď z~pohledu jednotlivce, nebo z~pohledu spolupráce v~nějakém týmu. Mezi tyto důvody může patřit například:
\begin{itemize}
\item\textbf{Sledování pracovních hodin:} Pomáhá zaměstnancům a~podnikům sledovat, kolik času strávili na jednotlivých úkolech, projektech a~aktivitách během pracovní doby. To může být užitečné pro sledování produktivity, hodnocení efektivity práce a~plánování rozpočtu času.
\item\textbf{Fakturace a~účtování:} Pro profesionály a~firmy, které účtují za své služby na základě odpracovaného času, umožňuje zaznamenávání snadné sledování času stráveného na určitých projektech a~klientech pro účely fakturace.
\item\textbf{Analýza produktivity:} Poskytuje data a~statistiky o~tom, jaký čas je věnován různým úkolům a~projektům. To umožňuje identifikovat trendy v~pracovních návycích, optimalizovat časové plánování a~zlepšit efektivitu práce.
\item\textbf{Správa projektů:} Pomáhá organizovat časové údaje spojené s~různými projekty a~úkoly, což usnadňuje plánování, delegování a~monitorování pokroku.
\item\textbf{Transparentnost a~komunikace:} Pro týmy umožňuje transparentně sdílet informace o~čase stráveném na různých aktivitách, což podporuje spolupráci a~komunikaci v~rámci týmu.
\end{itemize}

Celkově zaznamenávání slouží k~lepšímu řízení času, sledování produktivity a~optimalizaci využití pracovního času pro jednotlivce i~organizace.

\begin{figure}[h]
	\centering
	\includegraphics[width=10cm]{clockify-ios.png}
	\caption{Clockify – iOS aplikace pro měření času \cite{clockify-ios}}
	\label{fig:clockify-ios}
\end{figure}

%---------------------------------------------------------------
\section{Spouštěče měření času}\label{tracking-triggers}
%---------------------------------------------------------------

Jedním z~cílů této práce je prozkoumat různá řešení pro spouštěče měření času. Nejjednodušším spouštěčem je samozřejmě ruční zapnutí nějaké formy časovače v~samotné aplikaci, která pro měření času slouží. To ale umí kdejaké již existující řešení a~přidanou hodnotou této práce by měly být možnosti, jak spouštění uživateli ulehčit. Tyto možnosti se dají rozdělit do dvou kategorií – fyzické a~softwarové.

%---------------------------------------------------------------
\subsection{Fyzické}
%---------------------------------------------------------------

Za fyzické spouštěče měření času lze považovat jakoukoli formu fyzického ovladače, kterou může uživatel ovládat a~tím spouštět časovač měření času.

Častým typem fyzického spouštěče je nějaká forma takzvaného platónského tělesa (krychle, osmistěn, dvanáctistěn, atd.), které může uživatel otáčet. Podle toho, na kterou stranu ho položí, se spustí časovač s~danými parametry. Různé strany tělesa mohou sloužit například pro identifikování toho, na kterém projektu uživatel zrovna pracuje.

Mezi fyzické spouštěče patří například tyto produkty:
\begin{itemize}
\item\textbf{TIMEFLIP:} Dvanáctistěnné těleso, na které si uživatel může nalepovat cokoli, co bude identifikovat odpracovaný čas. Těleso se propojí s~mobilní aplikací, která poté spravuje naměřený čas. \cite{timeflip}
\item\textbf{TIMEULAR:} Stejný princip jako \emph{TIMEFLIP}, akorát používá osmistěnné těleso. Jednotlivé strany jdou také přizpůsobovat nálepkami. \cite{timeular}
\item\textbf{timeBuzzer:} Fyzické tlačítko, které lze umístit na stůl a~zapojit do počítače. Stisknutí tlačítka otevře okno měřící aplikace, otočením tlačítka lze vybrat projekt a~opětovným stisknutím se začne daný projekt měřit. \cite{timebuzzer}
\end{itemize}

\begin{figure}[h]
	\centering
	\begin{subfigure}[b]{5.2cm}
		\includegraphics[width=5.2cm]{timeflip.jpeg}
		\caption{TIMEFLIP – Kostka a~aplikace \cite{timeflip}}
		\label{pic:timeflip}
	\end{subfigure}
	\hspace{2cm}
	\begin{subfigure}[b]{7cm}
		\includegraphics[width=7cm]{timebuzzer.jpeg}
		\caption{timeBuzzer – tlačítko a~aplikace \cite{timebuzzer}}
		\label{pic:timebuzzer}
	\end{subfigure}
	\caption{TIMEFLIP a~timeBuzzer}
\end{figure}

Všechny ze zmíněných produktů poskytují veřejné API pro aplikace, se kterými komunikují \cite{timeflip-api} \cite{timeular-api} \cite{timebuzzer-api}. Pokud by tedy bylo potřeba produkty propojit s~vlastní aplikací, která by mohla naměřený čas spravovat, bylo by potřeba, aby aplikace komunikovala s~těmito nástroji. Pořád by byl potřeba prostředník, kterým by byla aplikace daného produktu. Výjimkou je v~těchto produktech pouze \emph{TIMEFLIP}, který poskytuje veřejný protokol pro BLE komunikaci \cite{timeflip-ble-api}. Tento produkt by tedy mohl komunikovat s~novou aplikací napřímo.

Pokud by nová aplikace měla umožňovat budoucí propojení s~jakýmkoli dalším hardwarovým produktem, tak by také měla poskytovat veřejné API, které by takovou komunikaci umožňovalo. Vývoj takových produktů by mohl být předmětem návrhu pro budoucí vylepšení aplikace.

%---------------------------------------------------------------
\subsection{Softwarové}\label{software-tracking-triggers}
%---------------------------------------------------------------

Za softwarové spouštěče měření času lze považovat určité formy integrace a~automatizace v~rámci zařízení, na kterém aplikace běží. V~případě aplikace na platformě iOS to mohou být následující možnosti:
\begin{itemize}
\item\textbf{Automatizace podle polohy:} Aplikace by mohla reagovat na polohu uživatele a~spouštět nebo vypínat časovač na základě informace, kde se uživatel nachází. Uživatel by si mohl například nastavit určitá místa, kde by chtěl, aby se automaticky spustil časovač. Mohl by se zapnout přímo s~konkrétními přednastavenými parametry (projekt, klient, \dots), nebo by se jen mohlo ukázat upozornění, aby si uživatel časovač zapnul sám, s~parametry, které zadá. Podobně by aplikace mohla reagovat i~na opuštění nějakého místa – například pokud uživatel opustí místo, které má označeno jako práci, tak by dostal upozornění, zda si nechce vypnout časovač.
\item\textbf{Automatizace podle času:} Jednoduchá forma automatizace by mohla fungovat na základě času. V~přednastavených časech by se mohl zapnout nebo vypnout časovač s~danými parametry, nebo by aplikace mohla uživatele jen upozornit.
\item\textbf{Automatizace podle kalendáře:} Užitečná by také mohla být integrace s~kalendářem uživatele. Podle naplánovaných událostí by se časovač mohl automaticky zapínat, přepínat, nebo vypínat, s~parametry z~kalendáře. Nebo opět pouze uživatele upozornit, zda si kvůli nějaké události nechce měření času aktualizovat.
\item\textbf{Automatizace podle režimu soustředění:} Systém iOS umožňuje od verze 15.0 uživatelům používat takzvané \emph{Režimy soustředění} \cite{ios-focus-modes}. Uživatel si může konfigurovat různé režimy a~nastavovat, jaké mu v~tomto režimu budou chodit upozornění, jaké aplikace může používat, kdo mu může volat, apod. Od iOS verze 16 také Apple umožňuje vývojářům přizpůsobovat chování aplikace podle toho, jaký režim soustředění je zapnutý \cite{ios-focus-modes-adjustment}. Toho by se dalo vhodně využít například tak, že by se mohl automaticky spouštět časovač (nebo by uživatel mohl být upozorněn) podle toho, do jakého režimu soustředění se uživatel právě přepnul.
\item\textbf{Automatizace pomocí zkratek:} Apple nabízí uživatelům s~iOS verze 13.0 nebo novější aplikaci \emph{Zkratky} \cite{ios-shortcuts-app}, která umožňuje tvorbu vlastních automatizujících procesů \cite{ios-shortcuts}. Tento způsob automatizace umožňuje uživatelům i~vývojářům velkou míru přizpůsobitelnosti – zkratky lze propojovat s~dalšími aplikacemi, s~hlasovým asistentem a~s~mnoha různými akcemi. Všechny předchozí způsoby automatizace lze také nějakým způsobem vytvořit pomocí zkratek. \cite{ios-shortcuts-developer}
\end{itemize}

%---------------------------------------------------------------
\section{Existující systémy pro měření a~správu odpracovaného času}\label{existing-tracking-solutions}
%---------------------------------------------------------------

Systémů pro měření a~správu odpracovaného času existuje mnoho. Existuje dokonce několik různých článků o~tom, které systémy jsou nejlepší a~jaké je jejich srovnání \cite{forbes-tracking-apps-article} \cite{zapier-tracking-apps-article}. Následující výběr příkladů je učiněn podle těchto článků, podle osobní preference a~podle počtu stažení na iOS platformě. U~každého příkladu jsou také rozebrány možnosti importu dat, od kterých se mohou odvíjet požadavky na export dat z~nové aplikace.

%---------------------------------------------------------------
\subsection{Clockify}
%---------------------------------------------------------------

Clockify je jednoduchá a~rozšířená aplikace primárně určená pro měření a~správu odpracovaného času. Mobilní aplikace má více než 1 milion stažení \cite{clockify-app-magic} a~celý systém používají miliony lidí \cite{clockify-customers}. 

V článku Forbes je hodnocena jako celkově nejlepší aplikace pro měření času pro rok 2024. Má široké spektrum funkcionalit – umožňuje měření a~správu času pro různé projekty, klienty a~zařízení. Mimo měření času nabízí také sledování prezence pro mzdy a~účetnictví, optimalizaci produktivity zaměstnanců, měření vykazovatelného času a~sdílení pokroku na projektech s~klienty. \cite{forbes-tracking-apps-article}

Clockify nabízí aplikace pro mnoho platforem: Desktopovou aplikaci pro Mac, Windows a~Linux, doplněk webového prohlížeče pro Chrome, Firefox a~Edge, mobilní aplikaci pro iOS a~Android a~nakonec sdílené Kioskové řešení, na kterém si uživatelé mohou zapisovat příchody, přestávky a~další. \cite{clockify-apps}

V~základní variantě je používání aplikace zdarma. Clockify ale nabízí i~placené plány, které nabízí přidané funkce, jako audity naměřených časů, používání přestávek a~další. Tyto placené plány mohou stát od \$3.99 měsíčně za uživatele až po \$11.99 měsíčně za uživatele. \cite{clockify-pricing} 

\begin{figure}[p]
	\centering
	\includegraphics[width=0.9\textwidth]{clockify-timer.jpg}
	\caption{Časovač na měření času v~aplikaci Clockify \cite{clockify-features}}
	\label{fig:clockify-timer}
	\vspace{1cm}
	\includegraphics[width=0.9\textwidth]{clockify-report.jpg}
	\caption{Reporty v~aplikaci Clockify \cite{clockify-features}}
	\label{fig:clockify-report}
\end{figure}

Clockify nabízí dobře zdokumentované API pro své služby \cite{clockify-api}. Napojení na funkce systému lze tedy jednoduše udělat i~z~vlastního řešení. Do Clockify lze také data importovat z~CSV tabulek, jsou-li data vhodně naformátována \cite{clockify-import-timesheets}.

Mezi nevýhody systému Clockify Forbes uvádí, že projekty nejdou označit jako dokončené, a~že shrnující reporty mohou být zpočátku matoucí. Celkově Clockify ale vnímá jako nejlepší řešení pro pokrytí celé škály funkcionalit.

Na obrázku \ref{fig:clockify-timer} je vidět hlavní obrazovka ve webové verzi aplikace – časovač. Uživatel si zde může časovač ručně zapnout a~vypnout, přidělit projekt, přidat značky, označit vykazovatelnost a~přidat popisek. Také má možnost přidat odpracovaný čas ručně, tedy ne pomocí časovače. K~tomu stačí jen napsat čas začátku a~konce. Pod ovládacím panelem časovače potom uživatel vidí seznam svých již zadaných časových záznamů, které zde může libovolně upravovat.

Dále je na obrázku \ref{fig:clockify-report} vidět obrazovka určená pro reporty. Uživatel si zde může vybrat časový úsek, pro který chce report vypracovat, a~může si nastavit filtry, podle kterých chce časové vstupy filtrovat (tým, klient, projekt, úkol, značka, stav a~popisek). Po aplikování filtru uživatel následně uvidí report své práce vyhovující zadaným parametrům, který si může exportovat ve formátech PDF, CSV a~Excel.

%---------------------------------------------------------------
\subsection{Toggl Track}
%---------------------------------------------------------------

Toggl track je další poměrně rozšířená aplikace pro měření a~správu času. Na platformě iOS má více než 1 milion stažení \cite{toggl-track-app-magic}. 

Forbes tuto aplikaci považuje za nejlepší pro malé týmy, protože nabízí plán zdarma pro týmy do 5 uživatelů a~nabízí neomezený počet klientů a~projektů. \cite{forbes-tracking-apps-article}

Aplikace nabízí určitou formu automatické detekce toho, že uživatel nemá zapnutý časovač, i~když pracuje, nebo automatickou detekci naopak toho, že uživatel už nepracuje, ale časovač má zapnutý. Aplikace nabízí detailní denní, týdenní a~měsíční reporty.

Toggl Track také pokrývá více platforem: Webovou aplikací, mobilní aplikací pro iOS a~Android a~desktopovou aplikací pro Windows a~Mac. \cite{toggl-track}

\begin{figure}[h]
	\centering
	\includegraphics[width=\textwidth]{toggl-track.png}
	\caption{Aplikace Toggl Track \cite{toggl-track}}
\end{figure}

Toggl Track také nabízí možnost importu dat z~CSV tabulky, ve velmi podobném formátu jako Clockify \cite{toggl-track-import-csv}. a~stejně jako Clockify nabízí vlastní otevřené API \cite{toggl-track-api}.

%---------------------------------------------------------------
\subsection{Deputy}\label{deputy}
%---------------------------------------------------------------

Deputy je další aplikací, která umožňuje měření a~správu odpracovaného času. Jejím hlavním účelem je však plánování směn zaměstnanců. Je uvedena jako jeden z~příkladů proto, aby šlo nahlédnout na problém měření času i~z~jiného úhlu, než striktně měření času primárně za účelem vykazování, fakturování nebo analýzy efektivity. Je také velice rozšířenou aplikací – počty stažení na mobilních telefonech překračují 5 milionů \cite{deputy-app-magic}.

Právě plánování směn zaměstnanců je hlavním důvodem, proč Forbes tuto aplikaci ve svém seznamu zmiňuje. Aplikace totiž nabízí plánování neomezeného počtu směn za měsíc. Uživatelé se mohou přihlásit k~aplikacím pro plánování a~pro prezenci zvlášť. I~neplacený plán umožňuje automatické plánování směn, což šetří čas manažerům. Toto plánování umožňuje i~započítání obědových pauz, přestávek a~dalšího, přímo do plánu směn, podle příslušných zákonů.  \cite{forbes-tracking-apps-article}

Deputy nabízí aplikaci pro měření odpracovaného času pro mobilní platformy iOS a~Android. 

\begin{figure}[h]
	\centering
	\includegraphics[width=\textwidth]{deputy.png}
	\caption{Aplikace Deputy \cite{deputy-time-tracking-app}}
\end{figure}

Stejně jako předchozí systémy, Deputy nabízí možnost importu dat z~CSV tabulky \cite{deputy-import-csv}. Vzhledem k~trochu jinému byznysovému modelu této aplikace ale tyto data musí na rozdíl od předchozích systémů obsahovat dodatečná data, jako místo, pauzy na oběd, a~další. a~ani u~této aplikace nechybí otevřené API \cite{deputy-api}.

%---------------------------------------------------------------
\section{Vývoj mobilních aplikací pro systém iOS}
%---------------------------------------------------------------

Tato sekce se věnuje několika klíčovým tématům souvisejícím s~vývojem aplikací pro zařízení s~operačním systémem iOS. Tato témata shrnuje a~poskytuje tím základ pro diskuzi o~vývoji aplikací pro tento systém.

%---------------------------------------------------------------
\subsection{Historie a~vývoj iOS platformy}
%---------------------------------------------------------------

Historie operačního systému iOS začala v~roce 2007, kdy byl představen a~začal se prodávat první mobilní telefon od společnosti Apple, kterým byl iPhone. V~tomto telefonu byl od výroby nainstalován operační systém, který ještě v~tu dobu nenesl název \emph{iOS}, ale \emph{iPhone OS}. Název \emph{iOS} se začal používat až později od verze systému 4. Přestože z~dnešního pohledu chybělo v~systému mnoho funkcí, se kterými si dnes mobilní telefony spojujeme (například možnost instalace aplikací třetích stran), tak v~tu dobu to byl velký posun. Věci jako \emph{Multitouch screen}, vizuální znázornění hlasové schránky, nebo integrace s~\emph{iTunes}, byly ohromnou výhodou. Mezi předinstalované aplikace patřil kalendář, fotky, fotoaparát, poznámky, prohlížeč webu \emph{Safari}, e-mailový klient, telefon a~\emph{iPod} (který se později rozdělil na aplikace pro hudbu a~aplikace pro videa).

\begin{figure}[h]
	\centering
	\includegraphics[width=10cm]{iphone.png}
	\caption{První model mobilního telefonu iPhone \cite{iphone-review}}
\end{figure}

O~rok později, v~roce 2008, byla představena nová generace mobilního telefonu s~názvem iPhone~3G, se kterým také vznikla nová verze operačního systému iOS (iPhone OS) 2.0. Hlavní novinkou této nové verze byl nový obchod s~aplikacemi třetích stran, \emph{App Store}. Až 500 aplikací bylo v~obchodě k~dispozici v~době, kdy byl zpřístupněn uživatelům.

V~roce 2009 poté s~novým iPhonem 3GS přišla verze iOS (iPhone OS) 3, která přinesla hlavně vylepšení, jako možnost používat kopírování a~vkládání, vyhledávání přes \emph{Spotlight}, podporu MMS a~možnost natáčet videa (do té doby iPhone uměl pouze pořizovat fotky). Také to byla první verze operačního systému, která fungovala i~na nových tabletech od společnosti Apple – na iPadech. První iPad byl představen v~roce 2010.

V~roce 2010 byl představen nový iPhone 4 a~s~ním verze operačního systému 4. Od této verze se začal používat název \emph{iOS}, který nahradil do té doby používaný \emph{iPhone OS}. Toto nové pojmenování mělo sjednocovat názvosloví pro více zařízení – iPhone, iPad a~iPod. iPhone 4 byl poměrně velkou změnou, protože disponoval novým hranatým designem, a~systém iOS 4 přinesl novinky, které tento operační systém pomalu začaly rýsovat do dnešní podoby. Mezi novinky patřil \emph{FaceTime}, \emph{multitasking}, \emph{iBooks}, organizování aplikací do složek, osobní hotspot nebo sdílení dat pomocí \emph{AirPlay} a~\emph{AirPrint}. Také to byla první verze iOS, která nepodporovala všechny dosud vydané iPhony, protože nebyla kompatibilní s~prvním iPhonem.

O~rok později, s~iPhonem 4S, byl představen iOS verze 5. Apple v~ní reagoval na rostoucí trend bezdrátovosti a~\emph{cloud computing} představením několika nových funkcí a~platforem. Mezi ně patřil třeba iCloud nebo synchronizace s~iTunes přes Wi-Fi. Také vznikla platforma \emph{iMessage} a~začalo se používat oznamovací centrum.

Dalším modelem byl iPhone 5, který poprvé změnil velikost displeje a~zvětšil ji z~3,5 palce na 4. S~tímto modelem přišla verze iOS 6, kterou doprovázelo mnoho problémů. V~této verzi Apple představil hlasového asistenta \emph{Siri}, který i~přes to, že ho později předčila konkurence, byl poměrně zásadní novinkou. Dále Apple v~nové verzi představil nové aplikace pro mapy, na které do té doby používal řešení od firmy Google. Tyto mapy od začátku obsahovaly mnoho chyb a~celkově byly považovány za dost nedokončené, což způsobilo ve firmě mnoho problémů.

V~roce 2013, s~novým modelem iPhone 5S, přišla verze iOS 7. Tato verze zásadním způsobem změnila vzhled uživatelského rozhraní, které se na začátku od uživatelů netěšilo příliš dobrému přijetí, ale po několika vylepšeních a~po tom, co si uživatelé na nový vzhled zvykli, problémy ustaly. Mezi nové funkce této verze se řadí sdílení dat s~dalšími uživateli přes \emph{AirDrop}, používání operačního systému v~displejích automobilů přes \emph{CarPlay}, ovládací centrum, nebo nový způsob odemykání zařízení pomocí otisku prstu – \emph{Touch ID}.

\begin{figure}[h]
	\centering
	\includegraphics[width=10cm]{ios7.jpg}
	\caption{Nový vzhled iOS verze 7 \cite{ios-7-design}}
\end{figure}

S~dalším rokem a~novým modelem iPhone 6 přišla verze iOS 8. iPhone 6 opět zásadně změnil svůj vzhled, ale v~operačním systému se moc zásadních změn nedělo. Apple se v~této verzi soustředil hlavně na nové funkce. Mezi ně se řadí \emph{Apple Music}, placení přes \emph{Apple Pay}, cloudové úložiště \emph{iCloud Drive}, sdílení práce mezi Apple zařízeními přes \emph{Handoff}, rodinné sdílení a~další.

Rok 2015 přinesl iPhone 6S a~iOS 9. V~této verzi se Apple soustředil hlavně na optimalizaci a~stabilitu, příliš nových funkcí představeno nebylo – pouze nasvícení pro noční používání \emph{Night shift}, režim nízké spotřeby a~možnost používat veřejný testovací beta program.

Dalším modelem byl v~roce 2016 iPhone 7 s~iOS~10. Opět nepřinesl moc nových funkcí, mimo například možnosti mazat původní nainstalované aplikace.

O~rok později poté s~iPhonem~8 přišel iOS~11. V~této verzi se Apple soustředil převážně na nové funkcionality pro iPad, které se jeho použití snažily přiblížit k~plnohodnotnému počítači. Nově přibyly funkce jako podpora tužky \emph{Apple Pencil}, podpora více otevřených oken a~pracovních ploch nebo prohlížeč souborů. Apple ale na závěr představil ještě další nový iPhone – iPhone X. Jednalo se o~další velký redesign, kdy se výrazně zvětšila plocha displeje, zmizelo přední tlačítko a~vzniklo ověření uživatele pomocí obličeje – \emph{Face ID}.

V~roce 2018 Apple představil iPhone~XS a~levnější variantu XR, se kterými přišel iOS~12, kde se jednalo opět jen o~nepatrná vylepšení. 

S~dalším rokem přišel iPhone~11 a~iOS~13. Tento systém už se rozdělil – pro zařízení iPad nyní vznikl oddělený \emph{iPadOS}. V~iOS~13 Apple představil možnost tmavého režimu pro celý systém, nové možnosti bezpečnosti a~soukromí a~s~tím související možnost přihlášení přes Apple.

V~roce 2020 s~iPhonem~12 a~iOS~14 Apple přidal několik menších změn a~vylepšení, jako \emph{Widgety} na domovské obrazovce.

Podobně tomu bylo i~v~roce 2021, kdy Apple s~novým iPhonem~13 a~iOS~15 přidával velké množství menších vylepšení, primárně do vlastních aplikací.

S~rokem 2022 přišel iPhone~14 a~iOS~16, který přinesl sadu vylepšení a~nových funkcí pro zamčenou obrazovku, spolu s~dalšími vylepšeními.

Poslední znatelný update iOS přišel v~roce 2023 s~iPhonem~15 a~iOS~17. a~jak už bylo v~posledních letech zvykem, tak tato aktualizace přinášela větší množství menších vylepšení a~aktualizací, které ale už nijak zásadně neměnily systém a~interakci s~ním. \cite{history-of-ios}

Za těchto 17 let vývoje se operační systém iOS dostal do stavu, kdy ho v~současnou chvíli používá 1,46 miliard aktivních uživatelů. Mobilních telefonů iPhone se od roku 2007 prodalo 2,3 miliard. \cite{iphone-user-statistics}

%---------------------------------------------------------------
\subsection{Vývojové nástroje a~prostředí}\label{ios-dev-tools}
%---------------------------------------------------------------

Jak již bylo zmíněno v~předchozí sekci, obchod s~aplikacemi \emph{App Store} byl uživatelům dostupný od roku 2008. Ve stejnou dobu také Apple zpřístupnil vývojářům iPhone SDK ve svém vývojovém prostředí \emph{Xcode}. Toto vývojové prostředí od Applu již existovalo od roku 2003, kdy vzniklo jako rebranding původního vývojového prostředí \emph{Project Builder}. S~příchodem App Storu Apple umožnil všem vývojářům v~Xcodu vyvíjet aplikace pro iPhone OS od verze 2.0 pomocí zmíněného iPhone SDK. 

Vývojové prostředí Xcode je jednotné integrované vývojové prostředí, které slouží pro vývoj aplikací pro všechny platformy společnosti Apple. Lze stáhnout zdarma z~obchodu \emph{Mac App Store} nebo ze stránek společnosti Apple. Instalace vývojového prostředí obsahuje i~instalaci dalších pomocných nástrojů, jako je simulátor zařízení (iPhone, iPad, \dots), nástroje příkazové řádky, a~další. \cite{xcode-history}

Pro nativní vývoj iOS aplikací se dříve používal programovací jazyk \emph{Objective-C} \cite{objc} a~pro tvorbu uživatelského rozhraní knihovna \emph{UIKit}. V~roce 2014 Apple přidal podporu svého nového programovacího jazyku \emph{Swift} \cite{swift} a~v~roce 2019 přidal podporu nové deklarativní knihovny pro tvorbu uživatelského rozhraní \emph{SwiftUI} \cite{swiftui}.

%---------------------------------------------------------------
\subsection{Architektura aplikací}\label{app-architecture}
%---------------------------------------------------------------

Jako u~každého softwarového projektu, architektura je důležitou součástí jeho návrhu. K~architekturám mobilních aplikací existuje mnoho přístupů. Například u~Android aplikací sám Google doporučuje, jaké architektonické vzory by měly být používány a~doporučuje architektury pro tyto aplikace \cite{android-app-arch}. Apple přímo žádné architektury pro aplikace cílící na iOS platformu nedoporučuje. 

Dokumentace pro iOS platformu ale obsahuje mnoho ukázek kódu věnujících se konkrétním tématům (například \cite{swift-ui-tutorial-complex-interfaces}), kde je vždy nějaký náznak architektury připraven. V~těchto ukázkách se vyskytuje architektura \emph{Model-View-Controller} (zkráceně MVC). Na rozdíl od tradiční MVC architektury se ta používaná Applem mírně liší, jelikož tradiční MVC architektura není moc dobře aplikovatelná na moderní iOS vývoj. Jak lze vidět na obrázcích \ref{fig:mvc}, v~tradičním MVC by mělo \emph{View} být beze stavu a~měl by ho pouze překreslovat \emph{View Controller}. Toto je sice možné v~iOS aplikaci implementovat, ale nedává to moc smysl, protože všechny tyto 3 entity jsou hluboce provázané, což dramaticky snižuje jejich opětovné použití. Při použití v~iOS aplikaci je \emph{Controller} mediátorem mezi \emph{View} a~\emph{Model}, kteří o~sobě navzájem nic nevědí. Jelikož je ale \emph{View Controller} zpravidla velice provázán s~\emph{View}, tak vzniká potřeba psát masivní \emph{View Controller}. Proto se v~iOS vývoji často referuje na MVC jako na \emph{Massive View Controller}.

\begin{figure}[h]
	\centering
	\begin{subfigure}[b]{0.35\textwidth}
		\centering
		\includegraphics[width=6cm]{traditional-mvc.png}
		\caption{Tradiční MVC architektura}
	\end{subfigure}
	\hspace{1cm}
	\begin{subfigure}[b]{0.45\textwidth}
		\centering
		\includegraphics[width=7cm]{apple-mvc.png}
		\caption{MVC architektura v~iOS aplikacích}
	\end{subfigure}
	\caption{Architektura Model-View-Controller \cite{ios-architecture-patterns}}
	\label{fig:mvc}
\end{figure}

Architektura MVC je ale v~ukázkách kódu od Applu používána pravděpodobně převážně proto, že se hodí pro výstižné a~krátké ukázky kódu – obsahuje totiž minimální \emph{overhead}. Pro potřeby větších projektů začíná být nedostačující. Nelze nijak testovat logiku prezentování, a~\emph{View} také nelze pomocí jednotkových testů otestovat. Jediné, co lze testovat, je \emph{Model}. a~velkou nevýhodou je právě zmíněný masivní \emph{View Controller}, který pro složitější obrazovky ztrácí přehlednost.

Velmi rozšířenou architekturou v~iOS vývoji je \emph{Model-View-ViewModel} (zkráceně MVVM). \emph{View} a~\emph{Model} zastávají stejnou funkci jako v~MVC, a~mediátorem je zde \emph{View Model}. \emph{View} je v~tomto případě konkrétní obrazovka a/nebo \emph{View Controller}. \emph{View Model} je nezávislý na knihovně uživatelského rozhraní, což umožňuje jeho lepší testovatelnost. Drží stav obrazovky, vyvolává změny pro \emph{Model} a~aktualizuje se podle něj. Vizuální reprezentace architektury MVVM lze nahlédnout v~obrázku \ref{fig:mvvm}.

\begin{figure}[h]
	\centering
	\includegraphics[width=10cm]{mvvm.png}
	\caption{Architektura MVVM \cite{ios-architecture-patterns}}
	\label{fig:mvvm}
\end{figure}

Další rozšířenou architekturou v~iOS aplikacích je \emph{Viper}. Tato architektura nepochází ze skupiny MV(X). \emph{Viper} se snaží vzít rozdělovaní odpovědností o~krok dále – obsahuje 5 vrstev:
\begin{itemize}
\item\textbf{View:} Zastává stejnou funkcionalitu jako v~MV(X), tedy \emph{View} a/nebo \emph{View Controller}.
\item\textbf{Interactor:} Obsahuje byznysovou logiku související s~daty, jako tvorba instancí nebo načítání dat ze serveru.
\item\textbf{Presenter:} Obsahuje byznysovou logiku UI, ale je nezávislý na UI knihovně. Volá metody \emph{Interactoru}.
\item\textbf{Entities:} Čisté objekty dat.
\item\textbf{Router:} Odpovědný za přechody mezi \emph{Viper} moduly.
\end{itemize}
Tato architektura je poměrně volná v~tom, co bude konkrétní \emph{Viper} modul reprezentovat. Může to být jedna samotná obrazovka, ale může to být celá část aplikace. Lze si všimnout několika rozdílů od architektur ze skupiny MV(X):
\begin{itemize}
\item Model (interakce s~daty) je přesunut do \emph{Interactoru} pomocí \emph{Entities}.
\item Pouze povinnosti UI reprezentace \emph{Controlleru}/\emph{View Modelu} je přesunuta do \emph{Presenteru}, ale ne byznysová logika s~daty.
\item \emph{Viper} explicitně adresuje odpovědnost navigace, kterou řeší \emph{Router}.
\end{itemize}
Vizualizace architektury \emph{Viper} lze nahlédnout v~obrázku \ref{fig:viper}. \cite{ios-architecture-patterns}

\begin{figure}[h]
	\centering
	\includegraphics[width=12cm]{viper.png}
	\caption{Architektura \emph{Viper} \cite{ios-architecture-patterns}}
	\label{fig:viper}
\end{figure}

Existuje také rozšíření architektury MVVM o~navigační logiku, takzvaná MVVM-C, kde písmeno C představuje \emph{Flow Coordinator}. Tato vrstva je odpovědná za navigační tok, vytváří instance \emph{View} a/nebo \emph{View Controller} a~prezentuje je, předává data mezi těmito instancemi a~obsluhuje akce uživatele. Struktura \emph{Flow Coordinatoru} je vhodně navržena tak, aby se dala také dobře testovat. \cite{ios-mvvm-c}

Jednotlivé architektury mají také své aktualizované varianty pro nově používanou deklarativní UI knihovnu \emph{SwiftUI}. Všechny ze zmíněných architektur se dají vhodně použít i~s~touto knihovnou. Ve \emph{SwiftUI} už se jednotlivé obrazovky nerozdělují na \emph{View} a~\emph{View Controller}, ale zůstává zde pouze \emph{View}. \emph{SwiftUI} ale ze své podstaty jako deklarativní knihovna přináší nové možnosti, jak k~architektuře aplikace přistupovat.

Jednou z~novějších architektur, které využívají výhody deklarativních UI knihoven, jako je \emph{SwiftUI}, je \emph{Model-View-Intent} (zkráceně MVI). \emph{View} a~\emph{Model} reprezentují stejné vrstvy jako doposud a~\emph{Intent} reprezentuje nějaký úmysl vyvolat určitou akci. Tím může být například kliknutí uživatele. Konkrétní \emph{Intent} poté pomocí nějaké byznysové logiky aktualizuje stav, podle kterého se následně aktualizuje \emph{View}. \cite{swiftui-mvi}

Všechny výše popsané architektury pracují s~daty pomocí nějakého \emph{Modelu} (\emph{Viper} pomocí \emph{Interactoru}). Práce s~daty ale obvykle není jednoduchá logika, která by si zasloužila pouze zmínku o~tom, že se o~ní stará nějaký \emph{Model}. Může se totiž jednat o~poměrně komplikovanou doménovou a~byznysovou logiku, stahování dat ze serveru, cache dat, práce s~databází, a~další. V~tomto ohledu je namístě diskutovat o~nějaké obecnější architektuře aplikace, ne pouze o~prezentační logice, tedy primárně o~logice uživatelského rozhraní. Obecným zvykem je aplikace rozdělovat do třívrstvé architektury. Tyto vrstvy se mohou nazývat různě, ale obvykle jde o~následující:
\begin{itemize}
\item\textbf{Prezentační vrstva:} Logika uživatelského rozhraní, interakce s~uživatelem, obrazovky, \dots
\item\textbf{Doménová/byznysová vrstva:} Byznysová a~doménová logika.
\item\textbf{Datová vrstva:} Práce s~daty (komunikace se serverem, databáze, \dots)
\end{itemize}
Rozšířená architektura, vhodná pro iOS aplikace, která definuje strukturu aplikace, je \emph{Clean Architecture}. Tato architektura je navržena pro dobrou testovatelnost, rozdělení odpovědnosti, a~vyhovění dalším zašlým návrhovým vzorům týkajících se návrhu softwaru. V~této architektuře se jednotlivé vrstvy skládají z~následujících částí:
\begin{itemize}
\item\textbf{Prezentační vrstva:} \emph{View}, \emph{View Controllery}, \emph{View Modely}, a~další, podle dané architektury (MVC, MVVM, Viper, \dots). Tato vrstva má závislost na doménové vrstvě a~volá její \emph{Use Cases}.
\item\textbf{Doménová vrstva:} Rozhraní \emph{Use Cases} a~jejich implementace, rozhraní \emph{Repositories}, které \emph{Use Cases} používají a~doménové objekty. \emph{Use Cases} obsahují byznysovou logiku aplikace a~datovou logiku nechávají na \emph{Repositories}.
\item\textbf{Datová vrstva:} Implementace \emph{Repositories}, rozhraní \emph{Providers} a~implementace \emph{Providers}. Implementace repozitářů jsou nezávislé na konkrétních knihovnách třetích stran, obsahují logiku s~daty. Implementace \emph{Providers} už jsou závislé na konkrétních knihovnách. Jednotlivé \emph{Providers} se mohou starat například o~práci s~konkrétní databází nebo s~konkrétní knihovnou pro serverovou komunikaci.
\end{itemize}
V~\emph{Clean Architecture} se také pracuje s~\emph{Dependency Injection}, což řeší logiku toho, které konkrétní implementace budou dosazeny jednotlivým rozhraním (\emph{Use Cases, Repositories, Providers}). \cite{ios-clean-arch-mvvm}

%---------------------------------------------------------------
\section{Cross-platformní a~multi-platformní možnosti}\label{crossplatform-multiplatform}
%---------------------------------------------------------------

V~současném prostředí mobilního vývoje se rozhodnutí ohledně vhodné platformy pro tvorbu aplikací stává klíčovým faktorem pro úspěch projektu. Existuje široká škála přístupů, které vývojáři mohou zvolit, od tradičních nativních řešení až po moderní multi-platformní a~cross-platformní frameworky. Tato sekce se zaměřuje na analýzu těchto možností s~ohledem na tvorbu mobilní aplikace pro zaznamenávání odpracovaného času. Zhodnocuje jejich výhody, nevýhody a~vhodnost pro konkrétní aplikaci. Tento přístup umožní lépe pochopit, jakým směrem se vydat při návrhu a~implementaci projektu.

Pro rozdíl mezi cross-platformním a~multi-platformním vývojem neexistuje přesná definice a~v~různých zdrojích lze nalézt různé interpretace (např. \cite{cross-multiplatform-alternative-intepretation-one} a~\cite{cross-multiplatform-alternative-intepretation-two}). V~kontextu vývoje pro mobilní aplikace si budeme tyto pojmy vykládat následovně:
\begin{itemize}
\item Cross-platformní vývoj představuje řešení, ve kterých sdílený kód běží na koncových platformách přímo přes nějakou formu abstrakce nebo interpretace.
\item Multi-platformní vývoj představuje řešení, ve kterých se sdílený kód přímo kompiluje do nativního kódu specifického pro každou platformu.
\end{itemize}

%---------------------------------------------------------------
\subsection{Nativní vývoj}
%---------------------------------------------------------------

Nativní vývoj pro platformu iOS znamená vytváření mobilních aplikací přímo v~jazyce \emph{Swift} nebo \emph{Objective-C} s~využitím oficiálních nástrojů poskytovaných společností Apple, jako je Xcode IDE a~iOS SDK. Tento přístup umožňuje vytvořit aplikaci, která je optimalizovaná pro konkrétní operační systém a~využívá všech funkcí a~výhod, které iOS nabízí.

Výhody nativního vývoje pro iOS spočívají především v~plné integraci s~ekosystémem Apple, což zajišťuje vysokou kvalitu, rychlost a~stabilitu aplikací. Vývojáři mají přístup ke kompletní sadě nástrojů, dokumentace a~podpory přímo od výrobce platformy, což usnadňuje vývoj a~řešení potíží. Díky nativnímu přístupu je možné vytvářet aplikace s~vysokou výkonností a~možnostmi, které jsou na míru prostředí iOS.

Nevýhody nativního vývoje spočívají v~tom, že vyžaduje znalost specifických jazyků a~nástrojů pro každou platformu (\emph{Swift}/\emph{Objective-C} pro iOS, \emph{Kotlin}/\emph{Java} pro Android), což může zvýšit náklady na vývoj a~časovou náročnost. Navíc nativní přístup vyžaduje oddělený vývoj pro každou platformu, což může být neefektivní pro projekty s~omezeným rozpočtem nebo krátkým časovým rámcem.

Nativní vývoj je ideální pro projekty, které se zaměřují na plné využití možností iOS platformy, jako jsou výkonné aplikace nebo aplikace s~náročnějším uživatelským rozhraním. Také je vhodný pro aplikace, které potřebují maximální stabilitu a~bezpečnost, například bankovní aplikace nebo aplikace pro zpracování citlivých údajů. Pro vývojáře, kteří chtějí mít plnou kontrolu nad každým aspektem aplikace a~využít všech funkcí, které iOS nabízí, je nativní vývoj nejlepší volbou.

%---------------------------------------------------------------
\subsection{Cross-platformní vývoj}
%---------------------------------------------------------------

Cross-platformní vývoj se zaměřuje na tvorbu mobilních aplikací, které mohou běžet na více než jedné platformě (např. iOS a~Android) s~využitím jediného kódu a~jednoho vývojového prostředí. Tento přístup umožňuje vývojářům sdílet co nejvíce kódu mezi různými platformami, čímž snižuje náklady a~zjednodušuje správu aplikací pro různé zařízení.

Mezi hlavní výhody cross-platformního vývoje patří efektivita a~rychlost vývoje díky sdílení kódu mezi platformami. Vývojáři mohou využít frameworky jako \emph{Flutter} \cite{flutter-cross-platform}, \emph{React Native} \cite{react-native-cross-platform} nebo \emph{Xamarin} \cite{xamarin-cross-platform}, které umožňují psát kód v~populárních jazycích (např. \emph{JavaScript}, \emph{TypeScript}, \emph{Dart}, \emph{C\#}) a~následně ho spouštět na různých platformách. Tento přístup také usnadňuje aktualizace a~údržbu aplikací, protože změny se projeví na všech podporovaných platformách současně.

Nevýhody cross-platformního vývoje se mohou projevit ve snížené flexibilitě a~omezení přístupu k~některým pokročilým funkcím a~knihovnám specifickým pro danou platformu. Rovněž může docházet k~omezení rychlosti nebo výkonu aplikace v~porovnání s~nativními aplikacemi. Další nevýhodou může být závislost na externích frameworcích a~jejich aktualizacích.

Cross-platformní vývoj je ideální pro projekty, které vyžadují rychlou dostupnost na více platformách s~omezenými zdroji. Hodí se pro aplikace s~jednodušším uživatelským rozhraním, obsahově orientované aplikace (např. novinky, e-commerce), nebo pro firemní aplikace, které nevyžadují specifické funkce jednotlivých platforem. Tento přístup je také vhodný pro malé a~střední projekty, kde je důležitá rychlá a~efektivní tvorba aplikace pro více platforem.

%---------------------------------------------------------------
\subsection{Multi-platformní vývoj}
%---------------------------------------------------------------

Multi-platformní vývoj se liší od cross-platformního vývoje tím, že přímo kompiluje zdrojový kód do nativního kódu specifického pro každou platformu, místo aby spoléhal na vrstvu abstrakce nebo interpretaci. To znamená, že výsledná aplikace běží jako nativní aplikace bez vrstvy prostřednictvím frameworku. Typickým příkladem multi-platformního vývoje je použití jazyka jako \emph{Kotlin} pro Android a~\emph{Kotlin/Native} pro iOS, které se kompilují do nativního kódu pro obě platformy. Tato technologie se nazývá \emph{Kotlin Multiplatform} \cite{kotlin-multiplatform}.

Výhody multi-platformního vývoje zahrnují možnost sdílet větší část kódu mezi různými platformami a~zároveň dosahovat výkonnosti a~funkčnosti nativních aplikací. To znamená, že vývojáři mohou využívat specifické funkce a~knihovny pro každou platformu, aniž by se museli spoléhat na obecné abstraktní vrstvy. Tento přístup také umožňuje lepší optimalizaci výkonu a~přístup k~pokročilým funkcím operačních systémů.

Nevýhody multi-platformního vývoje mohou zahrnovat větší složitost a~náročnost vývoje oproti čistě cross-platformním frameworkům. Každá platforma může vyžadovat jiné postupy a~úpravy, ačkoli základní kód je sdílen. Některé specifické funkce nebo optimalizace pro konkrétní platformu mohou být obtížnější dosáhnout pomocí multi-platformního přístupu než s~nativním vývojem.

Multi-platformní vývoj je vhodný pro projekty, které vyžadují vysokou výkonnost a~přístup k~nativním funkcím a~knihovnám, ale zároveň potřebují sdílet co nejvíce kódu mezi platformami. Ideální je pro rozsáhlejší projekty nebo aplikace, které potřebují plnou integraci s~operačním systémem, ale zároveň chtějí minimalizovat duplicitu práce a~zjednodušit správu a~údržbu kódu. Multi-platformní vývoj je také vhodný pro situace, kdy je důležitá konzistence a~shoda funkcí mezi různými verzemi aplikace na různých platformách.

%---------------------------------------------------------------
\section{Backendová řešení pro mobilní aplikace}\label{backend}
%---------------------------------------------------------------

Backendová část mobilních aplikací hraje klíčovou roli v~poskytování dat, zpracování požadavků a~správě uživatelských účtů a~obsahu. Tato sekce se zaměřuje na analýzu různých backendových řešení a~technologií, které jsou vhodné pro podporu mobilních aplikací. Při výběru správného backendového řešení je důležité pochopit úlohu, kterou backend hraje v~kontextu mobilního prostředí a~identifikovat faktory, které ovlivňují výběr a~implementaci.

Backendová část mobilní aplikace zajišťuje komunikaci mezi klientem (mobilní aplikací) a~serverem, zpracování dat, autentizaci uživatelů, a~další potřebné funkce. Tento centrální prvek infrastruktury zabezpečuje efektivní a~spolehlivé fungování mobilních aplikací, přičemž umožňuje sdílení dat mezi různými zařízeními a~poskytuje uživatelům personalizovaný a~interaktivní zážitek.

Při výběru backendového řešení pro mobilní aplikaci je důležité zvážit několik faktorů. Patří mezi ně škálovatelnost a~výkon serveru, podpora pro bezpečnostní standardy a~autentizaci, možnosti správy uživatelských dat, a~také kompatibilita s~konkrétními požadavky a~technologiemi použitými ve frontendové části aplikace. Dále je důležité zvážit náklady na provoz, údržbu a~rozvoj backendového systému v~průběhu životního cyklu aplikace.

Existuje široká škála backendových technologií a~frameworků, které lze použít při vývoji mobilních aplikací. Od tradičních serverových platforem a~frameworků až po moderní cloudová řešení a~služby. Každá možnost má své vlastní výhody a~nevýhody. Důkladná analýza těchto možností je zásadní k~správnému výběru backendové architektury pro konkrétní mobilní aplikaci.

%---------------------------------------------------------------
\subsection{Backend as a~Service (BaaS) a~jiná delegovaná řešení}\label{baas}
%---------------------------------------------------------------

Tato podsekce se zaměřuje na možnosti, kdy jsou části backendové infrastruktury mobilní aplikace outsourcovány na externí poskytovatele služeb, jako je \emph{Firebase}, \emph{AWS Amplify} nebo \emph{Parse}. Tento přístup umožňuje vývojářům rychle nasadit backendovou část aplikace bez nutnosti spravovat a~udržovat vlastní serverovou infrastrukturu.

Jednou z~hlavních výhod BaaS je urychlení vývoje aplikace a~snížení nákladů a~komplexity provozování backendu. Poskytovatelé BaaS nabízejí hotová řešení pro autentizaci uživatelů, ukládání dat, notifikace, analýzu chování uživatelů a~řadu dalších funkcí, což umožňuje vývojářům zaměřit se více na samotnou funkcionalitu aplikace a~méně na infrastrukturální detaily.

Další výhodou je škálovatelnost a~výkon poskytovaných služeb, který může být optimalizován poskytovatelem a~automaticky přizpůsobován podle potřeb aplikace. Tento přístup je obzvláště užitečný pro malé a~střední projekty s~omezenými zdroji nebo pro projekty, které potřebují rychle nasadit MVP (Minimum Viable Product) bez investice do vlastní infrastruktury.

Nevýhodou použití BaaS může být omezení v~možnostech škálování a~přizpůsobení, zejména u~složitějších aplikací nebo projektů se specifickými požadavky na infrastrukturu. Dále může být problémem závislost na externím poskytovateli služeb a~jejich možná změna podmínek nebo dostupnosti.

Jako příklad BaaS lze uvést \emph{Firebase} od společnosti Google \cite{firebase}. Firebase poskytuje širokou škálu služeb, včetně realtime databáze, autentizace, cloudového úložiště, notifikací, analýzy a~mnoha dalších. Tato platforma je oblíbená mezi vývojáři pro svou jednoduchost a~širokou nabídku poskytovaných funkcí, což umožňuje rychlý vývoj a~nasazení mobilních aplikací s~minimálním úsilím na správu backendové infrastruktury.

Dalšími příklady BaaS služeb jsou \emph{AWS Amplify} od společnosti Amazon \cite{aws-amplify} a~\emph{Parse} \cite{parse}, který je open-source frameworkem pro tvorbu aplikací. Tyto služby nabízejí podobné funkce jako \emph{Firebase} a~umožňují vývojářům využít hotová řešení pro backendovou část svých mobilních aplikací bez nutnosti psát a~spravovat vlastní kód pro serverovou stranu.

%---------------------------------------------------------------
\subsection{Vývoj vlastního backendu}
%---------------------------------------------------------------

Tato sekce se zaměřuje na možnosti vytvoření a~implementace vlastního backendového řešení pro podporu mobilních aplikací. Tento přístup umožňuje vývojářům plnou kontrolu nad backendovou infrastrukturou a~její přizpůsobení specifickým požadavkům aplikace.

Jednou z~hlavních výhod vývoje vlastního backendu je možnost plného přizpůsobení infrastruktury a~funkcí podle konkrétních potřeb mobilní aplikace. Vývojáři mají kontrolu nad škálovatelností, bezpečností a~výkonem backendu, což je zvláště důležité pro aplikace s~vyššími nároky na bezpečnost, správu dat nebo specifické obchodní požadavky.

Další výhodou je snížená závislost na externích poskytovatelích služeb a~jejich změnách v~podmínkách či dostupnosti. Vlastní backend umožňuje také integraci s~existujícími systémy a~infrastrukturou v~organizaci, což může být podstatné pro firemní aplikace nebo projekty s~komplexními integračními požadavky.

Nevýhodou vývoje vlastního backendu může být zvýšená náročnost a~časová zátěž vývoje a~údržby. Vývojáři si musí sami implementovat všechny potřebné funkce backendu, včetně autentizace, ukládání dat, správy uživatelů a~dalších. To může vést ke zvýšeným nákladům a~časovému zpoždění při nasazení aplikace na trh.

Mezi nejpoužívanější řešení pro vývoj vlastního backendu patří frameworky jako \emph{Node.js} \cite{node-js} s~frameworky \emph{Express} \cite{node-js-express} nebo \emph{NestJS} \cite{nest-js} pro \emph{JavaScript}/\emph{TypeScript}, \emph{Ruby on Rails} \cite{ruby-on-rails} pro \emph{Ruby}, \emph{Django} \cite{django} pro \emph{Python} nebo \emph{Spring Boot} \cite{spring-boot} pro \emph{Javu}. Tyto frameworky nabízejí komplexní sadu nástrojů pro rychlý vývoj a~nasazení backendové aplikace s~podporou různých funkcí, včetně routování, databázového přístupu, autentizace a~dalších.

Každý z~těchto frameworků má své výhody a~nevýhody. Například \emph{Node.js} s~\emph{Express} je velmi populární pro svou rychlost a~flexibilitu, zatímco \emph{Spring Boot} je oblíbený pro svou robustnost a~škálovatelnost v~prostředí \emph{Java} \cite{spring-boot-vs-node-js}. Výběr správného frameworku závisí na preferencích vývojářů, technologických požadavcích a~cílech aplikace.

%---------------------------------------------------------------
\subsection{Databáze}
%---------------------------------------------------------------

Implementace databáze je dalším důležitým prvkem vývoje vlastního backendu pro mobilní aplikace, neboť poskytuje úložiště pro data, která aplikace zpracovává a~uchovává. Existuje několik možností pro implementaci databází v~rámci backendového prostředí, které se liší podle typu databáze a~potřeb aplikace.

Mezi nejpoužívanější řešení patří relační databáze, jako je \emph{Oracle Database} \cite{oracle-database} nebo \emph{PostgreSQL} \cite{postgresql}, a~také \emph{noSQL} databáze, jako je \emph{MongoDB} \cite{mongodb} nebo \emph{Redis} \cite{redis}. Relační databáze jsou založené na modelu relačního datového skladu s~použitím \emph{SQL} (Structured Query Language) pro manipulaci s~daty. Tyto databáze jsou vhodné pro aplikace, které vyžadují komplexní transakční operace a~silné zajištění integrity dat. Na druhou stranu \emph{noSQL} databáze jsou navrženy tak, aby byly flexibilnější a~lépe škálovatelné pro různé typy dat. Jsou ideální pro aplikace, které mají různorodá data a~vyžadují rychlý a~flexibilní přístup k~nim.

Výhody relačních databází zahrnují silnou konzistenci dat, vysokou integritu a~možnost provádět komplexní dotazy pomocí SQL. Na druhou stranu \emph{noSQL} databáze nabízejí vyšší škálovatelnost, flexibilitu datového modelu a~lepší výkon pro určité aplikace s~velkým objemem dat. \cite{fit-lecture-no-sql}

Výběr správného typu databáze závisí na specifických požadavcích a~charakteristikách projektu. Pro aplikace s~potřebou silné konzistence a~transakcí jsou vhodné relační databáze, zatímco pro aplikace s~velkým objemem různorodých dat a~vyššími požadavky na škálovatelnost jsou vhodnější \emph{noSQL} databáze.

Výběr implementace databázového řešení lze také outsourcovat na externí poskytovatele služeb BaaS. Jak bylo zmíněno v~sekci \ref{baas}, poskytovatelé jako \emph{Firebase} \cite{firebase} nebo \emph{AWS Amplify} \cite{aws-amplify} již implementují databázové řešení, které lze v~rámci těchto platforem využít. Existují také řešení, která poskytují pouze databázi jako externí řešení. Tato řešení jsou někdy nazývána DBaaS (Database as a~Service) \cite{mongodb-dbaas}. Příkladem tohoto řešení je například \emph{MongoDB Atlas} \cite{mongodb-atlas}, právě pro databázi \emph{MongoDB}.

V implementaci vlastního backendu pro mobilní aplikace je důležité pečlivě zvážit výběr databázového řešení, aby byly splněny požadavky na výkon, škálovatelnost a~bezpečnost aplikace. Zvážení výhod a~nevýhod jednotlivých databázových systémů pomůže zajistit optimální implementaci a~správu dat pro mobilní aplikaci pro zaznamenávání odpracovaného času.

%---------------------------------------------------------------
\section{Závěr analýzy}
%---------------------------------------------------------------

V~této kapitole byly probrány všechny informace potřebné pro navazující návrh mobilní aplikace pro zaznamenávání odpracovaného času. 

Co se týče domény problému, byla provedena rešerše spouštěčů měření času, ze které mohou být vhodně navrženy požadavky pro aplikaci na propojení s~nimi. Dále byla také provedena rešerše existujících systémů pro zaznamenávání odpracovaného času, ze které lze také vhodným způsobem navrhnout požadavky na integraci s~těmito systémy.

Dále se kapitola věnovala vývoji mobilních aplikací pro systém iOS, různým možnostem cross-platformního a~multi-platformního vývoje a~možnostem, jak přistupovat k~backendovým řešením podporující mobilní aplikace. Tyto informace by měly být plně postačující pro vhodný návrh funkcionalit aplikace, jejího uživatelského rozhraní, vnitřní architektury a~nástrojů k~tomu potřebným.











































